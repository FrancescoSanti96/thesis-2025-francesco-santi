\chapter{Progettazione}

La fase di progettazione ha avuto l'obiettivo di tradurre i requisiti individuati
nei capitoli precedenti in un'architettura scalabile, accessibile e coerente con
l'identità aziendale. In questa sezione vengono presentate le decisioni
architetturali chiave, il processo metodologico adottato e l'organizzazione
dell'ecosistema di landing pages.

\section{Decisioni architetturali}
Il punto di partenza era una landing page unica, insufficiente per comunicare i
diversi verticali aziendali e raccogliere dati utili a livello di marketing. La
progettazione ha quindi introdotto tre scelte fondamentali:

\begin{itemize}
  \item \textbf{Architettura multi-landing}: adozione di un monorepo Next.js con sei
  route dedicate (/it/, /it/tech-recruiting/, /it/ai-adoption/, etc.), così da garantire 
  posizionamento SEO ottimale, riuso di componenti comuni e semplicità di manutenzione. 
  
  \item \textbf{Strategia di rendering}: adozione di Static Site Generation (SSG) 
  con Incremental Static Regeneration (ISR, revalidation 1 ora), che consente di 
  bilanciare performance, costi infrastruttura e 
  necessità di aggiornamento contenuti. Scelta motivata da requisiti performance 
  critici per conversioni e traffico organico, con HTML statico servito da CDN 
  per crawler Google.
  
  \item \textbf{Design system condiviso}: definizione di una libreria di componenti 
  accessibili basata su ShadCN UI (Radix UI primitives) e Tailwind CSS, costruita 
  su linee guida visive aziendali. Garantisce accessibilità WCAG 2.1 AA nativa, 
  customizzazione completa e riutilizzo del codice UI tra le landing.
\end{itemize}

\section{Metodologia di progettazione e collaborazione} Il progetto di redesign è stato organizzato come una \textbf{macro-release annuale},
considerata strategica per l'azienda. La fase di progettazione non si è limitata
alla definizione di scelte tecniche, ma ha previsto un percorso strutturato di
analisi, collaborazione e validazione:

\begin{itemize}
  \item \textbf{Analisi iniziale}: revisione delle soluzioni esistenti e
  identificazione dei pattern da mantenere o eliminare.
  \item \textbf{Collaborazione interfunzionale}: raccolta dei requisiti in
  incontri con i team di prodotto, marketing e design, così da garantire
  allineamento tra esigenze di business e soluzioni tecniche.
  \item \textbf{Iterazioni e feedback}: rilascio preliminare delle nuove landing
  a un gruppo ristretto di stakeholder interni, con raccolta feedback e successive
  ottimizzazioni.
  \item \textbf{Approccio Agile}: gestione del lavoro in sprint bisettimanali,
  con daily standup, retrospettive e momenti di confronto dedicati alla revisione
  della qualità.
  \item \textbf{Progettazione responsive}: attenzione fin dall'inizio alle tre
  principali dimensioni di visualizzazione (desktop, tablet e mobile), per
  garantire una user experience coerente e ottimizzata.
\end{itemize}

Questa metodologia ha consentito di ridurre i rischi, migliorare la qualità
finale e assicurare che le landing rispondessero realmente ai bisogni degli utenti
e dell'azienda.

\section{Processo di design e design system}
Il design system è stato sviluppato in stretta collaborazione con il team UX/UI,
a partire da wireframe validati con il product team fino a mockup
high-fidelity realizzati in Figma. Il processo ha incluso:

\begin{itemize}
  \item definizione di design tokens comuni (colori, tipografia, spaziatura),
  condivisi tra Figma e configurazioni Tailwind CSS di progetto;
  \item creazione di componenti modulari con varianti responsive, per garantire 
  consistenza visiva e riutilizzo tra le 6 landing pages;
\end{itemize}

L'architettura componenti ha permesso di standardizzare elementi ricorrenti garantendo il riutilizzo di molto del codice UI tra le diverse landing.

Questo approccio ha permesso di ridurre i tempi di sviluppo, garantire coerenza tra design e implementazione e abilitare una rapida iterazione.

\section{Architettura delle landing pages}
L'ecosistema è stato progettato con una struttura comune di base
(\textit{navbar}, \textit{footer}) e con
personalizzazioni specifiche per target:

\subsection{Strategia di differenziazione}
Ogni landing è stata progettata con tone of voice e UX mirati:

\begin{itemize}
  \item \textbf{B2B enterprise} (Recruiting, AI Adoption, AI Engineering): tone of voice 
  consulenziale, layout pulito con whitespace generoso, form di contatto avanzati con 
  campi aziendali, CTA orientate a demo/consulenza ("Richiedi demo", "Contatta team");
  
  \item \textbf{B2C candidati} (Jobs): stile diretto e colorato, call-to-action
  immediate ("Cerca lavoro", "Candidati ora"), esperienza mobile-first (58\% utenti 
  da mobile), trasparenza salary con RAL sempre visibile;
  
  \item \textbf{Community} (Tech Community): linguaggio informale e tech-savvy, elementi 
  visuali orientati alla cultura developer, CTA engagement per Discord e newsletter 
  "Commit", social proof con 500k+ iscritti.
\end{itemize}

% [INSERIRE IMMAGINE 1: Hero Variants]
\begin{figure}[h]
  \centering
  \includegraphics[width=0.9\textwidth]{images/hero-comparison.png}
  \caption{Hero component - Varianti per target: B2B (Recruiting), B2C (Jobs), Community}
  \label{fig:hero-variants}
\end{figure}

\subsection{Landing principali}
Tra le landing principali si distinguono:

\begin{itemize}
  \item \textbf{Home Page} (/it/): hub centrale con overview dei 4 verticali e routing 
  intelligente verso servizi specifici;
  \item \textbf{AI Engineering} (/it/ai-engineering/): showcase framework proprietario 
  "Datapizza AI" con gallery progetti (Copiloti Sales/HR/Legal/Customer), target CTO 
  e IT Decision Makers;
  \item \textbf{Jobs Platform} (/jobs/): ricerca posizioni con filtri avanzati 
  (tech stack, location, RAL), Tech Buddy matching e zero ghosting policy.
\end{itemize}

% [INSERIRE IMMAGINE 2: AI Engineering Mockup]
\begin{figure}[h]
  \centering
  \includegraphics[width=0.8\textwidth]{images/aiengineering-desktop.png}
  \caption{Landing AI Engineering - Mockup finale desktop con framework showcase}
  \label{fig:ai-engineering}
\end{figure}

\section{Tracking e misurabilità}
Il sistema di tracking è stato progettato per migliorare l'implementazione 
precedentemente presente, supportando funnel di conversione specifici per ogni 
verticale. L'integrazione con Mixpanel consente di tracciare page view, 
interazioni e conversioni con granularità maggiore rispetto alla soluzione unica 
precedente.

\subsection{Event taxonomy e funnel}
L'architettura tracking implementa una tassonomia strutturata di eventi:
\begin{itemize}
  \item \textbf{Page events}: page\_view, landing\_loaded
  \item \textbf{Interaction events}: cta\_clicked, scroll\_depth, section\_viewed
  \item \textbf{Conversion events}: form\_submitted, newsletter\_signup
\end{itemize}

Il tracking permette di analizzare funnel di conversione differenziati per verticale:
\begin{itemize}
  \item \textbf{B2B}: landing\_view → cta\_click → form\_view → form\_submit
  \item \textbf{B2C Jobs}: landing\_view → search\_used → position\_click → application\_start
  \item \textbf{Community}: landing\_view → scroll\_50\% → newsletter\_click → newsletter\_submit
\end{itemize}

Questa segmentazione consente analisi specifiche per ottimizzazione delle 
conversioni e identificazione dei punti di drop-off nel customer journey.

\subsection{GDPR compliance}
Particolare attenzione è stata data alla conformità normativa europea con 
approccio privacy-by-design: consenso esplicito tramite cookie banner prima 
di inizializzazione Mixpanel, anonimizzazione automatica degli indirizzi IP 
e gestione opt-out utente.

\bigskip
In sintesi, la progettazione ha permesso di definire una base solida dal punto di
vista tecnico e metodologico, con decisioni architetturali motivate e framework 
scalabile, che ha guidato lo sviluppo e il dispiegamento presentati nei capitoli 
successivi.