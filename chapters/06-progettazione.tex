\chapter{Progettazione}

\section{Metodologia e organizzazione del lavoro}
\subsection{Approccio Agile}
Il progetto è stato sviluppato seguendo metodologia Agile con:
\begin{itemize}
  \item Sprint bisettimanali con obiettivi definiti
  \item Daily standup per allineamento quotidiano
  \item Sprint planning e retrospettive
  \item One-to-one periodici per feedback
\end{itemize}

Strumenti utilizzati:
\begin{itemize}
  \item \textbf{Jira}: gestione task e sprint planning
  \item \textbf{Discord/Google Meet}: comunicazione team
  \item \textbf{Figma}: design handoff e collaborazione
  \item \textbf{Notion}: documentazione e knowledge base
\end{itemize}

\subsection{Struttura del team}
Il team di prodotto era composto da:
\begin{itemize}
  \item 1 UX/UI Designer - Progettazione esperienza utente
  \item 1 Product Designer - Definizione requisiti di prodotto
  \item 1 Tech Lead (Giuseppe Renna) - Leadership tecnica e referente diretto
  \item 1 AI Engineer - Sviluppo soluzioni AI
  \item 4 Software Engineer - Sviluppo frontend e backend
\end{itemize}

Il ruolo personale: Software Engineer con responsabilità 70\% frontend e 30\% backend.

\subsection{Workflow di sviluppo}
\begin{itemize}
  \item \textbf{Version control}: Git con GitHub/GitLab
  \item \textbf{Branching strategy}: [TODO: specificare - feature branches, gitflow, ecc.]
  \item \textbf{Code review}: Pull/Merge Request con peer review obbligatoria
  \item \textbf{CI/CD}: Pipeline automazione per testing e deploy
  \item \textbf{Ambienti}: Development, Staging, Production
\end{itemize}

\subsection{Timeline e milestone}
\begin{itemize}
  \item \textbf{Gennaio 2025}: Onboarding, analisi codebase, technical debt reduction
  \item \textbf{Febbraio-Marzo 2025}: [TODO: Progettazione architettura, design system]
  \item \textbf{Marzo-Aprile 2025}: [TODO: Sviluppo landing pages]
  \item \textbf{Aprile-Maggio 2025}: [TODO: Testing, ottimizzazione, deploy]
  \item \textbf{Maggio-Giugno 2025}: [TODO: Monitoring, iterazioni]
\end{itemize}

\section{Analisi dei requisiti}
\subsection{Requisiti funzionali}
\begin{itemize}
  \item Sistema di routing per 6 landing pages specializzate
  \item Tracking eventi utente con Mixpanel (GDPR-compliant)
  \item Form di lead generation differenziati per verticale
  \item Integrazione newsletter "Commit"
  \item Supporto multilingua (italiano/inglese)
  \item Sistema CMS [TODO: specificare se headless CMS usato]
\end{itemize}

\subsection{Requisiti non funzionali}
\begin{itemize}
  \item \textbf{Performance}: Core Web Vitals ottimizzati, FCP < 2s, LCP < 2.5s
  \item \textbf{SEO}: Struttura semantica, sitemap XML, meta tags dinamici
  \item \textbf{Accessibilità}: Conformità WCAG 2.1 livello AA
  \item \textbf{Scalabilità}: Architettura modulare per nuovi verticali futuri
  \item \textbf{Sicurezza}: HTTPS, GDPR compliance, cookie consent management
  \item \textbf{Manutenibilità}: Design system documentato, componenti riutilizzabili
\end{itemize}

\section{Architettura frontend ad alto livello}
\subsection{Stack tecnologico}
\begin{itemize}
  \item \textbf{Framework}: Next.js (basato su React)
  \item \textbf{Rendering}: Static Site Generation (SSG) + Server-Side Rendering (SSR)
  \item \textbf{Styling}: Tailwind CSS per utility-first styling
  \item \textbf{Componenti}: ShadCN UI customizzato
  \item \textbf{State management}: React Query per API calls e caching
  \item \textbf{Type safety}: TypeScript per tutto il codebase
\end{itemize}

\subsection{Architettura applicativa}
[TODO: Inserire diagramma architettura]

L'architettura si articola su più layer:
\begin{itemize}
  \item \textbf{Presentation Layer}: Componenti React (pages, layouts, UI components)
  \item \textbf{Business Logic}: Custom hooks, utilities, helper functions
  \item \textbf{Data Layer}: API integration, Mixpanel events, form handling
  \item \textbf{Routing}: Next.js App Router con supporto i18n (/it/, /en/)
  \item \textbf{Build/Deploy}: [TODO: Vercel/AWS/altro?] con pipeline CI/CD
\end{itemize}

\subsection{Pattern architetturali}
\begin{itemize}
  \item \textbf{Component composition}: Riutilizzo attraverso composizione
  \item \textbf{Atomic Design}: Organizzazione atoms → molecules → organisms → templates
  \item \textbf{Container/Presentational}: Separazione logica da presentazione
  \item \textbf{Custom hooks}: Astrazione logica condivisa
  \item \textbf{Server/Client Components}: [TODO: se Next.js 13+ App Router]
\end{itemize}

\section{Design system e collaborazione}
\subsection{Processo di design e handoff}
Stretta collaborazione tra sviluppo e design:
\begin{itemize}
  \item \textbf{Design in Figma}: Wireframe, mockup high-fidelity, prototipi interattivi
  \item \textbf{Design review}: Sessioni di validazione UX/UI con team
  \item \textbf{Handoff strutturato}: Specs precise (spacing, colors, typography, assets)
  \item \textbf{Feedback loop}: Iterazione continua durante sprint
\end{itemize}

\subsection{Fondamenta visive}
\textbf{Palette colori}
\begin{itemize}
  \item Primary: [TODO: hex code - es. \#FF6B35]
  \item Secondary: [TODO: hex code]
  \item Neutral scale: [TODO: grigi per testo/backgrounds]
  \item Semantic colors: success, warning, error [TODO: hex codes]
  \item [TODO: Dark mode support?]
\end{itemize}

\textbf{Typography}
\begin{itemize}
  \item Font principale: [TODO: nome font - es. Inter, Poppins]
  \item Fallback: -apple-system, BlinkMacSystemFont, Segoe UI, system-ui
  \item Type scale: [TODO: h1-h6 sizes in rem/px]
  \item Line height: [TODO: es. 1.5 body, 1.2 headings]
  \item Font weights: Regular (400), Medium (500), Semibold (600), Bold (700)
\end{itemize}

\textbf{Spacing e layout}
\begin{itemize}
  \item Grid system: 8pt base unit
  \item Spacing scale: 4, 8, 16, 24, 32, 48, 64, 96px
  \item Breakpoints responsive:
    \begin{itemize}
      \item Mobile: [TODO: < Xpx]
      \item Tablet: [TODO: X-Ypx]
      \item Desktop: [TODO: > Ypx]
    \end{itemize}
  \item Container max-width: [TODO: es. 1280px]
  \item Border radius: [TODO: es. 4px, 8px, 16px]
\end{itemize}

\subsection{Libreria componenti}
\textbf{ShadCN UI come base}
\begin{itemize}
  \item Component library headless basata su Radix UI
  \item Customizzazione completa con Tailwind CSS
  \item Accessibilità (ARIA) integrata
  \item Componenti unstyled adattabili al brand
\end{itemize}

\textbf{Componenti implementati}
\begin{itemize}
  \item \textbf{Atoms}: Button (varianti), Input, Badge, Icon, Link
  \item \textbf{Molecules}: Card, FormField, NavItem, Dropdown, Modal
  \item \textbf{Organisms}: Header, Footer, Hero, Feature Grid, Testimonials
  \item \textbf{Templates}: Landing layouts riutilizzabili
\end{itemize}

\textbf{Variazioni per target}
\begin{itemize}
  \item \textbf{B2B}: Tone professionale, whitespace generoso, imagery corporate
  \item \textbf{B2C}: Tone friendly, elementi colorati, illustrazioni
  \item \textbf{Community}: Elementi tech, code snippets, badge/tags
\end{itemize}

\section{Architettura delle 6 landing pages}
\subsection{Struttura comune}
Ogni landing condivide:
\begin{itemize}
  \item Layout base: Header navigazione + Content + Footer
  \item Pattern CTA consistente
  \item Social proof integrato (500k+ community)
  \item Form contatto/iscrizione
  \item SEO optimization (meta, structured data)
\end{itemize}

\subsection{Specifiche per landing}
\textbf{1. Home Page (/it/)}
\begin{itemize}
  \item Obiettivo: Hub centrale, overview verticali
  \item Target: Tutti (router verso servizi)
  \item Sezioni: Hero + 4 verticali + Community + Newsletter
  \item CTA: "Scopri [servizio]", "Iscriviti newsletter"
\end{itemize}

\textbf{2. Tech Recruiting (/it/tech-recruiting/)}
\begin{itemize}
  \item Obiettivo: Lead generation aziende B2B
  \item Target: HR, Hiring Manager
  \item Sezioni: Hero + Processo 6 step + Talent pool + Case studies + Form contatto
  \item CTA: "Trova talenti", "Contattaci"
\end{itemize}

\textbf{3. Tech Community (/it/tech-media-agency/)}
\begin{itemize}
  \item Obiettivo: Acquisizione membri, iscrizione newsletter
  \item Target: Developer, tech enthusiast
  \item Sezioni: Hero + Newsletter "Commit" + Eventi + Discord + Contenuti
  \item CTA: "Iscriviti", "Unisciti Discord"
\end{itemize}

\textbf{4. AI Adoption (/it/ai-adoption/)}
\begin{itemize}
  \item Obiettivo: Lead generation servizi upskilling
  \item Target: C-level, HR, Management
  \item Sezioni: Hero + Evidenze scientifiche + Percorsi + Clienti + Form
  \item CTA: "Richiedi demo", "Contatta team"
\end{itemize}

\textbf{5. AI Engineering (/it/ai-engineering/)}
\begin{itemize}
  \item Obiettivo: Lead generation progetti custom AI
  \item Target: CTO, IT Decision Maker
  \item Sezioni: Hero + Framework proprietario + Showcase progetti + Tech stack + Form
  \item CTA: "Scopri framework", "Avvia progetto"
\end{itemize}

\textbf{6. Jobs Platform (/jobs/)}
\begin{itemize}
  \item Obiettivo: Applicazioni candidati
  \item Target: Developer, Data Scientist in cerca lavoro
  \item Sezioni: Hero + Ricerca posizioni + Trasparenza salary + Tech Buddy + Process
  \item CTA: "Cerca lavoro", "Carica CV"
\end{itemize}

\section{Sistema di tracking e analytics}
\subsection{Architettura Mixpanel}
\begin{itemize}
  \item Event taxonomy per verticale [TODO: esempi eventi chiave]
  \item User properties: source, landing, device, location
  \item Funnel tracking: page view → interaction → conversion
  \item GDPR compliance: cookie consent, opt-out, data retention
\end{itemize}

\subsection{Metriche chiave}
\begin{itemize}
  \item Page views e sessioni per landing
  \item Bounce rate e average session duration
  \item Conversion rate (form submit, newsletter signup, CTA clicks)
  \item Scroll depth e engagement
\end{itemize}

\subsection{Dashboard e analisi}
\begin{itemize}
  \item Redash per query SQL e dashboard custom
  \item Real-time monitoring eventi
  \item A/B testing framework [TODO: tool usato - Mixpanel Experiments?]
\end{itemize}

\section{Scalabilità e manutenibilità}
\subsection{Gestione versioni}
\begin{itemize}
  \item Semantic versioning del codebase
  \item Feature flags per rollout graduale
  \item Rollback strategy
\end{itemize}

\subsection{Performance optimization}
\begin{itemize}
  \item Image optimization con next/image (WebP, lazy loading)
  \item Code splitting automatico Next.js
  \item Asset optimization: minification, compression
  \item CDN per static assets [TODO: provider - Cloudflare/AWS?]
  \item Caching strategy: ISR (Incremental Static Regeneration) [TODO: revalidation time]
\end{itemize}

\subsection{Processo per nuove landing}
\begin{itemize}
  \item Template riutilizzabili da duplicare
  \item Component library documentata
  \item Guidelines design system
  \item Checklist SEO/Performance/Accessibility
\end{itemize}