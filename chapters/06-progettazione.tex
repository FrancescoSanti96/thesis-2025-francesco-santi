\chapter{Progettazione}

\section{Metodologia e organizzazione del lavoro}
\subsection{Approccio Agile}
Il progetto è stato sviluppato seguendo metodologia Agile con:
\begin{itemize}
  \item Sprint bisettimanali con obiettivi definiti e demo finale
  \item Daily standup (15 min) per allineamento quotidiano
  \item Sprint planning per prioritizzazione backlog
  \item Sprint retrospettive per miglioramento continuo
  \item One-to-one periodici per feedback personalizzato
\end{itemize}

Strumenti utilizzati:
\begin{itemize}
  \item \textbf{Jira}: gestione task, sprint planning, burndown charts
  \item \textbf{Discord/Google Meet}: comunicazione sincrona team
  \item \textbf{Figma}: design handoff, collaborazione designer-dev
  \item \textbf{Notion}: documentazione tecnica e knowledge base
\end{itemize}

\subsection{Struttura del team}
Il team di prodotto era composto da:
\begin{itemize}
  \item 1 UX/UI Designer - Progettazione esperienza utente e wireframe
  \item 1 Product Designer - Definizione requisiti e user stories
  \item 1 Tech Lead (Giuseppe Renna, referente diretto) - Leadership 
        tecnica, code review, decisioni architetturali
  \item 1 AI Engineer - Sviluppo soluzioni AI e integrazione modelli
  \item 4 Software Engineer - Sviluppo frontend e backend
\end{itemize}

\textbf{Ruolo personale:} Software Engineer con responsabilità 70\% 
frontend (React, Next.js, UI) e 30\% backend (Django, API).

\textbf{Collaborazione:} Interazione diretta con designer per handoff 
Figma, code review peer con altri engineer, allineamento quotidiano 
con Tech Lead.

\subsection{Workflow di sviluppo}
\begin{itemize}
  \item \textbf{Version control}: Git con GitHub per repository
  \item \textbf{Branching strategy}: Feature branches con merge su main 
        tramite Pull Request
  \item \textbf{Code review}: PR obbligatoria con almeno 1 approvazione 
        prima di merge
  \item \textbf{CI/CD}: GitHub Actions per testing automatico e deploy
  \item \textbf{Ambienti}: Development (locale), Staging, Production
  \item \textbf{Testing}: Manuale su staging prima di deploy production
\end{itemize}

\subsection{Timeline del progetto}
\begin{itemize}
  \item \textbf{Gennaio 2025}: Onboarding, analisi codebase esistente, 
        technical debt reduction iniziale
  \item \textbf{Febbraio 2025}: Progettazione architettura landing pages, 
        setup Next.js, design system base
  \item \textbf{Marzo 2025}: Sviluppo Home Page e prime landing 
        specializzate (Recruiting, Community)
  \item \textbf{Aprile 2025}: Sviluppo landing AI (Engineering, Adoption), 
        Jobs platform, integrazione Mixpanel
  \item \textbf{Maggio 2025}: Testing, ottimizzazione performance, 
        SEO optimization, fix bug
  \item \textbf{Giugno 2025}: Deploy production graduale, monitoring, 
        iterazioni post-launch
\end{itemize}

\section{Analisi dei requisiti}
\subsection{Requisiti funzionali}
\begin{itemize}
  \item Sistema di routing per 6 landing pages specializzate 
        (/it/ e /en/)
  \item Tracking eventi utente con Mixpanel GDPR-compliant
  \item Form di lead generation differenziati per verticale
  \item Integrazione newsletter "Commit" con sistema iscrizione
  \item Supporto multilingua italiano/inglese
  \item Responsive design mobile-first
\end{itemize}

\subsection{Requisiti non funzionali}
\begin{itemize}
  \item \textbf{Performance}: Core Web Vitals ottimizzati 
        (FCP < 2s, LCP < 2.5s, CLS < 0.1)
  \item \textbf{SEO}: Struttura semantica, sitemap XML, meta tags 
        dinamici, structured data JSON-LD
  \item \textbf{Accessibilità}: Conformità WCAG 2.1 livello AA, 
        navigazione keyboard, screen reader support
  \item \textbf{Scalabilità}: Architettura modulare per futuri 
        verticali, componenti riutilizzabili
  \item \textbf{Sicurezza}: HTTPS enforcement, security headers 
        (CSP, HSTS), protezione XSS
  \item \textbf{Manutenibilità}: Design system documentato, 
        convenzioni codifica chiare, testing coverage
\end{itemize}

\section{Architettura frontend}
\subsection{Stack tecnologico}
\begin{itemize}
  \item \textbf{Framework}: Next.js 14 con App Router
  \item \textbf{Rendering}: Static Site Generation (SSG) per landing 
        pages, Incremental Static Regeneration (ISR) per aggiornamenti
  \item \textbf{Styling}: Tailwind CSS per utility-first styling
  \item \textbf{Componenti}: ShadCN UI customizzato su Radix UI
  \item \textbf{State}: React Query per server state, React hooks per 
        UI state
  \item \textbf{Type safety}: TypeScript strict mode
\end{itemize}

\subsection{Architettura applicativa}
L'architettura si articola su layer separati:

\textbf{Presentation Layer}
\begin{itemize}
  \item Pages Next.js per routing (/it/, /en/)
  \item Layout condivisi (Header, Footer)
  \item Componenti UI atomici e composti
\end{itemize}

\textbf{Business Logic Layer}
\begin{itemize}
  \item Custom hooks per logica riutilizzabile
  \item Utilities e helper functions
  \item Form validation con Zod
\end{itemize}

\textbf{Data Layer}
\begin{itemize}
  \item API integration con React Query
  \item Mixpanel events tracking
  \item Form submission handlers
\end{itemize}

\textbf{Infrastructure Layer}
\begin{itemize}
  \item Next.js routing con App Router
  \item Build e deploy su Vercel
  \item CDN e asset optimization
\end{itemize}

\subsection{Pattern architetturali}
\begin{itemize}
  \item \textbf{Component composition}: Riutilizzo tramite composizione 
        invece di ereditarietà
  \item \textbf{Atomic Design}: Organizzazione componenti in 
        atoms → molecules → organisms → templates
  \item \textbf{Container/Presentational}: Separazione logica (hooks) 
        da presentazione (JSX)
  \item \textbf{Custom hooks}: Astrazione business logic condivisa
  \item \textbf{Server/Client Components}: Next.js 14 App Router per 
        ottimizzazione rendering
\end{itemize}

\section{Design system e collaborazione}
\subsection{Processo di design e handoff}
Collaborazione strutturata tra sviluppo e design:

\textbf{Fase 1: Design in Figma}
\begin{itemize}
  \item Wireframe low-fidelity per validazione struttura
  \item Mockup high-fidelity con brand identity
  \item Prototipi interattivi per testing UX
  \item Varianti responsive (mobile, tablet, desktop)
\end{itemize}

\textbf{Fase 2: Design Review}
\begin{itemize}
  \item Sessioni di validazione con stakeholder
  \item Feedback iterativo su usabilità
  \item Approvazione finale design
\end{itemize}

\textbf{Fase 3: Handoff Sviluppo}
\begin{itemize}
  \item Figma Dev Mode per export specifiche
  \item Specs precise: spacing, colors, typography, assets
  \item Annotazioni per stati interattivi (hover, focus, active)
  \item Export asset ottimizzati (SVG, WebP)
\end{itemize}

\textbf{Fase 4: Implementazione e Feedback}
\begin{itemize}
  \item Sviluppo componenti fedeli al design
  \item Review visivo su staging
  \item Iterazioni per pixel-perfect alignment
\end{itemize}

\subsection{Fondamenta visive}
\textbf{Palette colori}
\begin{itemize}
  \item Primary: \#FF6B35 (arancione Datapizza)
  \item Secondary: \#1A1A1A (nero)
  \item Neutral scale: \#FFFFFF, \#F5F5F5, \#E0E0E0, \#9E9E9E, \#424242
  \item Semantic colors: Success \#10B981, Warning \#F59E0B, 
        Error \#EF4444
  \item Background: \#FAFAFA (light), \#0F0F0F (dark - se implementato)
\end{itemize}

\textbf{Typography}
\begin{itemize}
  \item Font principale: Inter (sans-serif moderno)
  \item Fallback: -apple-system, BlinkMacSystemFont, Segoe UI, system-ui
  \item Type scale: H1 48px, H2 36px, H3 28px, H4 24px, H5 20px, 
        H6 18px, Body 16px
  \item Line height: 1.5 per body text, 1.2 per headings
  \item Font weights: Regular (400), Medium (500), Semibold (600), 
        Bold (700)
\end{itemize}

\textbf{Spacing e layout}
\begin{itemize}
  \item Grid system: 8pt base unit per spacing consistente
  \item Spacing scale Tailwind: 4px, 8px, 16px, 24px, 32px, 48px, 
        64px, 96px
  \item Breakpoints responsive:
    \begin{itemize}
      \item Mobile: < 768px
      \item Tablet: 768px - 1024px
      \item Desktop: > 1024px
    \end{itemize}
  \item Container max-width: 1280px con padding laterale 24px
  \item Border radius: 4px (small), 8px (medium), 16px (large)
\end{itemize}

\subsection{Libreria componenti}
\textbf{Base: ShadCN UI}
\begin{itemize}
  \item Component library headless basata su Radix UI
  \item Customizzazione completa tramite Tailwind CSS
  \item Accessibilità ARIA integrata nativamente
  \item Componenti unstyled adattabili al brand
\end{itemize}

\textbf{Componenti implementati}
\begin{itemize}
  \item \textbf{Atoms}: Button (varianti primary/secondary/ghost), 
        Input (text/email/tel), Badge, Icon (Lucide React), Link
  \item \textbf{Molecules}: Card, FormField (label + input + error), 
        NavItem, Dropdown, Modal, Accordion
  \item \textbf{Organisms}: Header (navigation), Footer (links + social), 
        Hero (varianti per landing), Feature Grid, Testimonials, 
        Form Lead Generation
  \item \textbf{Templates}: Landing Layout (header + content + footer), 
        Section Wrapper (spacing consistente)
\end{itemize}

\textbf{Variazioni per target}
\begin{itemize}
  \item \textbf{B2B}: Tone professionale, whitespace generoso, imagery 
        corporate, CTA "Richiedi demo"/"Contattaci"
  \item \textbf{B2C}: Tone friendly, colori più vivaci, illustrazioni, 
        CTA "Iscriviti"/"Inizia ora"
  \item \textbf{Community}: Elementi tech (code snippets), 
        badge/tags tecnologie, CTA "Unisciti"
\end{itemize}

\section{Architettura delle 6 landing pages}
\subsection{Struttura comune}
Ogni landing condivide elementi base:
\begin{itemize}
  \item Layout: Header navigation + Content sections + Footer
  \item Pattern CTA consistente (above fold + in-page + footer)
  \item Social proof (500k+ community) come trust signal
  \item Form contatto/iscrizione con validazione
  \item SEO: meta tags dinamici, structured data JSON-LD, 
        Open Graph per social
\end{itemize}

\subsection{Specifiche per landing}
\textbf{1. Home Page (/it/)}
\begin{itemize}
  \item Obiettivo: Hub centrale con overview 4 verticali, 
        router verso servizi
  \item Target: Tutti (visitatori generici)
  \item Sezioni: Hero + 4 verticali cards + Community stats + 
        Newsletter CTA
  \item CTA principali: "Scopri [servizio]", "Iscriviti newsletter"
\end{itemize}

\textbf{2. Tech Recruiting (/it/tech-recruiting/)}
\begin{itemize}
  \item Obiettivo: Lead generation aziende B2B
  \item Target: HR Manager, Hiring Manager, CEO
  \item Sezioni: Hero B2B + Processo 6 step + Talent pool showcase + 
        Case studies + Form contatto
  \item CTA: "Trova talenti", "Richiedi consulenza"
\end{itemize}

\textbf{3. Tech Community (/it/tech-media-agency/)}
\begin{itemize}
  \item Obiettivo: Acquisizione membri, iscrizione newsletter
  \item Target: Developer, tech enthusiast, studenti
  \item Sezioni: Hero community + Newsletter "Commit" + Eventi showcase + 
        Discord CTA + Contenuti recenti
  \item CTA: "Iscriviti newsletter", "Unisciti Discord"
\end{itemize}

\textbf{4. AI Adoption (/it/ai-adoption/)}
\begin{itemize}
  \item Obiettivo: Lead generation servizi upskilling B2B
  \item Target: C-level, HR Director, Management
  \item Sezioni: Hero enterprise + Evidenze scientifiche (grafici BCG) + 
        Percorsi formativi + Clienti + Form demo
  \item CTA: "Richiedi demo", "Contatta team"
\end{itemize}

\textbf{5. AI Engineering (/it/ai-engineering/)}
\begin{itemize}
  \item Obiettivo: Lead generation progetti custom AI
  \item Target: CTO, IT Decision Maker, Tech Lead
  \item Sezioni: Hero tech + Framework proprietario "Datapizza AI" + 
        Showcase progetti (Copiloti Sales/HR/Legal/Customer) + 
        Tech stack + Form contatto
  \item CTA: "Scopri framework", "Avvia progetto"
\end{itemize}

\textbf{6. Jobs Platform (/jobs/)}
\begin{itemize}
  \item Obiettivo: Applicazioni candidati
  \item Target: Developer, Data Scientist in cerca lavoro
  \item Sezioni: Hero candidati + Ricerca posizioni + Trasparenza 
        salary (RAL sempre visibile) + Tech Buddy + Process 
        "Zero ghosting"
  \item CTA: "Cerca lavoro", "Carica CV"
\end{itemize}

\section{Sistema di tracking e analytics}
\subsection{Architettura Mixpanel}
\textbf{Event Taxonomy}
\begin{itemize}
  \item \textbf{Page events}: page\_view, landing\_loaded
  \item \textbf{Interaction events}: cta\_clicked, scroll\_depth, 
        section\_viewed
  \item \textbf{Conversion events}: form\_submitted, newsletter\_signup, 
        download\_cv
  \item \textbf{Navigation events}: link\_clicked, menu\_opened
\end{itemize}

\textbf{User Properties}
\begin{itemize}
  \item Source: UTM parameters (utm\_source, utm\_medium, utm\_campaign)
  \item Device: mobile/tablet/desktop
  \item Location: country, city (IP-based)
  \item Landing page: prima pagina visitata
  \item User type: new/returning
\end{itemize}

\textbf{Funnel Tracking}
\begin{itemize}
  \item B2B: landing\_view → cta\_click → form\_view → form\_submit
  \item B2C Jobs: landing\_view → search\_used → position\_click → application\_start
  \item Community: landing\_view → scroll\_50\% → newsletter\_click → 
        newsletter\_submit
\end{itemize}

\textbf{GDPR Compliance}
\begin{itemize}
  \item Cookie consent banner obbligatorio prima di tracking
  \item Opt-out automatico se consent negato
  \item Data retention: 12 mesi, poi cancellazione automatica
  \item IP anonymization attiva
\end{itemize}

\subsection{Metriche chiave monitorate}
\begin{itemize}
  \item Page views e unique visitors per landing
  \item Bounce rate e average session duration
  \item Conversion rate per funnel (form submit, newsletter signup)
  \item Scroll depth medio e engagement rate
  \item Source/medium distribution (organic/paid/direct/referral)
\end{itemize}

\subsection{Dashboard e analisi}
\begin{itemize}
  \item Redash per query SQL custom su database analytics
  \item Dashboard real-time per monitoring eventi Mixpanel
  \item Report settimanali automatici via email
  \item Alert su anomalie (spike/drop conversion > 30\%)
\end{itemize}

\section{Scalabilità e performance}
\subsection{Gestione versioni}
\begin{itemize}
  \item Semantic versioning del codebase (v1.0.0, v1.1.0, v2.0.0)
  \item Feature flags per rollout graduale nuove funzionalità
  \item Git tags per versioni stabili in produzione
  \item Rollback strategy: git revert + redeploy < 5 minuti
\end{itemize}

\subsection{Performance optimization}
\begin{itemize}
  \item \textbf{Image optimization}: next/image con WebP, lazy loading, 
        responsive images
  \item \textbf{Code splitting}: Automatico Next.js per route, 
        dynamic import per componenti pesanti
  \item \textbf{Asset optimization}: Minification JS/CSS, Gzip 
        compression, font subsetting
  \item \textbf{CDN}: CloudFront per static assets, edge caching
  \item \textbf{Caching strategy}: ISR con revalidation 3600s per 
        landing pages
\end{itemize}

\subsection{Processo per nuove landing}
Framework riutilizzabile per velocizzare sviluppo:
\begin{itemize}
  \item Template base con layout comune (Header/Footer)
  \item Component library già pronta (Hero, Feature Grid, Form, etc.)
  \item Design system documentato con varianti
  \item Mixpanel tracking pre-configurato
  \item Checklist SEO/Performance/Accessibility standardizzata
\end{itemize}

Tempo stimato per nuova landing: 1-2 sprint (2-4 settimane).