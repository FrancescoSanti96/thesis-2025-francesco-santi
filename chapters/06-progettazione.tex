\chapter{Progettazione}

La fase di progettazione ha avuto l’obiettivo di tradurre i requisiti individuati
nei capitoli precedenti in un’architettura scalabile, accessibile e coerente con
l’identità aziendale. In questa sezione vengono presentate le decisioni
architetturali chiave, il processo metodologico adottato e l’organizzazione
dell’ecosistema di landing pages.

\section{Decisioni architetturali}
Il punto di partenza era una landing page unica, insufficiente per comunicare i
diversi verticali aziendali e raccogliere dati utili a livello di marketing. La
progettazione ha quindi introdotto tre scelte fondamentali:

\begin{itemize}
  \item \textbf{Architettura multi-landing}: adozione di un monorepo con sei
  route dedicate, così da garantire posizionamento SEO, riuso di componenti
  comuni e semplicità di manutenzione.
  \item \textbf{Strategia di rendering}: adozione di pagine statiche con
  aggiornamento incrementale (SSG + ISR), che consente di bilanciare performance,
  costi e necessità di aggiornamento dei contenuti.
  \item \textbf{Design system condiviso}: definizione di una libreria di
  componenti accessibili e riutilizzabili, costruita su linee guida visive
  aziendali.
\end{itemize}

\section{Metodologia di progettazione e collaborazione}
Il progetto di redesign è stato organizzato come una \textbf{macro-release annuale},
considerata strategica per l’azienda. La fase di progettazione non si è limitata
alla definizione di scelte tecniche, ma ha previsto un percorso strutturato di
analisi, collaborazione e validazione:

\begin{itemize}
  \item \textbf{Analisi iniziale}: revisione delle soluzioni esistenti e
  identificazione dei pattern da mantenere o eliminare.
  \item \textbf{Collaborazione interfunzionale}: raccolta dei requisiti in
  incontri con i team di prodotto, marketing e design, così da garantire
  allineamento tra esigenze di business e soluzioni tecniche.
  \item \textbf{Iterazioni e feedback}: rilascio preliminare delle nuove landing
  a un gruppo ristretto di stakeholder interni, con raccolta feedback e successive
  ottimizzazioni.
  \item \textbf{Approccio Agile}: gestione del lavoro in sprint bisettimanali,
  con daily standup, retrospettive e momenti di confronto dedicati alla revisione
  della qualità.
  \item \textbf{Progettazione responsive}: attenzione fin dall’inizio alle tre
  principali dimensioni di visualizzazione (desktop, tablet e mobile), per
  garantire una user experience coerente e ottimizzata.
\end{itemize}

Questa metodologia ha consentito di ridurre i rischi, migliorare la qualità
finale e assicurare che le landing rispondessero realmente ai bisogni degli utenti
e dell’azienda.

\section{Processo di design}
Il design system è stato sviluppato in stretta collaborazione con il team UX/UI,
a partire da wireframe validati con il product team fino a mockup
high–fidelity realizzati in Figma. Il processo ha incluso:

\begin{itemize}
  \item definizione di design tokens comuni (colori, tipografia, spaziatura),
  condivisi tra Figma e configurazioni di progetto;
  \item creazione di componenti modulari con varianti responsive, per garantire
  consistenza visiva su desktop e mobile;
  \item handoff strutturato verso lo sviluppo, con esportazione automatica di
  specifiche e asset ottimizzati.
\end{itemize}

Questo approccio ha permesso di ridurre i tempi di sviluppo, garantire coerenza
tra design e implementazione e abilitare una rapida iterazione.

\section{Architettura delle landing pages}
L’ecosistema è stato progettato con una struttura comune di base
(\textit{header}, \textit{hero}, sezioni contenuto, \textit{footer}) e con
personalizzazioni per target specifici:

\begin{itemize}
  \item \textbf{B2B enterprise}: tone of voice consulenziale, layout pulito,
  form di contatto avanzati;
  \item \textbf{B2C candidati}: stile diretto e colorato, call-to-action
  immediate, esperienza mobile–first;
  \item \textbf{Community}: linguaggio informale e tech-savvy, elementi visuali
  orientati alla cultura developer.
\end{itemize}

Tra le landing principali si distinguono la Home Page, come hub di accesso ai
verticali, la sezione AI Engineering dedicata al framework proprietario, e la
Jobs Platform con funzionalità di ricerca e matching.

\section{Tracking e misurabilità}
Il sistema di tracking è stato progettato nativamente nell’architettura per
supportare funnel di conversione specifici per ogni verticale. L’integrazione con
Mixpanel consente di tracciare page view, interazioni e conversioni, con
attributi relativi al tipo di utente e alla sorgente di traffico.

Particolare attenzione è stata data alla conformità normativa:
consenso esplicito tramite cookie banner, anonimizzazione degli indirizzi IP,
meccanismi di opt–out e data retention limitata a 12 mesi.

\section{Scalabilità e riuso}
Per supportare l’aggiunta di nuove landing è stato predisposto un framework
riutilizzabile che comprende:

\begin{itemize}
  \item template base già dotato di layout, componenti e tracking integrato;
  \item checklist di verifica su SEO, performance, accessibilità e funnel;
  \item documentazione del design system a supporto dello sviluppo.
\end{itemize}

Grazie a questo approccio, la creazione di una nuova landing può essere stimata
in 1–2 sprint, mantenendo consistenza con l’ecosistema esistente.

\bigskip
In sintesi, la progettazione ha permesso di definire una base solida dal punto di
vista tecnico e metodologico, che ha guidato lo sviluppo e il dispiegamento
presentati nei capitoli successivi.
