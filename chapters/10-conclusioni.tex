\chapter{Conclusioni e sviluppi futuri}

\section{Sintesi dei risultati}
Il progetto di redesign delle landing pages ha trasformato la presenza web di Datapizza
da un'unica pagina generica a un ecosistema di sei landing specializzate. 
Questo ha permesso di posizionare chiaramente i verticali aziendali, differenziare i messaggi
per target specifici e introdurre strumenti concreti di marketing data-driven.

L'architettura realizzata ha dimostrato efficacia su tre dimensioni principali:
\begin{itemize}
  \item \textbf{Tecnica}: sistema scalabile e performante, design system modulare, 
        possibilità di sviluppare nuove landing in 2--4 settimane.
  \item \textbf{Business}: incremento di lead qualificati, funnel misurabili per verticale,
        riduzione del bounce rate e posizionamento credibile come \textit{AI Transformation Company}.
  \item \textbf{Operativa}: framework di tracking avanzato, capace di abilitare decisioni di marketing informate
        e ottimizzazioni continue delle conversioni.
\end{itemize}

\section{Crescita professionale}
Il progetto ha rappresentato un passaggio chiave nella mia formazione,
permettendo di integrare le conoscenze apprese in ambito universitario con la gestione
di un prodotto reale in un contesto dinamico e competitivo.

Le principali competenze sviluppate includono:
\begin{itemize}
  \item \textbf{Autonomia e ownership}: gestione completa del ciclo di vita software, 
        dalla raccolta dei requisiti fino al deploy e al monitoring.
  \item \textbf{Problem solving avanzato}: gestione rapida di criticità tecniche 
        (GDPR compliance, compatibilità cross-browser, ottimizzazione delle performance).
  \item \textbf{Collaborazione interfunzionale}: interazione continua con team 
        multidisciplinare (designer, product manager, altri developer) e demo 
        interne per validation funzionalità.
  \item \textbf{Approccio data-driven}: sviluppo orientato alle metriche di conversione
        e miglioramento basato su analytics reali.
  \item \textbf{Padronanza tecnologie moderne}: esperienza pratica approfondita 
        su stack altamente richiesto nel mercato (React, Next.js, TypeScript, 
        AWS, Docker), con responsabilità dirette su codebase production.
\end{itemize}

Accanto al progetto principale, i contributi trasversali a Datapizza Jobs, Company
e al gestionale interno hanno fornito una base di crescita significativa su frontend, 
backend e concetti trasversali allo sviluppo software. L'ambiente stimolante, 
caratterizzato da continui scambi, feedback costruttivi e code review rigorose, 
ha accelerato l'apprendimento ben oltre quanto possibile in contesto accademico.

\section{Limiti e criticità}
Nonostante i risultati raggiunti, il lavoro presenta alcune considerazioni:
\begin{itemize}
  \item A/B testing: pattern già utilizzato in altri prodotti aziendali, 
        potrebbe essere implementato sulle landing per confronti tra verticali 
        o validazione iterazioni future.
  \item Dipendenza infrastrutturale: scelte architetturali iniziali vincolano 
        evoluzioni future, trade-off normale nella progettazione software che 
        richiede manutenzione continua.
  \item Timeline ristretta: focus su delivery ha limitato tempo per ottimizzazioni 
        avanzate deployment e studi approfonditi su edge cases.
\end{itemize}

\section{Sviluppi futuri}

Il progetto landing pages rappresenta una macro-release in un ecosistema software 
più ampio. Gli sviluppi futuri variano in base alle priorità aziendali e ai diversi 
prodotti su cui il team lavora.

\subsection{Evoluzione landing pages}

Con l'arrivo di un team marketing dedicato, l'approccio allo sviluppo evolverà 
verso iterazioni più misurate e validate:

\begin{itemize}
  \item \textbf{Testing e validazione}: implementazione di test A/B strutturati 
        per validare ipotesi prima di rilasci definitivi. Possibilità di testare 
        varianti anche esternamente all'ecosistema landing per misurare impatto 
        prima dell'integrazione.
  
  \item \textbf{Rilascio nuovi verticali}: approccio meno invasivo rispetto alla 
        macro-release iniziale, con micro-iterazioni continue e collaborazione 
        più frequente con team design/product per aggiustamenti rapidi.
  
  \item \textbf{Ottimizzazione conversioni}: espansione tracking con heatmap e 
        session recording per identificare friction points e ottimizzare user 
        journey data-driven.
  
  \item \textbf{Espansione internazionale}: completamento traduzioni multilingua 
        per apertura mercati esteri.
\end{itemize}

\subsection{Evoluzione ecosistema tecnico}

L'ecosistema richiede manutenzione continua e aggiornamenti per rimanere 
competitivo:

\begin{itemize}
  \item \textbf{Aggiornamenti framework}: mantenimento stack tecnologico 
        aggiornato (Next.js, React, TypeScript) per beneficiare di nuove 
        funzionalità, performance improvements e security patches.
  
  \item \textbf{Ottimizzazione performance}: monitoraggio continuo Core Web Vitals 
        e implementazione ottimizzazioni progressive per mantenere standard elevati.
  
  \item \textbf{Valutazione monorepo}: considerata migrazione verso architettura 
        monorepo con component library condivisa, già adottata su altri prodotti 
        aziendali (Jobs, Company), per valutare benefici di riuso codice e 
        consistency anche sulle landing pages.
  
  \item \textbf{Gestione contenuti}: integrazione CMS headless in roadmap per 
        autonomia team marketing, pur mantenendo controllo tecnico su 
        performance e SEO.
\end{itemize}

\section{Riflessione finale}
L'esperienza in Datapizza ha confermato che la progettazione software non è solo
un esercizio tecnico, ma uno strumento per generare valore concreto per il business.
Ho maturato competenze tecnologiche avanzate e, soprattutto, un mindset orientato al prodotto:
comprendere non solo \textit{come} realizzare una soluzione, ma anche \textit{perché}
e quale impatto produce.

La prosecuzione del rapporto lavorativo a tempo pieno dopo il periodo documentato
testimonia la rilevanza del contributo fornito. Questa esperienza rappresenta una base solida
per la carriera professionale, con competenze immediatamente spendibili e la consapevolezza
che l'ingegneria del software è chiamata a bilanciare rigore tecnico e pragmatismo,
per generare innovazione sostenibile.