\chapter{Conclusioni e sviluppi futuri}

\section{Sintesi dei risultati}
Il progetto di redesign delle landing pages ha trasformato la presenza web aziendale 
da un'unica landing generica in un ecosistema di sei pagine specializzate e 
ottimizzate per target diversi. Questo ha permesso di posizionare chiaramente i 
verticali aziendali, differenziare i messaggi per target specifici e introdurre 
strumenti concreti, orientati al marketing data-driven.

L'architettura realizzata ha dimostrato efficacia su tre dimensioni principali:
\begin{itemize}
  \item \textbf{Tecnica}: sistema scalabile e performante, design system modulare, 
        possibilità di sviluppare nuove landing in 2--4 settimane.
  \item \textbf{Business}: incremento di lead qualificati, funnel misurabili per verticale,
        riduzione della percentuale di utenti che abbandonano la pagina senza interagire 
        e posizionamento credibile come \textit{AI Transformation Company}.
  \item \textbf{Operativa}: framework di tracking avanzato, capace di abilitare decisioni di marketing informate
        e ottimizzazioni continue delle conversioni.
\end{itemize}

\section{Crescita professionale}
Il progetto ha rappresentato un passaggio chiave nella mia formazione,
permettendo di integrare le conoscenze apprese in ambito universitario con la gestione
di un prodotto reale in un contesto dinamico e competitivo.

\medskip
Le principali competenze sviluppate riguardano tre aree fondamentali. 

Sul piano 
dell'autonomia, ho gestito il ciclo completo di sviluppo software dalla raccolta 
requisiti al deployment e monitoring in produzione. Il problem solving avanzato 
si è manifestato nella risoluzione rapida di criticità tecniche come problemi di 
compatibilità cross-browser nei form di lead generation e ottimizzazione delle 
performance. La collaborazione interfunzionale con designer, product manager e 
altri sviluppatori ha richiesto comunicazione continua e validazione iterativa 
delle funzionalità attraverso demo interne.

\medskip
L'approccio data-driven è diventato centrale nel processo di sviluppo, orientando 
le decisioni verso metriche di conversione misurabili e ottimizzazione basata su 
analytics reali. L'esperienza pratica su tecnologie moderne altamente richieste 
dal mercato (React, Next.js, TypeScript, AWS, Docker) con responsabilità dirette 
su codebase in produzione ha consolidato competenze immediatamente spendibili nel 
contesto professionale.

\medskip
Oltre al progetto principale, i contributi trasversali a Datapizza Jobs, Company
e al gestionale interno hanno fornito una base di crescita significativa su frontend, 
backend e concetti trasversali allo sviluppo software. L'ambiente stimolante, 
caratterizzato da continui scambi, feedback costruttivi e code review rigorose, 
ha accelerato l'apprendimento ben oltre quanto possibile in contesto accademico.

\section{Sviluppi futuri}

Il progetto landing pages rappresenta una macro-release (rilascio principale 
trimestrale) in un ecosistema software più ampio che continuerà ad evolversi 
in base alle priorità aziendali e alle necessità dei diversi verticali. Alcuni 
aspetti del progetto attuale, pur funzionali agli obiettivi iniziali, possono 
essere ulteriormente ottimizzati per supportare la crescita futura.

\subsection{Ottimizzazione continua e testing}

L'architettura attuale garantisce solidità e performance, ma può beneficiare 
di iterazioni progressive. L'implementazione di test A/B strutturati, pattern 
già utilizzato in altri prodotti aziendali, permetterà di validare ipotesi 
prima di rilasci definitivi e confrontare l'efficacia di differenti approcci 
per verticale. Con l'arrivo di un team marketing dedicato, sarà possibile 
testare varianti anche esternamente all'ecosistema per misurare l'impatto 
prima dell'integrazione completa.

L'ottimizzazione delle conversioni beneficerà dell'espansione del tracking 
con heatmap e session recording per individuare punti di attrito nei flussi 
di navigazione. Il rilascio di nuovi verticali seguirà un approccio iterativo 
con micro-release e collaborazione frequente con i team design e product per 
aggiustamenti rapidi.

\subsection{Evoluzione e manutenzione tecnologica}

L'architettura attuale, funzionale agli obiettivi di time-to-market, può essere 
evoluta per maggiore scalabilità. È in valutazione una migrazione verso 
architettura monorepo con component library condivisa, già adottata su altri 
prodotti aziendali (Jobs, Company), per massimizzare il riuso del codice e 
garantire consistency tra tutte le piattaforme.

L'ecosistema richiede manutenzione continua per rimanere competitivo. Il 
mantenimento dello stack tecnologico aggiornato (React, TypeScript, Next.js) 
garantirà l'accesso a nuove funzionalità, miglioramenti di performance e patch 
di sicurezza. Il monitoraggio continuo dei Core Web Vitals guiderà 
l'implementazione di ottimizzazioni progressive per mantenere standard elevati 
di user experience.

\medskip
Questi sviluppi, pianificati in accordo con le priorità aziendali e l'evoluzione 
del mercato, consentiranno di consolidare ulteriormente il valore dell'ecosistema 
landing pages nel medio-lungo termine.

\section{Riflessione finale}
L'esperienza in Datapizza ha confermato che la progettazione software non è solo
un esercizio tecnico, ma uno strumento per generare valore concreto per il business.
Ho maturato competenze tecnologiche avanzate e, soprattutto, un mindset orientato al prodotto:
comprendere non solo \textit{come} realizzare una soluzione, ma anche \textit{perché}
e quale impatto produce.

La prosecuzione del rapporto lavorativo a tempo pieno dopo il periodo documentato
testimonia la rilevanza del contributo fornito. Questa esperienza rappresenta una base solida
per la carriera professionale, con competenze immediatamente spendibili e la consapevolezza
che l'ingegneria del software è chiamata a bilanciare rigore tecnico e pragmatismo,
per generare innovazione sostenibile.