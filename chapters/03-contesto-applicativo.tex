\chapter{Contesto applicativo}

\section{Situazione iniziale delle landing pages}
Prima del progetto di redesign, la presenza web di Datapizza consisteva in un'unica landing page che accorpava genericamente community e recruiting, senza alcuna rappresentazione dei verticali AI (AI Engineering e AI Adoption).

\subsection{Problematiche riscontrate}
Questa soluzione, adeguata quando l'azienda contava 10-20 persone, presentava limitazioni critiche per un'azienda cresciuta a 60+ dipendenti con quattro verticali distinti:

\textbf{Mancanza di posizionamento}
\begin{itemize}
  \item Impossibilità di comunicare value proposition specifiche per servizi molto diversi
  \item Confusione tra target B2B enterprise (AI consulting) e B2C (candidati)
  \item Assenza totale dei nuovi verticali AI lanciati nel 2023-2025
\end{itemize}

\textbf{Scalabilità e gestione}
\begin{itemize}
  \item Architettura monolitica difficile da aggiornare
  \item Impossibilità di testare messaggi diversi per verticale
  \item Nessuna differenziazione UX per diversi customer journey
\end{itemize}

\textbf{Tracking e ottimizzazione}
\begin{itemize}
  \item Assenza di analytics strutturato per conversioni
  \item Impossibilità di tracciare comportamento utente per servizio
  \item Nessuna conformità GDPR per tracking
  \item Mancanza di dati per decisioni data-driven
\end{itemize}

\textbf{Performance}
\begin{itemize}
  \item [DA INTEGRARE: metriche Lighthouse/Core Web Vitals pre-redesign]
  \item Design system non strutturato
  \item SEO non ottimizzato per diversi servizi
\end{itemize}

\section{Necessità di refactor}
La crescita aziendale e l'evoluzione dei servizi hanno reso necessaria una trasformazione radicale dell'architettura web.

\subsection{Drivers strategici}
\begin{itemize}
  \item Rappresentare 4 verticali distinti (Recruiting, Community, AI Engineering, AI Adoption)
  \item Differenziare comunicazione B2B enterprise vs B2C candidati
  \item Abilitare tracking GDPR-compliant per ottimizzazione conversioni
  \item Supportare crescita da 10-20 a 60+ persone e 100+ clienti
\end{itemize}

\subsection{Obiettivi del redesign}
\textbf{Architettura}
\begin{itemize}
  \item 6 landing pages specializzate (Home, Tech Recruiting, Community, AI Adoption, AI Engineering, Jobs)
  \item Design system modulare e riutilizzabile
  \item Routing Next.js ottimizzato con SSR/SSG
\end{itemize}

\textbf{Tracking e Analytics}
\begin{itemize}
  \item Implementazione Mixpanel per eventi utente GDPR-compliant
  \item Dashboard Redash per analisi funnel
  \item Mappatura customer journey per verticale
\end{itemize}

\textbf{Performance e UX}
\begin{itemize}
  \item Ottimizzazione Core Web Vitals
  \item Esperienza utente dedicata per target (B2B vs B2C)
  \item SEO strutturato per servizi specifici
\end{itemize}

\section{Valore strategico per l'azienda}
\textbf{Impatto business}
\begin{itemize}
  \item Landing pages come canale principale acquisizione lead
  \item Riduzione CAC attraverso funnel ottimizzati
  \item Abilitazione campagne marketing verticale-specifiche
  \item [DA INTEGRARE: ROI atteso/conversion rate target]
\end{itemize}

\textbf{Posizionamento}
\begin{itemize}
  \item Comunicazione chiara come "AI Transformation Company"
  \item Credibilità enterprise per servizi AI
  \item Differenziazione competitiva (community 500k+ come trust signal)
\end{itemize}

\textbf{Scalabilità}
\begin{itemize}
  \item Framework pronto per nuovi verticali futuri
  \item Sistema A/B testing per iterazione continua
  \item Architettura manutenibile dal team
\end{itemize}