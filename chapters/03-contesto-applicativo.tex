\chapter{Contesto applicativo}

\section{Situazione iniziale delle landing pages}
Prima del progetto di redesign, la presenza web di Datapizza consisteva 
in un'unica landing page che accorpava genericamente community e 
recruiting, senza rappresentazione dei verticali AI (AI Engineering e 
AI Adoption).

\subsection{Problematiche riscontrate}
Questa soluzione, adeguata per un'azienda di 10-20 persone, presentava 
limitazioni critiche per una realtà cresciuta a 60+ dipendenti con 
quattro verticali distinti:

\textbf{Mancanza di posizionamento}
\begin{itemize}
  \item Impossibilità di comunicare value proposition specifiche
  \item Confusione tra target B2B enterprise e B2C candidati
  \item Assenza verticali AI lanciati nel 2023-2025
\end{itemize}

\textbf{Scalabilità e gestione}
\begin{itemize}
  \item Architettura monolitica difficile da aggiornare
  \item Impossibilità di testare messaggi diversi per verticale
  \item Nessuna differenziazione UX per customer journey
\end{itemize}

\textbf{Tracking e ottimizzazione}
\begin{itemize}
  \item Assenza di analytics strutturato
  \item Impossibilità di tracciare comportamento per servizio
  \item Mancanza dati per decisioni data-driven
\end{itemize}

\textbf{Performance e tecnico}
\begin{itemize}
  \item Accessibilità e conformità WCAG non garantite
  \item Design system non strutturato
  \item Codice non modulare e difficile da mantenere
  \item SEO e performance non ottimizzati
\end{itemize}

\section{Necessità di refactor}
La crescita aziendale e l'evoluzione dei servizi hanno reso necessaria 
una trasformazione radicale dell'architettura web.

\subsection{Drivers strategici del cambiamento}
\begin{itemize}
  \item \textbf{Crescita organizzativa}: da 10 a 60+ persone 
        richiede presenza web professionale
  \item \textbf{Nuovi verticali}: AI Engineering e AI Adoption 
        necessitano posizionamento dedicato
  \item \textbf{Differenziazione target}: servizi B2B vs B2C richiedono 
        comunicazione specifica
  \item \textbf{Espansione commerciale}: 100+ aziende clienti necessitano 
        canali acquisizione ottimizzati
\end{itemize}

\subsection{Limitazioni tecniche dell'esistente}
\begin{itemize}
  \item Impossibilità di implementare tracking granulare per servizio
  \item Difficoltà nel mantenere performance con contenuti crescenti
  \item Mancanza separazione responsabilità nel codice
\end{itemize}

\section{Valore strategico del progetto}
Il progetto rappresenta un investimento strategico per supportare la 
crescita aziendale:

\textbf{Impatto sul business}
\begin{itemize}
  \item Principale canale di acquisizione contatti qualificati (lead)
  \item Riduzione costo acquisizione cliente (CAC)
  \item Campagne marketing mirate per verticale
  \item Supporto strategia commerciale multi-target
\end{itemize}

\textbf{Posizionamento competitivo}
\begin{itemize}
  \item Comunicazione identità "AI Transformation Company"
  \item Credibilità enterprise per servizi AI
  \item Community 500k+ come trust signal
  \item Professionalizzazione immagine aziendale
\end{itemize}

\textbf{Scalabilità futura}
\begin{itemize}
  \item Architettura pronta per nuovi verticali
  \item Framework riutilizzabile
  \item Iterazione rapida basata su dati
  \item Manutenibilità a lungo termine
\end{itemize}