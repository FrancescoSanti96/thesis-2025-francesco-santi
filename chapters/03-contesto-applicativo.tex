\chapter{Contesto applicativo}

\section{Situazione iniziale delle landing pages}
Prima del progetto di redesign, la presenza web di Datapizza consisteva in un'unica landing page che accorpava genericamente community e recruiting, senza alcuna rappresentazione dei verticali AI (AI Engineering e AI Adoption).

\subsection{Problematiche riscontrate}
Questa soluzione, adeguata quando l'azienda contava 10-20 persone, presentava limitazioni critiche per un'azienda cresciuta a 60+ dipendenti con quattro verticali distinti:

\textbf{Mancanza di posizionamento}
\begin{itemize}
  \item Impossibilità di comunicare value proposition specifiche per servizi molto diversi
  \item Confusione tra target B2B enterprise (AI consulting) e B2C (candidati)
  \item Assenza totale dei nuovi verticali AI lanciati nel 2023-2025
\end{itemize}

\textbf{Scalabilità e gestione}
\begin{itemize}
  \item Architettura monolitica difficile da aggiornare
  \item Impossibilità di testare messaggi diversi per verticale
  \item Nessuna differenziazione UX per diversi customer journey
\end{itemize}

\textbf{Tracking e ottimizzazione}
\begin{itemize}
  \item Assenza di analytics strutturato per conversioni
  \item Impossibilità di tracciare comportamento utente per servizio
  \item Nessuna conformità GDPR per tracking
  \item Mancanza di dati per decisioni data-driven
\end{itemize}

\textbf{Performance e tecnico}
\begin{itemize}
  \item [DA INTEGRARE: metriche Lighthouse/Core Web Vitals pre-redesign]
  \item Design system non strutturato
  \item SEO non ottimizzato per diversi servizi
  \item Codice non modulare e difficile da mantenere
\end{itemize}

\section{Necessità di refactor}
La crescita aziendale e l'evoluzione dei servizi hanno reso necessaria una trasformazione radicale dell'architettura web.

\subsection{Drivers strategici del cambiamento}
\begin{itemize}
  \item \textbf{Crescita organizzativa}: da 10-20 a 60+ persone richiede presenza web professionale
  \item \textbf{Nuovi verticali}: AI Engineering e AI Adoption necessitano posizionamento dedicato
  \item \textbf{Differenziazione target}: servizi B2B enterprise vs B2C richiedono comunicazione specifica
  \item \textbf{Espansione commerciale}: 100+ aziende clienti necessitano canali di acquisizione ottimizzati
\end{itemize}

\subsection{Limitazioni tecniche dell'esistente}
\begin{itemize}
  \item Impossibilità di implementare tracking granulare per servizio
  \item Difficoltà nel mantenere performance ottimali con contenuti crescenti
  \item Assenza di framework per testing A/B
  \item Mancanza di separazione delle responsabilità nel codice
\end{itemize}

\section{Valore strategico del progetto}
Il progetto rappresenta un investimento strategico fondamentale per supportare la crescita aziendale:

\textbf{Impatto sul business}
\begin{itemize}
  \item Le landing pages costituiscono il principale canale di acquisizione lead
  \item Necessità di ridurre il costo di acquisizione cliente (CAC)
  \item Abilitazione di campagne marketing mirate per verticale
  \item Supporto alla strategia commerciale multi-target
\end{itemize}

\textbf{Posizionamento competitivo}
\begin{itemize}
  \item Comunicazione chiara dell'identità "AI Transformation Company"
  \item Credibilità enterprise per servizi AI ad alto valore
  \item Differenziazione attraverso community 500k+ come trust signal
  \item Professionalizzazione dell'immagine aziendale
\end{itemize}

\textbf{Scalabilità futura}
\begin{itemize}
  \item Architettura pronta per nuovi verticali di business
  \item Framework riutilizzabile per future iniziative
  \item Capacità di iterazione rapida basata su dati
  \item Manutenibilità a lungo termine del sistema
\end{itemize}