\chapter{Contesto applicativo}
\sloppypar
\section{Situazione iniziale}
Prima del progetto di redesign, la presenza web di Datapizza consisteva 
in un'unica landing page che accorpava genericamente community e 
recruiting, senza rappresentazione dei verticali AI (AI Engineering e 
AI Adoption).

\subsection{Problematiche riscontrate}
Questa soluzione, adeguata per un'azienda di 10-20 persone, presentava 
limitazioni critiche per una realtà cresciuta a 60+ dipendenti con 
quattro verticali distinti.

\medskip
Dal punto di vista del \textbf{posizionamento}, era impossibile comunicare 
value proposition specifiche per ciascun servizio, generando confusione 
tra il target B2B enterprise (aziende in cerca di talenti o consulenze AI) 
e il target B2C (candidati in cerca di opportunità). Inoltre, i nuovi 
verticali AI lanciati tra il 2023 e il 2025 non avevano alcuna 
rappresentazione dedicata.

\medskip
Per quanto riguarda \textbf{scalabilità e gestione}, l'architettura 
monolitica rendeva difficile aggiornare i contenuti e impediva di testare 
messaggi diversi per ogni verticale. Non esisteva alcuna differenziazione 
nell'esperienza utente (UX) in base al customer journey specifico.

\medskip
Sul fronte \textbf{tracking e ottimizzazione}, mancava completamente un 
sistema di analytics strutturato. Non era possibile tracciare il 
comportamento degli utenti per singolo servizio né raccogliere dati utili 
per prendere decisioni informate (approccio data-driven).

\medskip
Infine, dal punto di vista \textbf{tecnico e di performance}, 
l'accessibilità e la conformità agli standard WCAG non erano garantite, 
il design system non era strutturato, il codice risultava poco modulare 
e difficile da mantenere, e le ottimizzazioni SEO e di performance erano 
inadeguate.

\section{Necessità di intervento}
Le problematiche descritte nella sezione precedente non rappresentavano 
limitazioni isolate, ma ostacolavano concretamente la crescita aziendale. 
La mancanza di strumenti per misurare l'efficacia di ciascun verticale 
impediva al team marketing di ottimizzare le campagne e al team sales 
di comprendere quali messaggi funzionassero meglio per i diversi target. 
Inoltre, i nuovi servizi AI Engineering e AI Adoption, lanciati tra il 
2023 e il 2025, necessitavano di una comunicazione dedicata che la 
landing generica non poteva fornire. La crescita da 10-20 a oltre 60 
persone richiedeva una presenza web più professionale, capace di 
trasmettere credibilità a clienti enterprise.

\subsection{Fattori di spinta}
In questo contesto, i principali fattori che hanno spinto verso il 
redesign completo sono stati:

\begin{itemize}
  \item \textbf{Crescita organizzativa}: da 10 a 60+ persone 
        richiede presenza web professionale e scalabile
  
  \item \textbf{Nuovi verticali}: AI Engineering e AI Adoption 
        necessitano posizionamento dedicato per acquisire clienti 
        enterprise
  
  \item \textbf{Differenziazione target}: servizi B2B enterprise e 
        B2C candidati richiedono comunicazione e user journey 
        completamente diversi
  
  \item \textbf{Espansione commerciale}: 100+ aziende clienti 
    necessitano canali acquisizione ottimizzati e misurabili
\end{itemize}

\section{Valore strategico del progetto}
Il progetto rappresenta un investimento strategico per supportare la 
crescita aziendale su tre dimensioni fondamentali.

\medskip
Dal punto di vista dell'\textbf{impatto sul business}, le landing pages 
costituiscono il principale canale di acquisizione di contatti qualificati 
(lead). Il redesign mira a ridurre il costo di acquisizione cliente (CAC) 
attraverso campagne marketing mirate per ciascun verticale, supportando 
così la strategia commerciale multi-target dell'azienda.

\medskip
Per quanto riguarda il \textbf{posizionamento competitivo}, le nuove 
landing permettono di comunicare in modo efficace l'identità di 
"AI Transformation Company", costruendo credibilità enterprise per i 
servizi AI più avanzati. La community di oltre 500.000 iscritti diventa 
un elemento di fiducia (trust signal) immediato, contribuendo alla 
professionalizzazione dell'immagine aziendale.

\medskip
Infine, in termini di \textbf{scalabilità futura}, l'architettura 
progettata è pronta per accogliere nuovi verticali senza stravolgimenti 
strutturali. Il sistema di componenti modulari e il design system condiviso permettono 
di sviluppare una nuova landing page in 1-2 settimane riutilizzando la 
maggior parte del codice, riducendo significativamente il time-to-market 
per nuovi servizi. Inoltre, 
il tracking strutturato consente di analizzare rapidamente i dati raccolti 
e ottimizzare le landing in base ai risultati concreti.