\chapter{Contesto aziendale}
\sloppypar
\section{Descrizione dell'azienda}
Datapizza è una scale-up innovativa con sede legale in Via Giuseppe Ripamonti 190, 20141 Milano (MI), fondata nell'ottobre 2022 con la mission di rendere l'Italia competitiva nel settore tech attraverso soluzioni avanzate e servizi mirati.

L'azienda nasce nel 2021 ad oggi, dopo 4 anni, conta più di 60 persone e si conferma in forte crescita.

Datapizza si distingue per un approccio integrato unico nel panorama italiano, operando simultaneamente su quattro verticali strategici complementari:

\begin{itemize}
  \item \textbf{Tech Recruiting}: connessione tra aziende e talenti tech, con oltre 50.000 professionisti registrati. Lanciato ad aprile 2023, rappresenta il ponte tra persone tecniche e aziende che vogliono essere competitive grazie alla tecnologia.
  
  \item \textbf{Tech Community}: con oltre 500.000 iscritti, costituisce la più grande community tech italiana. Punto di riferimento per notizie, tendenze e approfondimenti tecnologici, genera oltre 6 milioni di impression mensili
  
  \item \textbf{AI Engineering}: sviluppo di agenti AI specializzati e soluzioni custom per trasformare i workflow aziendali. Include un framework proprietario (``Datapizza AI'') per l'orchestrazione di modelli, lanciato nel 2025
  
  \item \textbf{AI Adoption}: percorsi personalizzati di trasformazione interna per potenziare la workforce aziendale. Lanciato a maggio 2023, risponde alla necessità di guidare l'adozione dell'AI in tutta l'organizzazione
\end{itemize}

\section{Dimensioni e crescita}
L'azienda ha vissuto una crescita esponenziale in soli tre anni. Partita come community di appassionati nel 2021, è diventata società nell'ottobre 2022 con una squadra iniziale di 10-20 persone. L'arrivo di ChatGPT nel novembre 2022 ha accelerato la trasformazione, portando l'AI nelle mani di centinaia di milioni di persone.

La crescita si è articolata in due fasi corrispondenti all'evoluzione del mercato AI:
\begin{itemize}
  \item \textbf{Wave 1 (2023-2024)}: fase di sperimentazione con tool 
        AI generici. Le aziende iniziavano a porsi domande concrete 
        sull'utilizzo pratico di questi strumenti. Datapizza risponde 
        lanciando Jobs (per aumentare il talento tecnico interno) e AI 
        Adoption (per formare l'organizzazione)
  
  \item \textbf{Wave 2 (2025+)}: integrazione AI nei sistemi core 
        aziendali. Non bastano più tool generici ma serve co-progettazione 
        con l'azienda. Datapizza ha lanciato AI Engineering con framework 
        proprietario e approccio technology-first
\end{itemize}

\section{Struttura del team}
Durante il periodo documentato (3 gennaio - 20 giugno 2025), sono stato inserito nel team di prodotto, responsabile dello sviluppo e dell'evoluzione dei prodotti software aziendali. Il team era strutturato con competenze multidisciplinari:

\begin{itemize}
  \item \textbf{1 UX/UI Designer} - Progettazione esperienza utente e wireframe
  \item \textbf{1 Product Designer} - Definizione requisiti di prodotto e user stories
  \item \textbf{1 Tech Lead} Leadership tecnica, decisioni architetturali, code review
  \item \textbf{1 AI Engineer} - Sviluppo soluzioni AI e integrazione modelli
  \item \textbf{4 Software Engineer} - Sviluppo frontend e backend
\end{itemize}

Il mio ruolo è stato quello di \textbf{Software Engineer} con responsabilità prevalentemente frontend (70\%) e secondariamente backend (30\%).