\chapter{Contesto aziendale}

\section{Descrizione dell'azienda}
Datapizza è una startup innovativa con sede legale in Via Giuseppe Ripamonti 190, 20141 Milano (MI), fondata nell'ottobre 2022 con la mission di rendere l'Italia competitiva nel settore tech attraverso soluzioni avanzate e servizi mirati.

L'azienda nasce da un contesto particolare: nel 2021, quando si parlava di Machine Learning e Deep Learning, pochi avevano compreso che questi sistemi erano già ovunque, dentro i prodotti delle Big Tech e nelle decisioni quotidiane. Un gruppo di "nerd ossessionati" - tra cui i fondatori Pierpaolo e Alessandro - aveva creato una community per unire chi voleva approfondire, imparare e costruire in questo ambito.

Datapizza si distingue per un approccio integrato unico nel panorama italiano, operando simultaneamente su quattro verticali strategici complementari:

\begin{itemize}
  \item \textbf{Tech Recruiting}: connessione tra aziende e talenti tech, con oltre 50.000 professionisti registrati. Lanciato ad aprile 2023, rappresenta il ponte tra persone tecniche e aziende che vogliono essere competitive grazie alla tecnologia
  
  \item \textbf{Tech Community}: con oltre 500.000 iscritti, costituisce la più grande community tech italiana. Punto di riferimento per notizie, tendenze e approfondimenti tecnologici, genera oltre 6 milioni di impression mensili
  
  \item \textbf{AI Engineering}: sviluppo di agenti AI specializzati e soluzioni custom per trasformare i workflow aziendali. Include un framework proprietario ("Datapizza AI") per l'orchestrazione di modelli, lanciato nel 2025
  
  \item \textbf{AI Adoption}: percorsi personalizzati di trasformazione interna per potenziare la workforce aziendale. Lanciato a maggio 2023, risponde alla necessità di guidare l'adozione dell'AI in tutta l'organizzazione
\end{itemize}

\section{Dimensioni e crescita}
L'azienda ha vissuto una crescita esponenziale in soli tre anni. Partita come community di appassionati nel 2021, è diventata società nell'ottobre 2022 con una squadra iniziale di 10-20 persone. L'arrivo di ChatGPT nel novembre 2022 ha accelerato la trasformazione, portando l'AI nelle mani di centinaia di milioni di persone.

La crescita si è articolata in due fasi corrispondenti all'evoluzione del mercato AI:
\begin{itemize}
  \item \textbf{Wave 1 (2023-2024)}: fase di sperimentazione con tool AI generici. Le aziende iniziavano a porsi domande concrete sull'utilizzo pratico di questi strumenti. Datapizza risponde lanciando Jobs (per aumentare il talento tecnico interno) e AI Adoption (per formare l'organizzazione)
  
  \item \textbf{Wave 2 (2025+)}: integrazione AI nei sistemi core aziendali. Non bastano più tool generici ma serve co-progettazione con l'azienda. Datapizza lancia AI Engineering con framework proprietario e approccio technology-first
\end{itemize}

Oggi l'azienda conta oltre 60 dipendenti in costante crescita. I numeri chiave evidenziano il successo dell'approccio integrato:
\begin{itemize}
  \item \textbf{50.000+} talenti tech registrati
  \item \textbf{500.000+} iscritti alla community
  \item \textbf{100+} aziende partner/clienti
  \item \textbf{60+} dipendenti (da iniziali 10-20)
\end{itemize}

\section{Contributi trasversali alle aree aziendali}
Durante l'esperienza in Datapizza, ho contribuito trasversalmente alle diverse aree di business, acquisendo una visione completa dell'ecosistema aziendale.

\subsection{Tech Recruiting}
Nell'area recruiting, ho lavorato attivamente su:
\begin{itemize}
  \item \textbf{Datapizza Jobs}: piattaforma lato candidati con sviluppo di feature per matching e ottimizzazioni UX
  \item \textbf{Datapizza Company}: piattaforma lato aziende per pubblicazione posizioni e gestione candidature
  \item \textbf{Technical debt reduction}: standardizzazione API calls con React Query, migrazione UI verso ShadCN per consistenza, riduzione bundle size del 15\%
\end{itemize}

\subsection{Tech Community}
Per la community, il contributo principale è stato lo sviluppo della landing page dedicata, progettata per:
\begin{itemize}
  \item Acquisizione nuovi membri attraverso value proposition chiara
  \item Iscrizione alla newsletter "Commit" con contenuti settimanali su AI
  \item Promozione eventi, hackathon (es. "Hackapizza") e iniziative community
  \item Showcase dei 500k+ iscritti come social proof
\end{itemize}

\subsection{AI Engineering e AI Adoption}
Per i verticali AI, ho sviluppato landing pages specializzate che comunicano servizi complessi a target enterprise:
\begin{itemize}
  \item \textbf{AI Engineering}: presentazione framework proprietario, showcase progetti (Copiloti Sales, HR, Legal, Customer), tone tecnico per CTO e IT Decision Makers
  \item \textbf{AI Adoption}: comunicazione percorsi upskilling con evidenze scientifiche (+40\% qualità, +25\% velocità secondo Boston Consulting Group), approccio people-first per HR e Management
\end{itemize}

\subsection{Gestionale Interno}
Ho partecipato alla fase iniziale del gestionale aziendale interno:
\begin{itemize}
  \item Setup dell'architettura base del progetto
  \item Definizione del routing e delle convenzioni di sviluppo
  \item Condivisione best practice con il team
\end{itemize}

\section{L'importanza delle landing pages performanti}
In un contesto di crescita rapida come quello di Datapizza - passata da 10-20 a 60+ persone in tre anni - disporre di landing pages performanti e scalabili è diventato strategicamente essenziale.

Come evidenziato dal CEO: "Crediamo che con un alto grado di talento tecnico (valorizzato nel modo giusto) al giorno d'oggi la tecnologia fa la differenza nel business." Questa filosofia si traduce nella necessità di una presenza web che:

\begin{itemize}
  \item \textbf{Comunichi posizionamento chiaro}: con quattro verticali molto diversi (da recruiting a AI consulting), ogni servizio necessita di un messaggio dedicato per il proprio target
  
  \item \textbf{Abiliti acquisizione efficiente}: ridurre il costo di acquisizione cliente attraverso funnel ottimizzati e conversion rate elevati
  
  \item \textbf{Supporti scalabilità del marketing}: possibilità di testare rapidamente nuove proposte di valore e campagne mirate per ciascun verticale
  
  \item \textbf{Fornisca ottimizzazione data-driven}: tracciamento preciso del comportamento utente per miglioramenti continui basati su dati reali
  
  \item \textbf{Garantisca brand consistency}: design system condiviso che assicura coerenza visiva pur mantenendo flessibilità
\end{itemize}

La necessità di innovazione digitale si è resa evidente quando l'unica landing page esistente - che accorpava genericamente community e recruiting, senza rappresentare i nuovi verticali AI - è diventata inadeguata per la complessità dell'offerta aziendale attuale.