\chapter{Contesto aziendale}
\sloppypar
\section{Descrizione dell'azienda}
Datapizza è una scale-up innovativa con sede legale in Via Giuseppe Ripamonti 190, 20141 Milano (MI), fondata nell'ottobre 2022 con la mission di rendere l'Italia competitiva nel settore tech attraverso soluzioni avanzate e servizi mirati.

L'azienda nasce nel 2021 e ad oggi, dopo 4 anni, conta più di 60 persone e si conferma in forte crescita.

Datapizza si distingue per un approccio integrato unico nel panorama 
italiano. Questa strategia si basa su quattro verticali strategici 
complementari che operano in sinergia:


\begin{itemize}
  \item \textbf{Tech Recruiting}: connessione tra aziende e talenti tech, con oltre 50.000 professionisti registrati. Lanciato ad aprile 2023, rappresenta il ponte tra persone tecniche e aziende che vogliono essere competitive grazie alla tecnologia.
  
  \item \textbf{Tech Community}: con oltre 500.000 iscritti, costituisce la più grande community tech italiana. Punto di riferimento per notizie, tendenze e approfondimenti tecnologici, genera oltre 6 milioni di impression mensili.
  
  \item \textbf{AI Engineering}: sviluppo di agenti AI specializzati e soluzioni custom per trasformare i workflow aziendali. Include un framework proprietario (``Datapizza AI``) per l'orchestrazione di modelli, lanciato nel 2025.
  
  \item \textbf{AI Adoption}: percorsi personalizzati di trasformazione interna per potenziare la workforce aziendale. Lanciato a maggio 2023, risponde alla necessità di guidare l'adozione dell'AI in tutta l'organizzazione.
\end{itemize}

Questa struttura integrata permette all'azienda di offrire un supporto 
completo alle organizzazioni che vogliono crescere nel tech: dal 
recruiting dei talenti giusti, alla formazione delle persone, fino allo 
sviluppo di soluzioni AI personalizzate.

\section{Dimensioni e crescita}
L'azienda ha vissuto una crescita esponenziale in soli tre anni. Partita 
come community di appassionati nel 2021, è diventata società nell'ottobre 
2022 con una squadra iniziale di 10-20 persone. L'arrivo di ChatGPT nel 
novembre 2022 ha accelerato la trasformazione, portando l'AI nelle mani 
di centinaia di milioni di persone.

\medskip 
La crescita si è articolata in due fasi corrispondenti all'evoluzione del 
mercato AI. La prima fase, che copre il periodo 2023-2024, è stata 
caratterizzata dalla sperimentazione con tool AI generici. Le aziende 
iniziavano a porsi domande concrete sull'utilizzo pratico di questi 
strumenti: chi se ne doveva occupare? Come integrarli nei processi 
esistenti? Datapizza ha risposto a queste esigenze lanciando Jobs, per 
aumentare il talento tecnico interno alle organizzazioni, e AI Adoption, 
per formare le persone e favorire l'adozione consapevole della tecnologia.

\medskip 
La seconda fase, iniziata nel 2025 e tuttora in corso, vede 
l'integrazione dell'AI nei sistemi core aziendali. Non bastano più tool 
generici: serve co-progettazione con l'azienda per creare soluzioni su 
misura che rispondano a esigenze specifiche. Per questa nuova sfida, 
Datapizza ha lanciato AI Engineering con framework proprietario e 
approccio technology-first, mantenendo sempre l'essere umano al centro 
del processo decisionale.

\section{Struttura del team}
Durante il periodo documentato (3 gennaio - 20 giugno 2025), sono stato 
inserito nel team di prodotto, responsabile dello sviluppo e dell'evoluzione 
delle piattaforme software aziendali. La struttura era multidisciplinare, 
con competenze necessarie per gestire progetti complessi dall'ideazione 
al rilascio.

\medskip 
La componente design era coperta da un UX/UI Designer, che si occupava 
della progettazione dell'esperienza utente e dei wireframe, e da un 
Product Designer che definiva i requisiti di prodotto traducendo le 
esigenze di business in specifiche tecniche. Il coordinamento tecnico era 
affidato a un Tech Lead che guidava le decisioni architetturali e 
supervisionava la qualità del codice attraverso attività di code review. 
Completavano il team un AI Engineer dedicato allo sviluppo di soluzioni 
di intelligenza artificiale e integrazione dei modelli, e quattro Software 
Engineer, me compreso, focalizzati sullo sviluppo frontend e backend delle 
applicazioni.

\medskip 
Il mio ruolo è stato quello di Software Engineer con responsabilità 
prevalentemente frontend (70\%) e secondariamente backend (30\%). Questa 
suddivisione mi ha permesso di acquisire competenze approfondite sullo 
sviluppo dell'interfaccia utente e sull'esperienza d'uso, mantenendo al 
contempo una visione completa dello stack tecnologico aziendale.