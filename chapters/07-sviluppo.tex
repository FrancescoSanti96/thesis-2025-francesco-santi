\chapter{Sviluppo}

\section{Processo di sviluppo}
Lo sviluppo delle landing pages è stato organizzato con un approccio
iterativo basato su sprint bisettimanali. Ogni sprint prevedeva
pianificazione delle priorità, sviluppo incrementale, revisione del
codice e demo finale con stakeholder per raccogliere feedback.  
Il workflow ha incluso code review obbligatoria su ogni Pull Request,
testing continuo su ambiente \textit{staging} e rilascio progressivo
per singola landing.

\section{Fasi di implementazione}
Il progetto è stato articolato in quattro fasi principali:

\begin{itemize}
  \item \textbf{Fase 1 – Architettura e design system}: setup del progetto
  Next.js con App Router, configurazione multilingua e definizione del design
  system base con Tailwind e ShadCN.
  \item \textbf{Fase 2 – Sviluppo landing pages}: implementazione progressiva
  delle sei landing, a partire dalla Home Page (validazione del design system) fino alla Jobs Platform.
  \item \textbf{Fase 3 – Tracking e ottimizzazione}: integrazione completa di
  Mixpanel, funnel di conversione per verticale, cookie consent GDPR e
  ottimizzazioni SEO e performance.
  \item \textbf{Fase 4 – Deploy e monitoring}: rilascio graduale in produzione,
  monitoraggio in tempo reale con Mixpanel e Vercel Analytics, e documentazione finale.
\end{itemize}

\section{Sviluppo delle landing}
Le landing sono state sviluppate seguendo priorità strategiche. La
\textbf{Home Page} è stata la prima, hub centrale con routing verso i verticali
e social proof aziendali.  
Sono seguite le pagine \textbf{Tech Recruiting} e \textbf{Community}, rispettivamente orientate a lead generation B2B e alla crescita della community.  
Successivamente sono state realizzate \textbf{AI Adoption} e \textbf{AI Engineering}, focalizzate sui servizi enterprise, e infine la \textbf{Jobs Platform}, progettata mobile–first con ricerca avanzata e trasparenza salariale.  

\section{Implementazione architettura}
Come progettato nel capitolo precedente, l'implementazione ha seguito le decisioni 
architetturali chiave: monorepo Next.js con 6 route dedicate, SSG con ISR per 
performance ottimali, e design system basato su ShadCN UI.

Dal punto di vista tecnico, lo sviluppo ha previsto tre aree principali:

\textbf{Architettura componenti} – è stata realizzata una libreria di componenti 
riutilizzabili con TypeScript per garantire consistenza e riuso tra le 6 landing pages.

\textbf{Routing e rendering} – le landing sono state generate staticamente
(SSG) con revalidazione oraria (ISR), con gestione automatica di meta tags e
sitemap per l'ottimizzazione SEO.  

\textbf{Tracking e compliance} – l'SDK di Mixpanel è stato integrato con 
inizializzazione condizionale dopo consenso esplicito e anonimizzazione IP 
per conformità GDPR.

\section{Sfide affrontate e soluzioni}
Durante lo sviluppo sono emerse diverse difficoltà tecniche risolte con 
soluzioni specifiche:

\textbf{Cross-browser compatibility} – Safari mostrava incompatibilità con 
regex JavaScript per validazione form. La soluzione è stata implementare 
validazione alternativa con Zod schema per garantire funzionamento uniforme 
su tutti i browser.

\textbf{Performance con contenuti ricchi} – la landing AI Engineering, con
gallery e componenti interattivi, presentava tempi di caricamento elevati. 
Ottimizzazione immagini (WebP), lazy loading e dynamic import hanno migliorato 
significativamente le performance, raggiungendo i target Lighthouse prefissati.  

\textbf{Bundle optimization} – il bundle iniziale risultava eccessivamente 
pesante. L'implementazione di code splitting automatico per route e dynamic 
import per componenti pesanti ha ridotto il payload iniziale, migliorando 
First Contentful Paint.

\textbf{Tracking inconsistente} – eventi Mixpanel non venivano tracciati 
correttamente su dispositivi iOS in modalità Private Browsing. Implementazione 
graceful degradation e fallback per tracking senza localStorage ha risolto 
la problematica.

\section{Esempio implementazione}
Segue un esempio di implementazione tecnica significativa per il progetto:

\begin{lstlisting}[language=JavaScript, caption=Custom hook per tracking]
// hooks/use-mixpanel.ts
const track = useCallback((event: string, properties?: Record<string, any>) => {
  // Implementation details
}, []);
\end{lstlisting}

Questo esempio illustra l'approccio adottato per garantire tracking consistente 
e GDPR-compliant attraverso hook personalizzati riutilizzabili.

\section{Testing e qualità}
Il progetto è stato sottoposto a una strategia di quality assurance che ha
incluso test manuali su staging, validazione cross-browser e responsive,
verifica accessibilità WCAG 2.1 AA, monitoraggio performance con Lighthouse CI e
controlli regressivi su ogni landing.  

La qualità del codice è stata mantenuta tramite ESLint, Prettier, typing
rigoroso con TypeScript e code review obbligatorie su ogni Pull Request.

Testing specifico ha incluso:
\begin{itemize}
  \item Validazione funnel di conversione su ambiente staging
  \item Test cross-browser su Chrome, Safari, Firefox
  \item Responsive testing su dispositivi mobile e tablet
  \item Performance audit con Core Web Vitals monitoring
  \item Accessibilità con screen reader e keyboard navigation
\end{itemize}

\bigskip
In sintesi, lo sviluppo ha tradotto la progettazione in un ecosistema di landing
pages funzionante, scalabile e conforme agli standard qualitativi, preparando il
terreno al dispiegamento e monitoraggio descritti nel capitolo successivo.