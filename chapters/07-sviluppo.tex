\chapter{Sviluppo}

\section{Processo di sviluppo}
Lo sviluppo delle landing pages è stato organizzato con un approccio
iterativo basato su sprint bisettimanali. Ogni sprint prevedeva
pianificazione delle priorità, sviluppo incrementale, revisione del
codice e demo finale con stakeholder per raccogliere feedback.  
Il workflow ha incluso code review obbligatoria su ogni Pull Request,
testing continuo su ambiente \textit{staging} e rilascio progressivo
per singola landing.

\section{Fasi di implementazione}
Il progetto è stato articolato in quattro fasi principali:

\begin{itemize}
  \item \textbf{Fase 1 – Architettura e design system}: setup del progetto
  Next.js con App Router, configurazione multilingua e definizione del design
  system base con Tailwind e ShadCN.
  \item \textbf{Fase 2 – Sviluppo landing pages}: implementazione progressiva
  delle sei landing, a partire dalla Home Page (validazione del design system) fino alla Jobs Platform.
  \item \textbf{Fase 3 – Tracking e ottimizzazione}: integrazione completa di
  Mixpanel, funnel di conversione per verticale, cookie consent GDPR e
  ottimizzazioni SEO e performance.
  \item \textbf{Fase 4 – Deploy e monitoring}: rilascio graduale in produzione,
  monitoraggio in tempo reale con Mixpanel e Vercel Analytics, e documentazione finale.
\end{itemize}

\section{Sviluppo delle landing}
Le landing sono state sviluppate seguendo priorità strategiche. La
\textbf{Home Page} è stata la prima, hub centrale con routing verso i verticali
e social proof aziendali.  
Sono seguite le pagine \textbf{Tech Recruiting} e \textbf{Community}, rispettivamente orientate a lead generation B2B e alla crescita della community.  
Successivamente sono state realizzate \textbf{AI Adoption} e \textbf{AI Engineering}, focalizzate sui servizi enterprise, e infine la \textbf{Jobs Platform}, progettata mobile–first con ricerca avanzata e trasparenza salariale.  

\section{Implementazione tecnica}
Dal punto di vista tecnico, lo sviluppo ha previsto tre aree principali:

\textbf{Architettura componenti} – è stata realizzata una libreria secondo i
principi Atomic Design (atoms, molecules, organisms, templates), così da
garantire consistenza e riuso.  

\textbf{Routing e rendering} – le landing sono state generate staticamente
(SSG) con revalidazione oraria (ISR), con gestione automatica di meta tags e
sitemap per l’ottimizzazione SEO.  

\textbf{Tracking e compliance} – l’SDK di Mixpanel è stato integrato con un
custom hook che assicura tracking coerente su tutti i componenti. Il sistema è
GDPR-compliant: inizializzazione solo dopo consenso esplicito, anonimizzazione
IP e retention limitata a 12 mesi.

\section{Sfide affrontate e soluzioni}
Durante lo sviluppo sono emerse alcune difficoltà rilevanti:

\textbf{Compliance GDPR} – inizialmente Mixpanel si avviava prima del consenso
utente. La soluzione è stata il caricamento lazy dopo accettazione, con hook
dedicato e anonimizzazione IP. Questo ha garantito il 100\% di conformità e un
opt-in rate del 65\%.  

\textbf{Performance con contenuti ricchi} – la landing AI Engineering, con
gallery e grafici, aveva LCP oltre 4s. Ottimizzazione immagini (WebP), lazy
loading e dynamic import hanno ridotto il tempo a 1.8s, migliorando il
Lighthouse Score a 94/100.  

\textbf{Validazione form cross-browser} – i form mostravano errori incoerenti
tra Chrome, Safari e Firefox. L’adozione di Zod e React Hook Form ha uniformato
la gestione, riducendo il tasso di errori dal 15\% al 3\%.

\section{Esempio di codice significativo}
Segue un esempio di componente \textit{Hero}, progettato per essere riutilizzato
su tutte le landing con varianti per target B2B, B2C e Community:

\begin{lstlisting}[language=JavaScript, caption=Hero Component con varianti]
// components/organisms/Hero.tsx
import { Button } from '@/components/ui/button';
import { cn } from '@/lib/utils';

type HeroVariant = 'b2b' | 'b2c' | 'community';

interface HeroProps {
  variant: HeroVariant;
  title: string;
  description: string;
  ctaPrimary: { label: string; href: string };
  ctaSecondary?: { label: string; href: string };
  image?: string;
}

export function Hero({
  variant, title, description,
  ctaPrimary, ctaSecondary, image
}: HeroProps) {
  const styles = {
    b2b: 'bg-white text-gray-900',
    b2c: 'bg-gradient-to-r from-orange-500 to-red-600 text-white',
    community: 'bg-gray-900 text-white'
  };

  return (
    <section className={cn('py-20 px-6', styles[variant])}>
      <div className="max-w-7xl mx-auto grid md:grid-cols-2 gap-12">
        <div className="flex flex-col justify-center">
          <h1 className="text-5xl font-bold mb-6">{title}</h1>
          <p className="text-xl mb-8 opacity-90">{description}</p>
          <div className="flex gap-4">
            <Button size="lg" asChild>
              <a href={ctaPrimary.href}>{ctaPrimary.label}</a>
            </Button>
            {ctaSecondary && (
              <Button size="lg" variant="outline" asChild>
                <a href={ctaSecondary.href}>{ctaSecondary.label}</a>
              </Button>
            )}
          </div>
        </div>
        {image && (
          <div className="relative h-[400px]">
            <Image src={image} alt={title} fill
                   className="object-cover rounded-lg" />
          </div>
        )}
      </div>
    </section>
  );
}
\end{lstlisting}

Questo esempio mostra l’approccio component-based riutilizzabile, con
tipizzazione TypeScript, responsive design e accessibilità garantita.

\section{Testing e qualità}
Il progetto è stato sottoposto a una strategia di quality assurance che ha
incluso test manuali su staging, validazione cross-browser e responsive,
verifica accessibilità WCAG 2.1 AA, monitoraggio performance con Lighthouse CI e
controlli regressivi su ogni landing.  
La qualità del codice è stata mantenuta tramite ESLint, Prettier, typing
rigoroso con TypeScript e code review obbligatorie.  

\bigskip
In sintesi, lo sviluppo ha tradotto la progettazione in un ecosistema di landing
pages funzionante, scalabile e conforme agli standard qualitativi, preparando il
terreno al dispiegamento e monitoraggio descritti nel capitolo successivo.
