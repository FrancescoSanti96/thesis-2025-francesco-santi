\chapter{Sviluppo}

\section{Processo di sviluppo}
Lo sviluppo delle landing pages è avvenuto seguendo un approccio 
iterativo basato su sprint bisettimanali, con rilasci incrementali e 
feedback continuo dal team di design e product.

\subsection{Metodologia iterativa}
\begin{itemize}
  \item Sprint planning con prioritizzazione landing per impatto business
  \item Sviluppo incrementale con rilasci per singola landing page
  \item Code review obbligatoria per ogni Pull Request su GitHub
  \item Testing continuo su ambiente staging prima di deploy production
  \item Demo finale di sprint con stakeholder per feedback
\end{itemize}

\subsection{Fasi di implementazione}
Il progetto è stato sviluppato seguendo 4 fasi principali:

\textbf{Fase 1: Architettura e design system (Febbraio 2025)}
\begin{itemize}
  \item Setup progetto Next.js 14 con App Router e TypeScript
  \item Configurazione routing multilingua (/it/ e /en/)
  \item Implementazione design system base con Tailwind CSS custom config
  \item Integrazione ShadCN UI e creazione componenti core riutilizzabili
  \item Setup Mixpanel SDK e configurazione tracking base
\end{itemize}

\textbf{Fase 2: Sviluppo landing pages (Marzo-Aprile 2025)}
\begin{itemize}
  \item Implementazione Home Page come hub centrale con routing 
        intelligente
  \item Sviluppo landing Tech Recruiting (B2B) e Tech Community
  \item Sviluppo landing AI Adoption e AI Engineering per servizi enterprise
  \item Implementazione Jobs platform con ricerca e filtri
  \item Integrazione form lead generation con validazione Zod
\end{itemize}

\textbf{Fase 3: Tracking e ottimizzazione (Maggio 2025)}
\begin{itemize}
  \item Mappatura completa eventi Mixpanel per tutte le landing
  \item Setup funnel di conversione per verticale
  \item Implementazione cookie consent GDPR-compliant
  \item Setup dashboard Redash per analytics avanzata
  \item Testing performance e ottimizzazione Core Web Vitals
  \item SEO optimization: meta tags, structured data, sitemap
\end{itemize}

\textbf{Fase 4: Deploy e monitoring (Giugno 2025)}
\begin{itemize}
  \item Deploy su staging per testing finale
  \item Go-live in produzione con rollout graduale per landing
  \item Monitoring real-time con Mixpanel e Vercel Analytics
  \item Hotfix e iterazioni basate su feedback utenti
  \item Documentazione tecnica e handoff al team
\end{itemize}

\section{Sviluppo delle 6 landing pages}

\subsection{Ordine di implementazione}
Le landing sono state sviluppate seguendo priorità strategiche e 
dipendenze tecniche:

\textbf{1. Home Page} (/it/) - \textit{Durata: 1 sprint}
\begin{itemize}
  \item Hub centrale con overview dei 4 verticali
  \item Routing intelligente verso landing specializzate
  \item Social proof (500k+ community, 100+ aziende)
  \item Prima landing per validare design system
\end{itemize}

\textbf{2. Tech Recruiting} (/it/tech-recruiting/) - \textit{Durata: 1.5 sprint}
\begin{itemize}
  \item Funnel B2B per lead generation enterprise
  \item Processo 6 step visualizzato con animazioni
  \item Showcase talent pool con filtri interattivi
  \item Form contatto integrato con CRM HubSpot
\end{itemize}

\textbf{3. Tech Community} (/it/tech-media-agency/) - \textit{Durata: 1 sprint}
\begin{itemize}
  \item Acquisizione membri e iscrizioni newsletter
  \item Integrazione API newsletter "Commit" via Mailchimp
  \item Showcase eventi con calendario dinamico
  \item CTA Discord con deep linking
\end{itemize}

\textbf{4. AI Adoption} (/it/ai-adoption/) - \textit{Durata: 1.5 sprint}
\begin{itemize}
  \item Value proposition servizi upskilling
  \item Visualizzazione evidenze scientifiche con Chart.js
  \item Percorsi formativi differenziati per ruolo
  \item Form demo request con campi condizionali
\end{itemize}

\textbf{5. AI Engineering} (/it/ai-engineering/) - \textit{Durata: 1.5 sprint}
\begin{itemize}
  \item Showcase framework proprietario "Datapizza AI"
  \item Gallery progetti interattiva (Copiloti Sales, HR, Legal, Customer)
  \item Tech stack visualization con animazioni
  \item CTA per avvio progetto custom con form multi-step
\end{itemize}

\textbf{6. Jobs Platform} (/jobs/) - \textit{Durata: 1 sprint}
\begin{itemize}
  \item Esperienza candidati ottimizzata mobile-first
  \item Ricerca posizioni con filtri avanzati (tech stack, location, RAL)
  \item Trasparenza salary (RAL sempre visibile)
  \item Sistema Tech Buddy con matching automatico
\end{itemize}

\section{Implementazione tecnica}

\subsection{Architettura componenti}
\textbf{Design System Setup}
\begin{itemize}
  \item Configurazione Tailwind con palette custom Datapizza 
        (primary \#FF6B35)
  \item Implementazione componenti base ShadCN: Button, Input, Card, 
        Modal, Form
  \item Creazione component library secondo Atomic Design:
    \begin{itemize}
      \item \textbf{Atoms}: Button, Input, Badge, Icon (Lucide React)
      \item \textbf{Molecules}: Card, FormField, NavItem, Dropdown
      \item \textbf{Organisms}: Header, Footer, Hero, Feature Grid, 
            Form Lead Generation
      \item \textbf{Templates}: Landing Layout con slot per content
    \end{itemize}
  \item Documentazione componenti in Storybook (considerato ma non 
        implementato per timeline)
\end{itemize}

\textbf{Routing e Navigazione}
\begin{itemize}
  \item Setup Next.js App Router per /it/ e /en/
  \item SSG (Static Site Generation) per tutte le landing pages
  \item ISR (Incremental Static Regeneration) con revalidation ogni ora
  \item Gestione meta tags dinamici per SEO con Metadata API Next.js 14
  \item Sitemap XML generation automatica con next-sitemap
\end{itemize}

\subsection{Integrazione tracking Mixpanel}
\textbf{Event Mapping Implementation}
\begin{itemize}
  \item SDK Mixpanel integrato come React Context Provider
  \item Custom hook \texttt{useMixpanel()} per tracking consistente
  \item Event taxonomy implementata:
    \begin{itemize}
      \item Page events: \texttt{page\_view}, \texttt{landing\_loaded}
      \item Interaction: \texttt{cta\_clicked}, \texttt{scroll\_depth}, 
            \texttt{section\_viewed}
      \item Conversion: \texttt{form\_submitted}, 
            \texttt{newsletter\_signup}
    \end{itemize}
  \item User properties automatiche: source (UTM), device, location, 
        landing\_page
\end{itemize}

\textbf{GDPR Compliance Implementation}
\begin{itemize}
  \item Cookie consent banner con CookieYes
  \item Tracking opt-out automatico se consent negato
  \item Mixpanel inizializzato solo dopo explicit consent
  \item IP anonymization abilitata su tutti gli eventi
\end{itemize}

\textbf{Funnel Implementation}
Funnel specifici implementati per verticale:
\begin{itemize}
  \item \textbf{B2B Recruiting}: landing\_view → cta\_click → 
        form\_view → form\_submit
  \item \textbf{B2C Jobs}: landing\_view → search\_used → 
        position\_click → application\_start
  \item \textbf{Community}: landing\_view → scroll\_50\% → 
        newsletter\_click → newsletter\_submit
\end{itemize}

\subsection{Performance Optimization}
\begin{itemize}
  \item \textbf{Image optimization}: next/image con formato WebP, 
        lazy loading, srcset responsive
  \item \textbf{Code splitting}: Automatico per route, dynamic import 
        per Chart.js (bundle pesante)
  \item \textbf{Critical CSS}: Inlining automatico Next.js per 
        above-the-fold content
  \item \textbf{Font optimization}: Inter font con preload e 
        display: swap
  \item \textbf{Bundle analysis}: webpack-bundle-analyzer per 
        identificare bottleneck
  \item \textbf{Risultati}: Lighthouse Performance Score 92-96 su 
        tutte le landing
\end{itemize}

\section{Difficoltà incontrate e soluzioni}

\subsection{Challenge 1: GDPR Compliance e Tracking}
\textbf{Problema}
\begin{itemize}
  \item Necessità di tracciare utenti rispettando normativa europea
  \item Mixpanel si inizializzava prima del cookie consent
  \item Risk di multe GDPR per tracking non consensuale
\end{itemize}

\textbf{Soluzione adottata}
\begin{itemize}
  \item Implementato cookie consent banner con CookieYes
  \item Refactor Mixpanel init: caricamento lazy dopo explicit consent
  \item Created custom hook \texttt{useMixpanelConsent()} per gestire 
        stato consent
  \item IP anonymization abilitata di default
  \item Data retention policy configurata a 12 mesi con auto-delete
\end{itemize}

\textbf{Risultato}
\begin{itemize}
  \item 100\% conformità GDPR
  \item Tracking opt-in rate: ~65\% utenti (in linea con industry 
        standard)
  \item Zero issue legali o reclami privacy
\end{itemize}

\subsection{Challenge 2: Performance con Contenuti Rich}
\textbf{Problema}
\begin{itemize}
  \item Landing AI Engineering con gallery progetti causava LCP > 4s
  \item Immagini high-res non ottimizzate (2-3MB ciascuna)
  \item Chart.js bundle pesante (200KB) impattava FCP
  \item Lighthouse Performance Score: 68/100 (inaccettabile)
\end{itemize}

\textbf{Soluzione adottata}
\begin{itemize}
  \item Conversione immagini da PNG a WebP con compressione quality 80
  \item Implementato lazy loading per gallery con Intersection Observer
  \item Chart.js caricato con dynamic import solo quando visibile
  \item Preload immagini above-the-fold con \texttt{<link rel="preload">}
  \item CDN CloudFront per asset statici con edge caching
\end{itemize}

\textbf{Risultato}
\begin{itemize}
  \item LCP ridotto da 4.2s a 1.8s (-57\%)
  \item Bundle size ridotto da 850KB a 420KB (-50\%)
  \item Lighthouse Performance Score: 94/100
  \item Core Web Vitals tutti "Good" (verde)
\end{itemize}

\subsection{Challenge 3: Form Validation Cross-Browser}
\textbf{Problema}
\begin{itemize}
  \item Form lead generation con validazione complex (email, phone, 
        company size)
  \item Comportamento inconsistente tra Chrome, Safari, Firefox
  \item Safari non supportava alcune regex JavaScript per phone validation
  \item UX confusa: errori mostrati in modo diverso per browser
\end{itemize}

\textbf{Soluzione adottata}
\begin{itemize}
  \item Adottato Zod per schema validation type-safe
  \item React Hook Form per gestione stato form cross-browser
  \item Custom regex semplificata per phone (solo numeri + prefisso)
  \item Error messages consistenti con ShadCN Form components
  \item Testing manuale su Chrome, Safari, Firefox, Edge
\end{itemize}

\textbf{Risultato}
\begin{itemize}
  \item 100\% cross-browser compatibility
  \item Form submission error rate ridotto da ~15\% a ~3\%
  \item UX consistente su tutti i browser
  \item Type safety garantita con TypeScript + Zod
\end{itemize}

\section{Esempi di codice significativi}

\subsection{Esempio 1: Hero Component Riutilizzabile}
Componente Hero con varianti per diversi target (B2B/B2C/Community):

\begin{lstlisting}[language=JavaScript, caption=Hero Component con varianti]
// components/organisms/Hero.tsx
import { Button } from '@/components/ui/button';
import { cn } from '@/lib/utils';

type HeroVariant = 'b2b' | 'b2c' | 'community';

interface HeroProps {
  variant: HeroVariant;
  title: string;
  description: string;
  ctaPrimary: { label: string; href: string };
  ctaSecondary?: { label: string; href: string };
  image?: string;
}

export function Hero({ 
  variant, 
  title, 
  description, 
  ctaPrimary, 
  ctaSecondary,
  image 
}: HeroProps) {
  const styles = {
    b2b: 'bg-white text-gray-900',
    b2c: 'bg-gradient-to-r from-orange-500 to-red-600 text-white',
    community: 'bg-gray-900 text-white'
  };

  return (
    <section className={cn(
      'py-20 px-6',
      styles[variant]
    )}>
      <div className="max-w-7xl mx-auto grid md:grid-cols-2 gap-12">
        <div className="flex flex-col justify-center">
          <h1 className="text-5xl font-bold mb-6">{title}</h1>
          <p className="text-xl mb-8 opacity-90">{description}</p>
          <div className="flex gap-4">
            <Button size="lg" asChild>
              <a href={ctaPrimary.href}>{ctaPrimary.label}</a>
            </Button>
            {ctaSecondary && (
              <Button size="lg" variant="outline" asChild>
                <a href={ctaSecondary.href}>{ctaSecondary.label}</a>
              </Button>
            )}
          </div>
        </div>
        {image && (
          <div className="relative h-[400px]">
            <Image src={image} alt={title} fill 
                   className="object-cover rounded-lg" />
          </div>
        )}
      </div>
    </section>
  );
}
\end{lstlisting}

\textbf{Motivazioni design}
\begin{itemize}
  \item Riutilizzabile su tutte le landing con prop \texttt{variant}
  \item Type-safe con TypeScript interfaces
  \item Styling consistente ma flessibile con Tailwind utilities
  \item Responsive by default con grid system
  \item Accessibile: semantic HTML (section, h1) e button proper
\end{itemize}

\subsection{Esempio 2: Mixpanel Tracking Hook}
Custom hook per tracking eventi consistente:

\begin{lstlisting}[language=JavaScript, caption=useMixpanel Hook]
// hooks/useMixpanel.ts
import { useEffect } from 'react';
import mixpanel from 'mixpanel-browser';
import { useConsent } from '@/context/ConsentContext';

export function useMixpanel() {
  const { hasConsent } = useConsent();

  useEffect(() => {
    if (hasConsent && !mixpanel.get_distinct_id()) {
      mixpanel.init(process.env.NEXT_PUBLIC_MIXPANEL_TOKEN!, {
        track_pageview: false,
        ip: false // IP anonymization
      });
    }
  }, [hasConsent]);

  const trackEvent = (eventName: string, properties?: object) => {
    if (!hasConsent) return;
    
    mixpanel.track(eventName, {
      ...properties,
      page: window.location.pathname,
      timestamp: new Date().toISOString()
    });
  };

  const trackPageView = (pageName: string) => {
    trackEvent('page_view', { page_name: pageName });
  };

  return { trackEvent, trackPageView };
}

// Usage in component:
// const { trackEvent } = useMixpanel();
// trackEvent('cta_clicked', { cta_label: 'Richiedi Demo' });
\end{lstlisting}

\textbf{Impatto e risultati}
\begin{itemize}
  \item Tracking GDPR-compliant: init solo se consent granted
  \item Consistent API per tutti i componenti
  \item Auto-enrichment: page e timestamp aggiunti automaticamente
  \item Type-safe con TypeScript
  \item Permesso di tracciare 50k+ eventi/mese con conversion tracking 
        accurato
\end{itemize}

\section{Testing e quality assurance}

\subsection{Strategy di testing}
\begin{itemize}
  \item \textbf{Manual testing}: Completo su staging prima di ogni 
        deploy production
  \item \textbf{Cross-browser}: Chrome, Firefox, Safari, Edge (desktop 
        e mobile)
  \item \textbf{Responsive testing}: iPhone, iPad, Android (vari 
        breakpoints)
  \item \textbf{Accessibility}: WAVE tool per WCAG 2.1 AA compliance, 
        screen reader testing con NVDA
  \item \textbf{Performance}: Lighthouse CI per monitoring continuo 
        Core Web Vitals
  \item \textbf{Regression}: Checklist per ogni landing (link, form, 
        tracking, SEO)
\end{itemize}

\subsection{Bug tracking e risoluzione}
\begin{itemize}
  \item Bug reportati su Jira con priority (Critical/High/Medium/Low)
  \item Critical bugs: hotfix immediate con deploy fuori sprint
  \item Tutti i bug risolti prima di considerare sprint completato
  \item Post-mortem per bug critici per prevenire recurrence
\end{itemize}

\subsection{Code quality}
\begin{itemize}
  \item ESLint + Prettier per linting e formatting automatico
  \item TypeScript strict mode per type safety
  \item Code review obbligatoria: almeno 1 approval prima di merge
  \item Pre-commit hooks con Husky per bloccare commit con errori lint
\end{itemize}