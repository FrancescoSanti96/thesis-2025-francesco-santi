\chapter{Sviluppo}

\section{Processo di sviluppo}
Lo sviluppo delle landing pages è stato organizzato con un approccio
\textbf{agile iterativo} basato su sprint bisettimanali. Ogni sprint prevedeva
pianificazione delle priorità, sviluppo incrementale, revisione del
codice.
Il workflow ha incluso code review obbligatoria su ogni \textit{Pull Request} 
prima del merge sul branch principale, testing continuo su ambiente 
\textit{staging} e rilascio progressivo per singola landing.

\section{Fasi di implementazione}
Il progetto è stato articolato in quattro fasi principali:

\begin{itemize}
  \item \textbf{Fase 1 – Architettura e design system}: setup del progetto
  Next.js con App Router, configurazione multilingua e definizione del design
  system base con Tailwind CSS e ShadCN UI.
  \item \textbf{Fase 2 – Sviluppo landing pages}: implementazione progressiva
  delle sei landing, a partire dalla Home Page (validazione del design system) 
  fino alla Jobs Platform.
  \item \textbf{Fase 3 – Tracking e ottimizzazione}: integrazione completa di
  Mixpanel e altri provider analytics, funnel di conversione per verticale, 
  cookie consent GDPR e ottimizzazioni SEO e performance.
  \item \textbf{Fase 4 – Deploy e monitoring}: rilascio graduale in produzione,
  monitoraggio in tempo reale e documentazione finale.
\end{itemize}

\section{Sviluppo delle landing}
Le landing sono state sviluppate seguendo priorità strategiche, con approccio 
mobile-first applicato a tutte le sei pagine per garantire esperienza ottimale 
su dispositivi mobili. La \textbf{Home Page} è stata sviluppata per prima come 
hub centrale con routing verso i verticali e social proof aziendale, permettendo 
di validare il design system e i componenti base riutilizzabili. Sono seguite 
le pagine \textbf{Tech Recruiting}, orientata a lead generation B2B per aziende 
in cerca di talenti tech, e \textbf{Tech Community}, focalizzata sulla crescita 
della community con target B2C (developer e tech enthusiast) e B2B (brand 
awareness per aziende partner). Queste prime landing hanno consolidato i pattern 
di interazione e i funnel di conversione differenziati per target. Successivamente 
sono state realizzate \textbf{AI Adoption} e \textbf{AI Engineering}, entrambe 
focalizzate su servizi enterprise B2B con contenuti più complessi e showcase 
progetti. Infine è stata implementata la \textbf{Jobs Platform}, orientata B2C 
per candidati in cerca di opportunità, che integra ricerca avanzata, filtri 
dinamici e trasparenza salariale riutilizzando i componenti già validati nelle 
landing precedenti.

\section{Implementazione architettura}
Come progettato nel capitolo precedente, l'implementazione ha seguito le decisioni 
architetturali chiave: monorepo Next.js con sei route dedicate, Static Site 
Generation con Incremental Static Regeneration per performance ottimali, e 
design system basato su ShadCN UI.

Dal punto di vista tecnico, lo sviluppo ha previsto tre aree principali. 
L'architettura componenti ha realizzato una libreria di elementi riutilizzabili 
con TypeScript per garantire consistenza e riuso tra le sei landing pages. 
Il routing e rendering ha implementato generazione statica con revalidazione 
oraria, gestione automatica di meta tags e sitemap per ottimizzazione SEO. 
Il sistema di tracking ha integrato diversi provider analytics con particolare 
attenzione alla conformità GDPR.

\section{Integrazione tracking e analytics}

Il sistema di tracking è stato integrato nelle landing pages implementando 
diversi provider analytics per garantire copertura completa del customer journey. 
Mixpanel fornisce il tracciamento principale degli eventi utente con funnel 
differenziati per verticale. Google Analytics 4 e Google Tag Manager gestiscono 
metriche di traffico e orchestrazione eventi verso i diversi provider. Facebook 
Pixel traccia conversioni per campagne marketing con event deduplication per 
evitare doppi conteggi.

L'implementazione ha richiesto particolare attenzione alla conformità GDPR: 
tutti i provider sono inizializzati solo dopo consenso esplicito dell'utente 
tramite cookie banner, con gestione differenziata delle categorie analytics 
e marketing. Gli indirizzi IP vengono anonimizzati automaticamente e gli utenti 
possono revocare il consenso in qualsiasi momento attraverso le preferenze cookie.

\section{Sfide affrontate e soluzioni}
Durante lo sviluppo sono emerse diverse difficoltà tecniche legate all'implementazione 
di animazioni complesse e effetti visuali avanzati, risolte con soluzioni specifiche.

L'implementazione di animazioni ed effetti parallax sulle landing, progettati per 
creare un'esperienza immersiva su desktop, richiedeva particolare attenzione alla 
resa su dispositivi mobile. Gli effetti parallax, che su schermi grandi garantivano 
profondità visiva e dinamismo durante lo scroll, rischiavano di compromettere le 
performance su mobile a causa del rendering più oneroso e della gestione touch 
differente rispetto al mouse. La soluzione ha implementato detection automatica 
del tipo di dispositivo con disabilitazione selettiva degli effetti più pesanti 
su mobile, mantenendo animazioni semplificate che preservassero l'esperienza visiva 
senza impattare negativamente le performance.

La gestione delle animazioni cross-browser presentava inconsistenze significative, 
particolarmente per transizioni complesse e animazioni sincronizzate tra componenti. 
L'adozione di Framer Motion come libreria di animazione standardizzata ha garantito 
comportamento uniforme su tutti i browser, eliminando le differenze di rendering 
e timing che causavano esperienze utente frammentate.

La landing AI Engineering, caratterizzata da gallery interattive e showcase 
progetti con contenuti multimediali ricchi, presentava tempi di caricamento 
elevati. L'ottimizzazione delle immagini ha richiesto particolare attenzione 
per bilanciare qualità visiva e dimensioni file, con un processo iterativo di 
compressione e ridimensionamento che ha permesso di ridurre significativamente 
i tempi di caricamento mantenendo l'impatto visivo richiesto dal design. 
L'implementazione di lazy loading per il caricamento progressivo dei contenuti 
fuori viewport ha ulteriormente migliorato le performance iniziali della pagina.

\section{Testing e qualità}
Il progetto è stato sottoposto a una strategia di quality assurance che ha
incluso test manuali su staging, validazione cross-browser e responsive,
verifica accessibilità WCAG 2.1 AA, monitoraggio performance e controlli 
regressivi su ogni landing.

Il controllo della qualità del codice è stato garantito tramite ESLint, Prettier, 
typing rigoroso con TypeScript e code review obbligatorie su ogni Pull Request.
Il testing specifico ha coperto validazione funnel di conversione su ambiente 
staging, test cross-browser su Chrome, Safari e Firefox, responsive testing 
su dispositivi mobile e tablet, performance audit con Core Web Vitals monitoring, 
e accessibilità con screen reader e keyboard navigation.