\chapter{Verifica sperimentale}

\section{Testing e validazione}

La fase di verifica ha coperto testing funzionale (routing multilingua, form 
submission, tracking Mixpanel), cross-browser compatibility (Chrome, Firefox, 
Safari, Edge su desktop e mobile), e usabilità con beta testing su 10 utenti 
interni (rating medio 8/10). Le modifiche iterative hanno migliorato contrasto 
testi e dimensione call-to-action mobile.

\section{Risultati performance e accessibilità}

Il redesign ha portato miglioramenti significativi su tutte le metriche Lighthouse, 
come mostrato nella Figura \ref{fig:lighthouse-comparison}. Il performance score 
è passato da 65/100 a 94/100, l'accessibilità da 78/100 a 98/100, mentre SEO e 
Best Practices hanno raggiunto 100/100. Il bundle JavaScript è stato ridotto del 
50\% (da 850KB a 420KB) tramite code splitting e cleanup dependencies.

\begin{figure}[h!]
    \centering
    \includegraphics[width=\textwidth]{chapters/figures/lighthouse-comparison.pdf}
    \caption{Confronto score Lighthouse pre e post-redesign.}
    \label{fig:lighthouse-comparison}
\end{figure}

Le Core Web Vitals hanno raggiunto tutti i target: First Contentful Paint 1.4s, 
Largest Contentful Paint 2.1s, Cumulative Layout Shift 0.04, Time to Interactive 
2.6s, Total Blocking Time 180ms (tutti in range "Good" su Google Search Console). 
L'accessibilità WCAG 2.1 livello AA è compliant al 100\% con test screen reader 
NVDA, keyboard navigation e contrasto colori verificati. La validazione SEO 
conferma meta tags ottimizzati, structured data JSON-LD e tutte le sei landing 
indicizzate correttamente entro due settimane.

\section{Tracking e analytics}

Il sistema Mixpanel ha monitorato 95.000 eventi mensili su 30 giorni post-lancio. 
La Figura \ref{fig:mixpanel-funnel} mostra i funnel di conversione per verticale.

\begin{figure}[h!]
    \centering
    \includegraphics[width=\textwidth]{chapters/figures/mixpanel-funnel.pdf}
    \caption{Funnel di conversione per verticale B2B, B2C e Community.}
    \label{fig:mixpanel-funnel}
\end{figure}

I funnel evidenziano performance differenziate: B2B (Recruiting, AI Adoption, AI 
Engineering) con 4.5\% conversion rate complessivo da 15.200 visualizzazioni a 
680 lead qualificati mensili, Community newsletter con 24\% conversion da 11.800 
visualizzazioni a 2.800 iscrizioni mensili, e Jobs Platform con 5\% application 
rate da 18.500 visualizzazioni a 920 candidature mensili. Le metriche engagement 
confermano efficacia dell'esperienza utente: tempo medio sessione 2 minuti e 45 
secondi (superiore al target di 2 minuti), bounce rate 42\% rispetto al baseline 
del 68\% (riduzione relativa del 38\%), scroll depth medio 58\%, e click-through 
rate sulle call-to-action del 19\%. Il traffico di 45.000 utenti mensili è 
prevalentemente organico (62\%) e mobile (58\%), validando l'approccio mobile-first 
del design.

\section{Problemi riscontrati e risoluzioni}

Nelle prime settimane post-lancio è emersa una problematica sui form di contatto 
delle landing B2B: il sistema registrava doppi invii causando confusione utente 
e lead duplicati nel CRM del team sales. L'analisi della root cause ha rivelato 
due problematiche correlate: il pulsante "Invia" non si disabilitava immediatamente 
permettendo click multipli durante il processing, e mancava feedback visivo chiaro 
post-submit lasciando l'utente in stato di incertezza sul completamento dell'azione.

La risoluzione in due iterazioni ha implementato disabilitazione automatica del 
pulsante al primo click e alert di conferma esplicito post-submit (Figura 
\ref{fig:form-success}) che comunica chiaramente l'avvenuto completamento e le 
tempistiche di follow-up. Il deployment completato in una settimana ha eliminato 
completamente i doppi invii.

\begin{figure}[h!]
    \centering
    \includegraphics[width=0.8\textwidth]{chapters/figures/form-success-alert.pdf}
    \caption{Alert di conferma implementato per feedback post-submit.}
    \label{fig:form-success}
\end{figure}

Un issue minore riguardava il tracking Mixpanel su Safari in modalità Private 
Browsing (circa 5\% utenti iOS) dove il blocco di localStorage impediva la 
persistenza degli eventi. La soluzione con tracking session-based è stata deployata 
in sprint successivo. Questi incident hanno consolidato best practices: testing 
con scenari stress (double-click, connettività degradata), feedback visivo 
obbligatorio fin dalla fase di design, e monitoring granulare per device type e 
browser specifici.

\section{Risultati rispetto agli obiettivi}

Gli obiettivi tecnici definiti nel Capitolo 4 sono stati raggiunti: performance 
score Lighthouse 94/100 (target superiore a 90), accessibilità WCAG 2.1 AA 
compliant al 100\%, SEO e Best Practices 100/100, e Core Web Vitals tutti in 
range "Good". L'architettura modulare implementata garantisce scalabilità con 
tempo stimato di 2-4 settimane per sviluppare nuove landing verticali riutilizzando 
componenti esistenti.

Sul fronte business, le landing generano complessivamente 4.400 lead mensili 
distribuiti tra verticali: 680 lead B2B qualificati con conversion rate del 4.5\%, 
2.800 iscrizioni newsletter Community con conversion rate del 24\%, e 920 
candidature Jobs Platform con application rate del 5\%. Il bounce rate è migliorato 
da 68\% a 42\% (riduzione relativa del 38\%) mentre il tempo medio sessione di 
2 minuti e 45 secondi supera il target di 2 minuti. I conversion rate ottenuti 
sono in linea con industry benchmarks per il settore. Il traffico totale di 45.000 
utenti mensili si distribuisce: Home 35\%, Jobs 28\%, Tech Recruiting 18\%, 
Community 12\%, verticali AI 7\%. Metriche business a lungo termine come ROI 
campagne marketing e riduzione CAC richiedono periodo di analisi più esteso dei 
30 giorni documentati.

\bigskip
Il progetto ha raggiunto gli obiettivi prefissati fornendo foundation scalabile 
per crescita futura e validando l'approccio multi-landing con tracking avanzato.