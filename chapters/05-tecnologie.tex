\chapter{Tecnologie}

\section{Stack tecnologico frontend}
\subsection{React e TypeScript}
\begin{itemize}
  \item React 18 per sviluppo componenti UI
  \item TypeScript 5.x per type safety e migliore developer experience
  \item Custom hooks per logica riutilizzabile
  \item JSX/TSX per sintassi dichiarativa
\end{itemize}

\subsection{Next.js}
\begin{itemize}
  \item Next.js 14 con App Router per routing avanzato
  \item Server-Side Rendering (SSR) e Static Site Generation (SSG)
  \item Ottimizzazione automatica immagini con next/image
  \item Supporto i18n nativo per /it/ e /en/
  \item Code splitting automatico per performance
\end{itemize}

\subsection{Styling e UI}
\textbf{Tailwind CSS}
\begin{itemize}
  \item Tailwind CSS 3.x per utility-first styling
  \item Configurazione custom per palette Datapizza
  \item Plugin typography per contenuti rich text
  \item Purge CSS per ottimizzazione bundle produzione
\end{itemize}

\textbf{ShadCN UI}
\begin{itemize}
  \item Component library basata su Radix UI
  \item Accessibilità WCAG 2.1 integrata
  \item Componenti headless customizzabili
  \item Design tokens per consistenza visiva
\end{itemize}

\subsection{State management}
\begin{itemize}
  \item React Query (TanStack Query) per API calls e caching
  \item Server state management con automatic refetching
  \item Optimistic updates per UX fluida
  \item Gestione automatica loading/error states
\end{itemize}

\section{Stack tecnologico backend}
\subsection{Django e PostgreSQL}
\begin{itemize}
  \item Django 4.x per API REST
  \item Django REST Framework per serialization
  \item PostgreSQL 15 come database principale
  \item ORM Django per query optimization
\end{itemize}

\subsection{Infrastructure}
\begin{itemize}
  \item AWS per cloud infrastructure
  \item S3 per asset statici (immagini, font, media)
  \item CloudFront CDN per distribuzione globale
  \item EC2/ECS per application hosting
\end{itemize}

\section{Analytics e tracking}
\subsection{Mixpanel}
\begin{itemize}
  \item Event tracking per comportamento utente
  \item Funnel analysis e segmentazione
  \item GDPR compliance con cookie consent
  \item Real-time analytics dashboard
\end{itemize}

\subsection{Redash}
\begin{itemize}
  \item Dashboard SQL per business intelligence
  \item Visualizzazione dati e reportistica
  \item Query scheduling per report automatici
  \item Integrazione con PostgreSQL
\end{itemize}

\section{Development tools}
\subsection{IDE e editor}
\begin{itemize}
  \item Visual Studio Code come IDE principale
  \item Cursor AI per coding assistance
  \item ESLint e Prettier per code quality
  \item TypeScript language server per intellisense
\end{itemize}

\subsection{Design e collaboration}
\begin{itemize}
  \item Figma per design handoff e prototipazione
  \item Figma Dev Mode per export CSS/React
  \item Design system documentation in Figma
\end{itemize}

\subsection{Version control e CI/CD}
\begin{itemize}
  \item Git per version control
  \item GitHub per repository management e code review
  \item GitHub Actions per CI/CD pipeline
  \item Vercel per deployment automatizzato
\end{itemize}

\subsection{Database management}
\begin{itemize}
  \item DBeaver per query development e debugging
  \item pgAdmin per PostgreSQL administration
  \item Database migrations con Django
\end{itemize}

\subsection{Documentation}
\begin{itemize}
  \item Notion per knowledge base e documentazione
  \item Markdown per documentation as code
  \item Storybook per component documentation (considerato)
\end{itemize}

\section{Tecnologie in altri progetti}
Durante l'esperienza ho utilizzato le stesse tecnologie anche per:
\begin{itemize}
  \item Technical debt reduction su Jobs/Company platforms
  \item Setup iniziale gestionale interno aziendale
  \item Feature development con stack React/Django condiviso
\end{itemize}

Le competenze acquisite sul progetto principale sono state direttamente 
trasferibili alle altre attività, garantendo coerenza tecnologica 
nell'ecosistema aziendale.

\section{Motivazioni delle scelte tecnologiche}
\textbf{Next.js}: SEO nativo tramite SSR/SSG, performance ottimizzate 
con code splitting automatico, developer experience eccellente con 
hot reload e App Router.

\textbf{TypeScript}: Type safety che riduce bug in produzione, migliore 
refactoring grazie al type system, documentazione implicita del codice.

\textbf{Tailwind CSS}: Rapid development senza context switching, 
performance runtime ottimali (zero runtime overhead), design system 
facilmente scalabile.

\textbf{ShadCN UI}: Accessibilità built-in con Radix UI, customizzazione 
completa del codice sorgente, componenti production-ready senza 
dipendenze pesanti.

\textbf{React Query}: Gestione server state semplificata, caching 
intelligente che riduce chiamate API, developer tools eccellenti per 
debugging.

\textbf{Mixpanel}: Event tracking granulare senza rallentare il frontend, 
GDPR compliance nativa, funnel analysis potenti per ottimizzazione 
conversioni.