\chapter{Tecnologie}

\section{Stack tecnologico landing pages}
Il progetto di redesign delle landing pages si basa su uno stack moderno 
orientato a performance, SEO e developer experience. Le scelte tecnologiche 
sono state guidate dalla necessità di garantire caricamenti rapidi, 
indicizzazione ottimale sui motori di ricerca e scalabilità futura.

\subsection{Frontend}

Il framework principale è Next.js 15.3, che si basa su React 19 e TypeScript 
5.7. Queste tecnologie erano già utilizzate nel progetto preesistente e sono 
state mantenute durante il redesign per coerenza con l'intero ecosistema 
tecnologico aziendale.

\paragraph{React 19}
React è la libreria JavaScript per costruire interfacce utente component-based, 
utilizzata in tutti i prodotti Datapizza (Jobs, Company, gestionale). Questa 
coerenza garantisce riutilizzo di componenti, competenze consolidate del team e 
facilita l'onboarding di nuovi sviluppatori. L'approccio component-based permette 
di creare UI modulari e riutilizzabili attraverso paradigma dichiarativo, 
facilitando manutenzione e scalabilità del codice.

Uno degli aspetti che rende React particolarmente efficiente è l'uso del Virtual 
DOM, una rappresentazione virtuale dell'interfaccia. Quando qualcosa cambia 
nell'applicazione, React identifica automaticamente solo le parti effettivamente 
modificate e aggiorna esclusivamente quelle, evitando di ricaricare elementi 
inutilmente. Questo approccio migliora significativamente le performance 
percepite dall'utente. Gli Hooks permettono di gestire stato e logica in 
componenti funzionali, semplificando il codice e la riutilizzabilità della 
logica applicativa.

\paragraph{TypeScript 5.7}
A supporto di React, il progetto utilizza TypeScript 5.7, un superset di 
JavaScript che aggiunge type safety statico, ormai standard per applicazioni 
moderne in contesti production. L'adozione di TypeScript riduce drasticamente i 
bug in produzione grazie al type checking in fase di sviluppo: errori comuni 
come accesso a proprietà inesistenti o passaggio di parametri errati vengono 
intercettati dal compilatore prima del deployment. In un contesto production 
come le landing pages, dove bug critici impattano direttamente le conversioni, 
TypeScript è fondamentale per minimizzare errori.

TypeScript migliora significativamente il lavoro in team rendendo il codice più 
strutturato e auto-documentato. L'autocompletion intelligente negli IDE e il 
refactoring sicuro automatizzato accelerano lo sviluppo e riducono il tempo di 
onboarding di nuovi sviluppatori sul codebase.

\paragraph{Next.js 15}
Next.js è il framework React che fornisce le funzionalità essenziali per landing 
pages SEO-oriented. La scelta di Next.js rispetto a React vanilla si basa su 
tre vantaggi fondamentali per il progetto.

Il rendering ibrido combina Server-Side Rendering per pagine dinamiche con 
contenuti statici memorizzati in cache che possono essere aggiornati 
periodicamente senza rigenerare l'intero sito (Incremental Static Regeneration). 
Questo approccio ottimizza sia i tempi di caricamento che l'aggiornamento dei 
contenuti. Il SEO nativo garantisce che i motori di ricerca ricevano HTML 
completo immediatamente, fondamentale per indicizzazione e traffico organico. 
Il code splitting automatico suddivide il codice in bundle separati per ogni 
route, riducendo il payload iniziale dell'applicazione.

Next.js supporta nativamente la configurazione multilingua, estesa in questo 
progetto tramite la libreria next-intl per gestire i percorsi dedicati (/it/ e 
/en/). Il componente next/image ottimizza automaticamente le immagini 
convertendole in formato WebP, caricandole solo quando visibili sullo schermo 
(lazy loading) e adattandole alle diverse dimensioni di dispositivo, riducendo 
il peso delle immagini principali mantenendo qualità visiva.

\subsection{Styling e UI}

Per garantire coerenza visiva con l'identità aziendale, il sistema di styling 
si basa su Tailwind CSS 3.4 con component library ShadCN UI. Tailwind CSS 
permette di costruire interfacce utilizzando classi predefinite direttamente nel 
codice HTML, senza necessità di scrivere file CSS separati. Tutte le classi 
vengono compilate in CSS statico durante la fase di build, garantendo che 
nessun codice aggiuntivo venga eseguito nel browser.

ShadCN UI fornisce componenti accessibili basati su Radix UI con conformità agli 
standard di accessibilità WCAG 2.1 integrata. A differenza di una libreria 
installabile tradizionale, ShadCN UI genera componenti direttamente nel progetto 
tramite CLI, permettendo pieno controllo sul codice e garantendo accessibilità 
nativa attraverso le primitive di Radix UI.

\subsection{Librerie complementari}

Per animazioni e interattività avanzate, il progetto utilizza Framer Motion per 
animazioni fluide nelle sezioni principali (Hero sections) e Three.js con React 
Three Fiber per visualizzazioni 3D sulla landing AI Engineering. La gestione dei 
form si basa su React Hook Form per performance ottimali e Zod per validazione 
dello schema dati con supporto TypeScript integrato sui form di acquisizione 
lead.

\subsection{Backend per dati dinamici}

Alcune funzionalità delle landing pages necessitano di un backend per 
elaborazioni più complesse. Il backend Django gestisce l'invio dei form con 
validazione lato server per garantire sicurezza, l'integrazione con il servizio 
di email automation Customer.io per la newsletter "Commit", il recupero in tempo 
reale delle opportunità lavorative dal database PostgreSQL, e il tracking degli 
eventi utente tramite Mixpanel con conformità GDPR. Lo stack si basa su Django 
4.x con PostgreSQL 15 e Python 3.9.

\section{Stack tecnologico aziendale}

L'ecosistema Datapizza (Jobs, Company, gestionale) condivide uno stack 
unificato per garantire coerenza tecnologica e riutilizzo di competenze tra 
prodotti. La Tabella~\ref{tab:stack-aziendale} fornisce una panoramica 
completa delle tecnologie adottate a livello aziendale.

\begin{table}[h]
\centering
\caption{Stack tecnologico aziendale}
\label{tab:stack-aziendale}
\begin{tabular}{|l|l|p{6.5cm}|}
\hline
\textbf{Layer} & \textbf{Tecnologie} & \textbf{Note} \\
\hline
Frontend & React 19, TypeScript 5.7 & Base comune tutti i prodotti \\
\hline
Frontend & React Query & State management server-side \\
\hline
Frontend & TanStack Router & Routing type-safe (CRM) \\
\hline
Backend & Django 4.x, PostgreSQL 15 & Stack verticale Python \\
\hline
Infrastructure & AWS (eu-south-1) & Cloud provider \\
\hline
Infrastructure & Docker, AWS ECR & Containerizzazione \\
\hline
Infrastructure & S3, CloudFront & Storage e distribuzione globale \\
\hline
Infrastructure & Lambda & Integrazioni AI serverless \\
\hline
Tools & Mixpanel & Analytics GDPR-compliant \\
\hline
Tools & Customer.io & Email automation (500k+ iscritti) \\
\hline
\end{tabular}
\end{table}

\subsection{Motivazioni architetturali}

Lo stack è stato selezionato per tre obiettivi principali. Sul frontend, React 
con TypeScript garantisce coerenza tecnologica tra tutti i prodotti aziendali, 
permettendo riutilizzo di componenti UI e competenze condivise del team. React 
Query gestisce la sincronizzazione dei dati con il server attraverso un sistema 
di cache intelligente, riducendo le chiamate ripetute alle API e migliorando la 
velocità percepita dall'utente. TanStack Router, adottato progressivamente sul 
gestionale CRM, fornisce routing type-safe con validazione statica dei percorsi.

Sul backend, Django con PostgreSQL costituisce lo stack verticale Python, 
scelto per la consolidata expertise del team e l'ecosistema ricco di librerie 
per intelligenza artificiale e machine learning.

L'infrastruttura AWS fornisce scalabilità automatica e conformità GDPR 
attraverso data center localizzati in Europa (region eu-south-1). Le landing 
pages sono distribuite come container Docker gestiti tramite AWS Elastic 
Container Registry, con S3 per l'archiviazione di immagini e file statici, e 
CloudFront CDN per distribuzione globale dei contenuti. AWS Lambda gestisce le 
integrazioni AI serverless con scalabilità automatica in base al carico.

\section{Development tools e workflow}

Il team utilizza una suite integrata di strumenti per garantire efficienza, 
qualità del codice e collaborazione efficace. La Tabella~\ref{tab:dev-tools} 
riassume le principali tecnologie adottate.

\begin{table}[h]
\centering
\caption{Strumenti di sviluppo e workflow}
\label{tab:dev-tools}
\begin{tabular}{|l|l|p{6cm}|}
\hline
\textbf{Categoria} & \textbf{Tecnologia} & \textbf{Utilizzo} \\
\hline
IDE & VS Code + Cursor AI & Editor principale con AI assistance \\
\hline
Version Control & GitLab self-hosted & Repository privato, Merge Request \\
\hline
Project Management & Jira & Task tracking, sprint planning \\
\hline
Deployment & Docker + AWS ECR & Containerizzazione e registry \\
\hline
Communication & Discord, Google Meet & Daily standup, sync team \\
\hline
Documentation & Notion & Knowledge base, onboarding \\
\hline
\end{tabular}
\end{table}

Il workflow di sviluppo segue approccio Agile con sprint bisettimanali. Il 
codice viene sviluppato su branch Git dedicati per ogni nuova funzionalità 
(feature branches), e ogni modifica richiede Merge Request su GitLab con 
revisione del codice (code review) obbligatoria da parte di almeno un 
collega prima dell'integrazione nel branch principale. Jira traccia le user 
stories con stima della complessità in story points, mentre grafici di 
avanzamento (burndown charts) monitorano il progresso di ogni sprint. La 
documentazione tecnica è centralizzata su Notion per garantire accessibilità 
rapida a tutto il team.