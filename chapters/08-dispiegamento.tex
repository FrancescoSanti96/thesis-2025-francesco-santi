\chapter{Dispiegamento in opera}

\section{Architettura deployment}
Il deployment delle landing pages Next.js è stato progettato per garantire 
scalabilità, performance e reliability attraverso un'architettura cloud-native 
basata su containerizzazione e distribuzione globale.

\subsection{Stack deployment}
\begin{itemize}
  \item \textbf{Next.js rendering ibrido}: Combinazione SSG per contenuti statici 
  con ISR (revalidation 1 ora) per aggiornamenti dinamici, bilanciando performance 
  e freshness dei contenuti
  \item \textbf{Containerizzazione Docker}: Multi-stage builds per ottimizzazione 
  immagini e consistency ambienti sviluppo-produzione
  \item \textbf{AWS Cloud Infrastructure}: Ecosystem integrato per container 
  registry, CDN globale e monitoring
  \item \textbf{Multi-environment strategy}: Separazione staging/production per 
  validation workflow e risk mitigation
\end{itemize}

\section{Decisioni infrastruttura}

\subsection{Cloud provider selection}
La scelta di AWS rispetto ad alternative come Vercel o Netlify è stata guidata 
da requisiti specifici:

\textbf{Controllo infrastruttura}: Necessità di configurazioni custom per tracking 
Mixpanel, integrazione API Django e compliance GDPR con data residency EU.

\textbf{Scalabilità cost-effective}: Previsione crescita traffico organico post-SEO 
optimization richiedeva pricing model predittibile vs pay-per-request.

\textbf{Integration ecosystem}: Già presente infrastruttura AWS per backend Django, 
database PostgreSQL e servizi AI, garantendo coerenza architetturale.

\subsection{Container strategy}
Implementazione Docker multi-stage per ottimizzazioni deployment:
\begin{itemize}
  \item \textbf{Build optimization}: Separazione dependencies/runtime riduce 
  immagine finale del 60\%, accelerando deploy e riducendo attack surface
  \item \textbf{Layer caching}: Build incrementali veloci attraverso cache 
  intelligente dependencies npm
  \item \textbf{Environment consistency}: Eliminazione "works on my machine" 
  tra sviluppo, staging e production
\end{itemize}

\subsection{CDN e distribuzione globale}
CloudFront CDN per performance globali:
\begin{itemize}
  \item \textbf{Edge caching}: Asset statici (immagini, CSS, JS) serviti da 
  edge locations per riduzione latency
  \item \textbf{Cache invalidation}: Automatic invalidation su deploy per 
  freshness contenuti aggiornati
  \item \textbf{Geographic optimization}: 180+ edge locations riducono TTFB 
  globalmente, critico per SEO e conversioni
\end{itemize}

\section{Pipeline CI/CD}

\subsection{Workflow automation}
Design pipeline per garantire quality gates e deployment sicuri:

\textbf{Trigger automatico}: Git push su branch main attiva pipeline completa 
build → test → security scan → deploy staging → validation → production.

\textbf{Parallel execution}: Build Docker e testing qualità codice (ESLint, 
TypeScript check) eseguiti in parallelo per riduzione tempo pipeline.

\textbf{Artifact promotion}: Stessa immagine Docker promossa da staging a 
production elimina discrepanze ambiente e garantisce consistency.

\subsection{Environment strategy}
Strategia multi-ambiente per risk mitigation:
\begin{itemize}
  \item \textbf{Staging validation}: Deploy automatico staging per validation 
  funzionalità, performance testing e UAT stakeholder
  \item \textbf{Production gating}: Manual approval obbligatorio per production 
  deploy dopo validation staging completa
  \item \textbf{Feature flags}: Deployment features disabilitate di default 
  con abilitazione graduale post-validation
\end{itemize}

\subsection{Quality gates}
Automated quality assurance nel pipeline:
\begin{itemize}
  \item \textbf{Code quality}: ESLint, Prettier e TypeScript strict mode 
  bloccano deploy se quality threshold non raggiunti
  \item \textbf{Performance testing}: Lighthouse CI automatico con threshold 
  Performance > 90, Accessibility > 95
  \item \textbf{Security scanning}: Container vulnerability scan con blocco 
  deploy per vulnerabilità HIGH/CRITICAL
  \item \textbf{Functional testing}: Automated smoke tests post-deploy per 
  validation funzionalità core (form, tracking, navigation)
\end{itemize}

\section{Gestione deployment}

\subsection{Zero-downtime strategy}
Architettura deployment per eliminare service interruption:

\textbf{Rolling updates}: Nuove istanze container avviate prima di terminare 
precedenti, garantendo continuità servizio durante deploy.

\textbf{Health checks}: Automated health verification prima di routing traffico 
verso nuove istanze, prevenendo deploy problematici.

\textbf{Load balancer integration}: Automatic service discovery e traffic 
routing verso istanze healthy, isolamento istanze problematiche.

\subsection{Rollback architecture}
Strategia rollback per incident response rapido:
\begin{itemize}
  \item \textbf{Immutable deployments}: Ogni deploy crea nuova versione immutabile, 
  rollback rapido a versione precedente verificata
  \item \textbf{Database migration safety}: Migrations backward-compatible per 
  evitare data loss in rollback scenarios
  \item \textbf{CDN cache management}: Automated cache invalidation e rollback 
  asset statici per consistency completa
  \item \textbf{Monitoring-driven rollback}: Automated rollback trigger basato 
  su threshold metriche (error rate, response time)
\end{itemize}

\subsection{Configuration management}
Gestione sicura configurazioni e secrets:
\begin{itemize}
  \item \textbf{Environment separation}: Configurazioni staging/production 
  completamente separate per prevenire data leakage
  \item \textbf{Secrets management}: AWS Secrets Manager per API keys, database 
  credentials con automatic rotation
  \item \textbf{Runtime injection}: Environment variables iniettate runtime 
  vs build-time per security e flexibility
  \item \textbf{Audit trail}: Logging completo accesso secrets e configuration 
  changes per compliance
\end{itemize}

\section{Monitoring e performance}

\subsection{Real-time monitoring}
Monitoring stack per observability deployment:
\begin{itemize}
  \item \textbf{Application monitoring}: Custom metrics performance, error 
  tracking e user behavior tramite Mixpanel integration
  \item \textbf{Infrastructure monitoring}: Container health, resource utilization 
  e network performance con alerting automatico
  \item \textbf{Performance tracking}: Core Web Vitals monitoring continuo con 
  alert su degradation threshold
  \item \textbf{Security monitoring}: Automated detection anomalie traffico, 
  failed authentication attempts e potential attacks
\end{itemize}

\subsection{Post-deploy validation}
Validation automatica successful deployment:
\begin{itemize}
  \item \textbf{Smoke testing}: Automated verification funzionalità critical 
  path (homepage load, form submission, tracking events)
  \item \textbf{Performance validation}: Lighthouse audit automatico post-deploy 
  con alert se performance regression
  \item \textbf{Analytics verification}: Validation tracking Mixpanel funzionante 
  e dati flowing correttamente
  \item \textbf{Error monitoring}: Monitoring spike error rate nelle prime 
  30 minuti post-deploy per early detection issues
\end{itemize}

\section{Risultati e ottimizzazioni}

\subsection{Performance achieved}
L'architettura deployment ha conseguito risultati significativi:
\begin{itemize}
  \item \textbf{Deploy reliability}: 99.8\% success rate deployment con zero 
  downtime negli ultimi 6 mesi
  \item \textbf{Deploy speed}: Riduzione tempo deploy da 15-20 minuti a 5-7 
  minuti tramite pipeline optimization
  \item \textbf{Performance consistency}: Lighthouse scores stabili 94/100 
  media, variazione < 2 punti tra deploy
  \item \textbf{Global performance}: TTFB medio < 200ms per utenti EU, < 500ms 
  globalmente tramite CDN optimization
\end{itemize}

\subsection{Incident response}
Capability incident management dimostrate:
\begin{itemize}
  \item \textbf{MTTR (Mean Time To Recovery)}: 8 minuti media per rollback 
  completo in caso problemi critici
  \item \textbf{Detection speed}: Automated monitoring rileva issues medi 
  in < 2 minuti vs detection manuale
  \item \textbf{Blast radius limitation}: Feature flags e gradual rollout 
  limitano impatto issues a < 5\% utenti
\end{itemize}

\subsection{Lessons learned e miglioramenti}
Evoluzione architecture basata su esperienza operativa:
\begin{itemize}
  \item \textbf{Cache strategy refinement}: Ottimizzazione TTL CloudFront 
  per bilanciare performance e content freshness
  \item \textbf{Monitoring granularity}: Aumento granularità metrics per 
  detection più rapida performance regression
  \item \textbf{Automation expansion}: Automated rollback basato su threshold 
  per riduzione MTTR senza intervento manuale
  \item \textbf{Cost optimization}: Reserved instances e spot instances per 
  riduzione costi infrastruttura del 30\% mantenendo performance
\end{itemize}

\bigskip
L'architettura deployment implementata ha fornito foundation scalabile e 
reliable per ecosistema landing pages, supportando crescita traffico organico 
e garantendo experience utente consistent con operational excellence.