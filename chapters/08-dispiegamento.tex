\chapter{Dispiegamento in opera}

\section{Pipeline CI/CD}
\subsection{Automazione build e deploy}
Il processo di deploy è completamente automatizzato attraverso GitHub 
Actions e Vercel:

\begin{itemize}
  \item \textbf{Trigger}: Push su branch \texttt{main} (production) o 
        \texttt{staging}
  \item \textbf{Build automatico}: Next.js build con ottimizzazioni 
        produzione
  \item \textbf{Testing pre-deploy}: ESLint, TypeScript check, build 
        verification
  \item \textbf{Deployment}: Deploy automatico su Vercel (production o 
        preview)
  \item \textbf{Notifiche}: Alert su Discord per esito deploy 
        (success/failure)
\end{itemize}

\subsection{Stages della pipeline}
La pipeline GitHub Actions esegue i seguenti step:

\begin{enumerate}
  \item \textbf{Checkout}: Clone del repository
  \item \textbf{Setup Node.js}: Installazione Node.js 18 e dependencies
  \item \textbf{Lint}: ESLint per code quality check
  \item \textbf{Type Check}: TypeScript compiler per type errors
  \item \textbf{Build}: Next.js build per verifica assenza errori
  \item \textbf{Deploy Vercel}: Deploy automatico su Vercel (staging o 
        production)
  \item \textbf{Lighthouse CI}: Verifica performance score post-deploy
  \item \textbf{Notification}: Alert su Discord con URL deploy e status
\end{enumerate}

Durata media pipeline: 3-5 minuti.

\section{Hosting e infrastruttura}
\subsection{Provider e configurazione}
\begin{itemize}
  \item \textbf{Hosting}: Vercel (Next.js native platform)
  \item \textbf{Architettura}: Serverless con Edge Functions
  \item \textbf{CDN}: Vercel Edge Network (distribuzione globale)
  \item \textbf{Storage asset}: AWS S3 per immagini e media
  \item \textbf{Database}: AWS RDS PostgreSQL per backend Django
  \item \textbf{Backend API}: AWS EC2 per Django REST API
\end{itemize}

\subsection{Configurazione dominio}
\begin{itemize}
  \item Domain principale: \texttt{datapizza.tech}
  \item Routing multilingua: \texttt{/it/} e \texttt{/en/}
  \item Subdomain Jobs: \texttt{jobs.datapizza.tech}
  \item SSL/TLS: Certificato automatico Let's Encrypt via Vercel
  \item DNS: Configurazione A record e CNAME su Vercel DNS
\end{itemize}

\section{Gestione ambienti}
\subsection{Ambienti di sviluppo}
\textbf{Development (locale)}
\begin{itemize}
  \item Ambiente su macchina developer con \texttt{npm run dev}
  \item Database locale PostgreSQL o Docker container
  \item Hot reload Next.js per sviluppo rapido
  \item Mock API con MSW (Mock Service Worker) per testing isolato
  \item Environment variables in \texttt{.env.local}
\end{itemize}

\textbf{Staging}
\begin{itemize}
  \item Ambiente pre-produzione per testing QA
  \item Replica configurazione production (Vercel, AWS)
  \item URL: \texttt{staging.datapizza.tech}
  \item Database staging separato da production
  \item Testing manual e UAT (User Acceptance Testing)
  \item Mixpanel project separato per non inquinare dati production
\end{itemize}

\textbf{Production}
\begin{itemize}
  \item Ambiente live accessibile agli utenti finali
  \item URL: \texttt{datapizza.tech} e \texttt{jobs.datapizza.tech}
  \item Performance monitoring attivo con Vercel Analytics
  \item Backup automatizzati database ogni 24h
  \item High availability: 99.9\% uptime SLA Vercel
\end{itemize}

\subsection{Variabili ambiente}
Gestione configurazioni sensibili per ambiente:

\begin{itemize}
  \item \textbf{API keys}: Mixpanel token, CookieYes ID, Mailchimp API key
  \item \textbf{Database}: Connection string PostgreSQL (diverso per 
        staging/prod)
  \item \textbf{Backend API}: Base URL Django REST API
  \item \textbf{Feature flags}: Booleani per abilitare/disabilitare 
        features
  \item \textbf{Tools}: Vercel Environment Variables per gestione 
        secrets sicura
\end{itemize}

Nessuna variabile sensibile committata su Git (verificato con git-secrets).

\section{Strategia di deploy}
\subsection{Deploy progressivi}
\begin{itemize}
  \item \textbf{Rollout graduale}: Deploy landing per landing, non tutte 
        insieme
  \item \textbf{Preview deploys}: Ogni PR su GitHub genera preview URL 
        Vercel unico
  \item \textbf{Canary deployment}: Non implementato (complessità vs 
        beneficio)
  \item \textbf{Feature flags}: Usati per abilitare gradualmente nuove 
        features
\end{itemize}

Esempio rollout: Home Page → Recruiting → Community → AI Adoption → 
AI Engineering → Jobs (1 landing per settimana).

\subsection{Rollback strategy}
In caso di problemi critici post-deploy:

\begin{itemize}
  \item \textbf{Rollback automatico}: Se Lighthouse CI score < 80, 
        deploy bloccato
  \item \textbf{Rollback manuale}: Git revert + redeploy in < 5 minuti
  \item \textbf{Git tags}: Ogni deploy production taggato 
        (es. \texttt{v1.2.3})
  \item \textbf{Vercel rollback}: Rollback istantaneo a deployment 
        precedente da dashboard
  \item \textbf{Tempo medio rollback}: 3-5 minuti dall'identificazione 
        issue
\end{itemize}

\subsection{Zero-downtime deployment}
\begin{itemize}
  \item Vercel garantisce zero-downtime nativo con atomic deployments
  \item Nuovo deploy diventa live solo dopo build success completo
  \item Vecchia versione rimane live durante build della nuova
  \item Database migrations gestite separatamente da deploy frontend
  \item Cache warming automatico Vercel Edge Network
\end{itemize}

\section{Automazioni e monitoring}

\subsection{Automazioni deploy}
\begin{itemize}
  \item \textbf{Cache invalidation}: CloudFront cache clear automatico 
        post-deploy
  \item \textbf{Sitemap regeneration}: Update automatico 
        \texttt{sitemap.xml} con nuovo deploy
  \item \textbf{Webhook notifications}: Alert Discord con dettagli deploy 
        (branch, commit, autore)
  \item \textbf{Analytics tracking}: Verifica Mixpanel init corretto 
        post-deploy
  \item \textbf{Lighthouse CI}: Run automatico e report score su PR GitHub
\end{itemize}

\subsection{Monitoraggio post-deploy}
\textbf{Error Tracking}
\begin{itemize}
  \item Vercel Error Tracking per runtime errors frontend
  \item Alert real-time su Discord se error rate > 1\%
  \item Source maps per debugging in produzione
  \item Error rate threshold: se > 5\%, consideriamo rollback
\end{itemize}

\textbf{Performance Monitoring}
\begin{itemize}
  \item Vercel Analytics per Core Web Vitals real-user monitoring
  \item Lighthouse CI per tracking performance nel tempo (trend)
  \item API response time monitoring su backend Django
  \item Dashboard Vercel per visualizzazione metriche real-time
\end{itemize}

\textbf{Analytics Real-Time}
\begin{itemize}
  \item Mixpanel live view per monitorare eventi post-deploy
  \item Verifica funnel conversione funzionanti correttamente
  \item Monitoring traffico per landing page (spike detection)
  \item Alert su anomalie metriche: spike > 200\% o drop > 50\% 
        improvvisi
\end{itemize}

\subsection{Health Checks}
\begin{itemize}
  \item Endpoint \texttt{/api/health} per status check automatico
  \item Ping ogni 5 minuti da UptimeRobot per verifica uptime
  \item Database connection check incluso in health endpoint
  \item SLA uptime target: 99.9\% (massimo ~43 minuti downtime/mese)
  \item Uptime effettivo raggiunto: 99.95\% (gennaio-giugno 2025)
\end{itemize}

\section{Sicurezza e compliance}
\begin{itemize}
  \item \textbf{HTTPS enforcement}: Redirect automatico HTTP → HTTPS
  \item \textbf{Security headers}: CSP (Content Security Policy), 
        HSTS, X-Frame-Options, X-Content-Type-Options
  \item \textbf{Rate limiting}: Vercel Edge Middleware per protezione 
        API abuse
  \item \textbf{GDPR compliance}: Cookie consent, opt-out tracking, 
        IP anonymization
  \item \textbf{Dependency scanning}: Dependabot GitHub per 
        vulnerabilità dependencies
\end{itemize}

\section{Backup e disaster recovery}
\begin{itemize}
  \item \textbf{Backup database}: Automatici ogni 24h su AWS RDS, 
        retention 30 giorni
  \item \textbf{Git repository}: Backup automatico codice sorgente su 
        GitHub
  \item \textbf{Asset backup}: S3 versioning abilitato per rollback asset
  \item \textbf{Recovery Time Objective (RTO)}: < 15 minuti per ripristino 
        completo
  \item \textbf{Recovery Point Objective (RPO)}: Massimo 24h di dati 
        persi (1 backup/giorno)
\end{itemize}

Test disaster recovery effettuato: ripristino completo da backup in 
12 minuti (sotto target RTO).