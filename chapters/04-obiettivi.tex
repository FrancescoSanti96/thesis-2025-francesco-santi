\chapter{Obiettivo della tesi}

\section{Obiettivi generali del progetto}
Il progetto di redesign delle landing pages ha mirato a risolvere le problematiche identificate nel contesto applicativo, trasformando la presenza web aziendale da monolite generico a ecosistema di landing specializzate.

Gli obiettivi generali comprendevano:
\begin{itemize}
  \item Contribuire allo sviluppo frontend e backend dell'ecosistema web aziendale
  \item Ridurre technical debt e migliorare qualità del codice esistente
  \item Ridisegnare la presenza web tramite landing pages specializzate per verticale
  \item Acquisire e consolidare competenze tecniche e trasversali in ambiente aziendale
\end{itemize}

\section{Obiettivi principali delle landing pages}

\subsection{Posizionamento e comunicazione}
\textbf{Differenziazione per verticale}
\begin{itemize}
  \item Creare value proposition chiare e specifiche per ciascun servizio (AI Engineering, AI Adoption, Tech Recruiting, Community)
  \item Separare chiaramente la comunicazione B2B enterprise da quella B2C candidati
  \item Posizionare Datapizza come "AI Transformation Company" attraverso contenuti mirati
\end{itemize}

\textbf{Brand consistency}
\begin{itemize}
  \item Implementare design system coerente ma flessibile per adattarsi ai diversi target
  \item Mantenere elementi di continuità visiva (palette, typography, componenti)
  \item Integrare social proof (500k+ community) come elemento di credibilità trasversale
\end{itemize}

\subsection{Performance e user experience}
\textbf{Ottimizzazione tecnica}
\begin{itemize}
  \item Raggiungere punteggi Lighthouse superiori a 90 per Performance, Accessibility, Best Practices, SEO
  \item Ottimizzare Core Web Vitals: FCP < 2s, LCP < 2.5s, CLS < 0.1
  \item Implementare lazy loading e code splitting per ridurre tempo di caricamento iniziale
\end{itemize}

\textbf{Esperienza utente}
\begin{itemize}
  \item Progettare customer journey ottimizzati per ciascun target
  \item Implementare CTA strategiche e form di conversione dedicati per verticale
  \item Garantire responsive design per tutti i dispositivi (mobile-first approach)
\end{itemize}

\subsection{Scalabilità e manutenibilità}
\textbf{Architettura modulare}
\begin{itemize}
  \item Sviluppare framework riutilizzabile per future landing pages
  \item Implementare design system con componenti atomici (atoms, molecules, organisms)
  \item Strutturare codebase per facilitare aggiornamenti e nuove implementazioni
\end{itemize}

\textbf{Gestione contenuti}
\begin{itemize}
  \item Abilitare aggiornamenti rapidi dei contenuti senza deploy
  \item [TODO: Specificare se CMS headless implementato]
  \item Permettere A/B testing su elementi chiave delle landing
\end{itemize}

\section{Obiettivi secondari}

\subsection{Tracking e analytics}
\textbf{Implementazione Mixpanel}
\begin{itemize}
  \item Mappare eventi utente specifici per ciascuna landing page
  \item Implementare tracking GDPR-compliant con cookie consent management
  \item Creare funnel di conversione per misurare efficacia per verticale
  \item Abilitare segmentazione utenti basata su comportamento e provenienza
\end{itemize}

\textbf{Dashboard e reportistica}
\begin{itemize}
  \item Integrare Redash per analisi avanzata e dashboard personalizzate
  \item Implementare real-time monitoring delle performance
  \item Creare alert automatici per anomalie nel traffico o conversioni
\end{itemize}

\subsection{SEO e visibilità}
\begin{itemize}
  \item Ottimizzare struttura URL e meta tags per ciascun servizio
  \item Implementare structured data (JSON-LD) per rich snippets
  \item Creare sitemap dinamica e gestione robots.txt
  \item Ottimizzare per parole chiave specifiche per verticale (AI consulting, tech recruiting, etc.)
\end{itemize}

\subsection{Integrazione con ecosistema esistente}
\begin{itemize}
  \item Collegare seamlessly con piattaforme Jobs e Company
  \item Integrare newsletter "Commit" e sistema di iscrizione
  \item Connettere con Discord community e eventi
  \item Implementare single sign-on dove applicabile
\end{itemize}

\section{Criteri di successo e KPI}

\subsection{Metriche di performance tecnica}
\begin{itemize}
  \item Lighthouse Performance Score > 90
  \item First Contentful Paint < 2 secondi
  \item Largest Contentful Paint < 2.5 secondi
  \item Cumulative Layout Shift < 0.1
  \item Time to Interactive < 3 secondi
\end{itemize}

\subsection{Metriche di business}
\textbf{Conversion rate}
\begin{itemize}
  \item [TODO: Baseline conversion rate pre-redesign se disponibile]
  \item Target miglioramento conversion rate del X\% per landing B2B
  \item Target iscrizioni newsletter Commit: Y iscrizioni/mese
  \item Riduzione bounce rate del Z\% rispetto alla situazione pre-redesign
\end{itemize}

\textbf{Engagement}
\begin{itemize}
  \item Aumento tempo medio sessione
  \item Miglioramento scroll depth medio
  \item Incremento interazioni con CTA principali
\end{itemize}

\subsection{Metriche di qualità}
\begin{itemize}
  \item Zero errori critici in produzione post-deploy
  \item Accessibilità WCAG 2.1 AA compliance al 100\%
  \item Cross-browser compatibility (Chrome, Firefox, Safari, Edge)
  \item Mobile responsiveness su tutti i breakpoints
\end{itemize}

\section{Vincoli e limitazioni}
\subsection{Vincoli tecnici}
\begin{itemize}
  \item Compatibilità con stack esistente (Django backend, AWS infrastructure)
  \item Rispetto budget performance per hosting e CDN
  \item Timeline di sviluppo: [TODO: specificare durata sprint dedicati]
  \item Risorse team: 1 Software Engineer principale (70\% frontend, 30\% backend)
\end{itemize}

\subsection{Vincoli di compliance}
\begin{itemize}
  \item Conformità GDPR per tracking utenti EU
  \item Privacy policy aggiornata per nuovi sistemi di tracking
  \item Cookie consent management compliant
  \item Data retention policies per analytics
\end{itemize}

\section{Roadmap di implementazione}
\textbf{Fase 1}: Architettura e design system
\begin{itemize}
  \item Setup Next.js e configurazione routing
  \item Implementazione design system base con ShadCN UI
  \item Creazione componenti core riutilizzabili
\end{itemize}

\textbf{Fase 2}: Sviluppo landing pages
\begin{itemize}
  \item Implementazione Home Page come hub centrale
  \item Sviluppo landing specializzate per verticali AI
  \item Implementazione landing recruiting e Jobs platform
\end{itemize}

\textbf{Fase 3}: Tracking e ottimizzazione
\begin{itemize}
  \item Integrazione Mixpanel e implementazione eventi
  \item Setup dashboard Redash
  \item Testing e ottimizzazione performance
\end{itemize}

\textbf{Fase 4}: Deploy e monitoring
\begin{itemize}
  \item Deploy su ambiente di staging per testing
  \item Go-live in produzione con rollout graduale
  \item Monitoring post-launch e iterazioni
\end{itemize}