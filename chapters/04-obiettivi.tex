\chapter{Obiettivi}

\section{Obiettivi strategici e di business}
Il progetto di redesign delle landing pages aveva l'obiettivo di 
trasformare la presenza web aziendale da un'unica pagina generica a un 
ecosistema di landing specializzate, risolvendo le problematiche 
identificate nel capitolo precedente.

\paragraph{Posizionamento e differenziazione}
Un primo obiettivo strategico riguardava il posizionamento chiaro dei 
quattro verticali aziendali. Era necessario creare value proposition 
specifiche per AI Engineering, AI Adoption, Tech Recruiting e Community, 
separando nettamente la comunicazione verso aziende clienti 
business-to-business (B2B) da quella rivolta a singoli 
candidati business-to-consumer (B2C)
Questo avrebbe permesso di consolidare l'identità aziendale come 
"AI Transformation Company" attraverso messaggi mirati e coerenti per 
ciascun target.

\paragraph{Marketing data-driven e conversione}
Per abilitare decisioni basate sui dati concreti, era necessario 
implementare un sistema di tracking strutturato e conforme al regolamento 
europeo GDPR (General Data Protection Regulation) per la protezione dei 
dati personali. Gli obiettivi specifici in quest'area erano:

\begin{itemize}
  \item Implementare tracking avanzato per customer journey completi, 
        tracciando ogni interazione utente dall'arrivo sulla landing 
        fino alla conversione.
  
  \item Creare funnel di conversione misurabili per ottimizzazione 
        continua, differenziati per ciascun verticale.
  
  \item Integrare strumenti analytics (Mixpanel e Redash) per fornire 
        al team marketing dati concreti su cui basare le decisioni.
  
  \item Migliorare il conversion rate complessivo attraverso 
        l'ottimizzazione iterativa basata sui dati raccolti.
  
  \item Aumentare i lead qualificati B2B per i servizi enterprise e 
        le iscrizioni alla newsletter community.
  
  \item Ridurre il bounce rate attraverso customer journey ottimizzati 
        per ciascun tipo di utente.
\end{itemize}

\paragraph{Supporto strategia commerciale}
Le nuove landing dovevano diventare il canale primario di acquisizione 
contatti qualificati per il team sales, abilitando campagne marketing con 
misurazione accurata del ritorno sull'investimento (ROI). L'obiettivo era 
fornire visibilità completa sull'efficacia di ogni verticale e permettere 
l'allocazione ottimale del budget marketing basata su dati misurabili.

\section{Obiettivi tecnici}
Gli obiettivi tecnici erano orientati a costruire un'infrastruttura 
solida, performante e scalabile per supportare gli obiettivi di business.

\subsection{Architettura e scalabilità}
L'obiettivo architetturale principale era costruire un sistema frontend 
moderno basato su un design system modulare. La scelta di ShadCN UI e 
Tailwind CSS come fondamenta permetteva di creare componenti riutilizzabili 
e mantenere consistenza visiva tra tutte le landing. L'architettura doveva 
supportare il multilingua attraverso percorsi dedicati (/it/ e /en/) e 
garantire la scalabilità futura: ogni nuova landing page doveva poter 
riutilizzare la maggior parte del codice esistente, riducendo 
significativamente i tempi di sviluppo per futuri verticali.

\subsection{Performance e user experience}
Dal punto di vista delle performance, l'obiettivo era garantire 
caricamenti rapidi attraverso tecniche di ottimizzazione come code 
splitting, lazy loading e compressione delle immagini. L'esperienza 
doveva essere fluida su tutti i dispositivi (desktop, tablet, mobile). 

Particolare attenzione era richiesta per l'accessibilità: conformità 
agli standard WCAG 2.1 (Web Content Accessibility Guidelines)
per supportare utenti con disabilità attraverso screen reader, navigazione 
da tastiera e contrasto colori adeguato. Questi standard internazionali 
definiscono criteri tecnici per rendere i contenuti web accessibili a 
persone con diverse tipologie di disabilità visive, uditive, motorie e 
cognitive.

\subsection{Tracking e analytics}
L'obiettivo sul fronte analytics era integrare Mixpanel con una event 
taxonomy strutturata per categorizzare i comportamenti utente in tre 
categorie principali: page view (visualizzazioni), interaction 
(interazioni con elementi della pagina) e conversion (azioni di valore 
come submit form o iscrizione newsletter). Ogni verticale doveva avere 
funnel di conversione specifici e misurabili, differenziati per tipo di 
utente (B2B enterprise, B2C candidati, Community). La conformità GDPR 
era un requisito non negoziabile, da garantire attraverso consenso 
esplicito ai cookie, anonimizzazione degli indirizzi IP e gestione 
delle preferenze privacy degli utenti.