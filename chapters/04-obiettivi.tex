\chapter{Obiettivi}

\section{Obiettivi del progetto}
Il progetto di redesign delle landing pages aveva l'obiettivo di trasformare la presenza web aziendale da un'unica pagina generica a un ecosistema di landing specializzate, risolvendo le problematiche identificate nel capitolo precedente.

\subsection{Obiettivi strategici e di business}

\textbf{Posizionamento e differenziazione}
\begin{itemize}
  \item Creare value proposition specifiche per ciascun verticale (AI Engineering, AI Adoption, Tech Recruiting, Community)
  \item Separare la comunicazione business-to-business (B2B) enterprise da quella business-to-consumer (B2C) candidati
  \item Consolidare l'identità aziendale come "AI Transformation Company"
\end{itemize} 

\textbf{Marketing data-driven e conversione}
\begin{itemize}
  \item Implementare tracking avanzato GDPR-compliant per customer journey completi
  \item Creare funnel di conversione misurabili per ottimizzazione continua
  \item Fornire analytics (Mixpanel + Redash) per decisioni marketing informate
  \item Migliorare conversion rate complessivo per verticale
  \item Aumentare lead qualificati B2B e iscrizioni newsletter community
  \item Ridurre bounce rate attraverso customer journey ottimizzati
\end{itemize}

\textbf{Supporto strategia commerciale}
\begin{itemize}
  \item Fornire canale primario acquisizione contatti qualificati (lead) per team sales
  \item Abilitare campagne marketing con tracking ROI accurato
\end{itemize}

\section{Obiettivi tecnici}

\subsection{Architettura e scalabilità}
Costruire un'architettura frontend moderna basata su design system modulare (ShadCN UI + Tailwind CSS) con componenti riutilizzabili e supporto multilingua (/it/, /en/). L'obiettivo era creare un applicativo scalabile, con codice condiviso e riutilizzato.

\subsection{Performance e user experience}
Ottimizzare i tempi di caricamento attraverso code splitting, lazy loading e ottimizzazione immagini, garantendo esperienza fluida su tutti i dispositivi. Implementare accessibilità WCAG 2.1 AA per utenti con disabilità (screen reader, keyboard navigation, contrasto colori).

\subsection{Tracking e analytics}
Integrare Mixpanel con event taxonomy strutturata (page view, interaction, conversion) per tracciare comportamenti utente. Configurare funnel di conversione specifici per verticale (B2B, B2C, Community) con compliance GDPR (cookie consent, IP anonymization).