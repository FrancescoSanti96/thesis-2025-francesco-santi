\chapter{Obiettivi}

\section{Obiettivi del progetto}
Il progetto di redesign delle landing pages aveva l'obiettivo di 
trasformare la presenza web aziendale da un'unica pagina generica a 
un ecosistema di landing specializzate, risolvendo le problematiche 
identificate nel capitolo precedente.

\subsection{Obiettivi principali}
\textbf{Posizionamento e differenziazione}
\begin{itemize}
  \item Creare value proposition specifiche per ciascun verticale 
        (AI Engineering, AI Adoption, Tech Recruiting, Community)
  \item Separare comunicazione B2B enterprise da quella B2C candidati
  \item Posizionare Datapizza come "AI Transformation Company" 
        attraverso contenuti mirati
\end{itemize}

\textbf{Performance e scalabilità}
\begin{itemize}
  \item Raggiungere Lighthouse Performance Score superiore a 90
  \item Ottimizzare Core Web Vitals: FCP < 2s, LCP < 2.5s, CLS < 0.1
  \item Implementare architettura modulare riutilizzabile per future 
        landing pages
\end{itemize}

\textbf{User experience}
\begin{itemize}
  \item Progettare customer journey ottimizzati per ciascun target
  \item Garantire responsive design mobile-first per tutti i dispositivi
  \item Implementare accessibilità WCAG 2.1 livello AA
\end{itemize}

\subsection{Obiettivi secondari}
\textbf{Tracking e analytics}
\begin{itemize}
  \item Implementare tracking eventi utente con Mixpanel 
        GDPR-compliant
  \item Creare funnel di conversione per misurare efficacia per 
        verticale
  \item Abilitare segmentazione utenti e dashboard analytics con Redash
\end{itemize}

\textbf{SEO e integrazione}
\begin{itemize}
  \item Ottimizzare struttura URL, meta tags e structured data (JSON-LD)
  \item Integrare seamlessly con piattaforme esistenti (Jobs, Company)
  \item Collegare newsletter "Commit" e Discord community
\end{itemize}

\section{Criteri di successo}
Gli obiettivi erano misurabili attraverso KPI definiti a priori:

\subsection{Metriche di performance tecnica}
\begin{itemize}
  \item Lighthouse Performance Score > 90
  \item First Contentful Paint < 2 secondi
  \item Largest Contentful Paint < 2.5 secondi
  \item Cumulative Layout Shift < 0.1
  \item Time to Interactive < 3 secondi
\end{itemize}

\subsection{Metriche di business}
\begin{itemize}
  \item Miglioramento conversion rate per landing B2B
  \item Aumento iscrizioni newsletter "Commit"
  \item Riduzione bounce rate rispetto alla situazione iniziale
  \item Incremento tempo medio sessione e scroll depth
  \item Crescita interazioni con CTA principali
\end{itemize}

\subsection{Metriche di qualità}
\begin{itemize}
  \item Zero errori critici in produzione post-deploy
  \item Accessibilità WCAG 2.1 AA compliance al 100\%
  \item Cross-browser compatibility (Chrome, Firefox, Safari, Edge)
  \item Mobile responsiveness su tutti i breakpoints
\end{itemize}

\section{Vincoli e limitazioni}
Il progetto operava entro vincoli definiti che ne hanno influenzato 
l'esecuzione:

\subsection{Vincoli tecnici}
\begin{itemize}
  \item Compatibilità con stack esistente (Django backend, AWS 
        infrastructure)
  \item Budget performance per hosting e CDN
  \item Timeline: sviluppo in sprint bisettimanali (gennaio-giugno 2025)
  \item Risorse: 1 Software Engineer principale (70\% frontend, 
        30\% backend)
\end{itemize}

\subsection{Vincoli di compliance}
\begin{itemize}
  \item Conformità GDPR per tracking utenti EU
  \item Cookie consent management obbligatorio
  \item Privacy policy aggiornata per nuovi sistemi di tracking
  \item Data retention policies per analytics
\end{itemize}