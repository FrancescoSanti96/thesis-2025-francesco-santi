\documentclass[12pt,a4paper,openright,twoside]{book}
\usepackage[utf8]{inputenc}
\usepackage{disi-thesis}
\usepackage{code-lstlistings}
\usepackage{notes}
\usepackage{shortcuts}
\usepackage{acronym}

\school{\unibo}
\programme{DIPARTIMENTO DI INFORMATICA – SCIENZA E INGEGNERIA

Laurea in Tecnologie dei Sistemi Informatici}
\title{Fancy Title}
\author{Francesco Santi}
\date{\today}
\subject{Supervisor's course name}
\supervisor{Prof. Supervisor Here}
\cosupervisor{Dott. CoSupervisor 1}
\academicyear{2024--2025}

% Definition of acronyms
\acrodef{IoT}{Internet of Thing}
\acrodef{vm}[VM]{Virtual Machine}


\mainlinespacing{1.241} % line spacing in mainmatter, comment to default (1)

\begin{document}

% ===== FRONTESPIZIO ORIGINALE =====
\frontmatter\frontispiece

% ===== Dedica (opzionale) =====
\begin{dedication}
Dedico questo traguardo inanziutto a Maggie che in questo mio percorso mi è sempre stata vicina, inoltre famiglia e amici.
\end{dedication}

% ===== Introduzione (front matter, prima dell'indice) =====
% Se il tuo stile NON avesse \chapterWithoutNumber, usa la forma generica qui sotto:
\chapter*{Introduzione}
\addcontentsline{toc}{chapter}{Introduzione}
Questa è l’introduzione di una pagina che riassume la relazione:
scopo, periodo svolto (3 gennaio 2025 – 20 giugno 2025), ruolo in Datapizza
e una panoramica di ciò che verrà descritto nei capitoli (contesto aziendale,
contesto applicativo, obiettivi, tecnologie, progettazione, sviluppo, dispiegamento,
verifica, conclusioni).

% ===== Indice =====
\tableofcontents

\mainmatter
% ----------------------------------------------------------------------------------------
\chapter{Introduzione}
La presente relazione è redatta ai fini della richiesta di convalida dell'attività lavorativa svolta da Francesco Santi nel periodo compreso tra il 3 gennaio 2025 e il 20 giugno 2025.\\
Il rapporto di lavoro è tuttora in corso, in quanto attualmente impiegato con contratto full-time presso \textbf{Datapizza}.

% ----------------------------------------------------------------------------------------
\chapter{Contesto aziendale}
\section{Descrizione dell’azienda}
Datapizza è una startup innovativa con sede legale in Via Giuseppe Ripamonti 190, 20141 Milano (MI).  
La missione è rendere l’Italia competitiva nel settore tech, attraverso soluzioni avanzate e servizi mirati.  

Principali competenze:
\begin{itemize}
  \item \textbf{Tech Recruiting}: connessione tra aziende e talenti tech;
  \item \textbf{Tech Community}: oltre 500k iscritti, riferimento su notizie e tendenze tecnologiche;
  \item \textbf{AI Engineering}: consulenze personalizzate e sviluppo soluzioni di intelligenza artificiale;
  \item \textbf{AI Adoption}: percorsi interni per aumentare produttività e adozione tecnologica.
\end{itemize}

\section{Dimensioni e crescita}
Datapizza conta oltre 50 dipendenti, in costante crescita, con approccio dinamico tipico delle startup ma processi strutturati per garantire scalabilità e qualità.

% ----------------------------------------------------------------------------------------
\chapter{Contesto applicativo}
L’attività ha riguardato principalmente tre prodotti software:
\begin{itemize}
  \item \textbf{Datapizza Jobs} – piattaforma recruiting lato candidati;
  \item \textbf{Datapizza Company} – piattaforma recruiting lato aziende;
  \item \textbf{Datapizza Tech} – landing pages aziendali.
\end{itemize}

% ----------------------------------------------------------------------------------------
\chapter{Obiettivo della tesi}
Gli obiettivi erano:
\begin{itemize}
  \item contribuire allo sviluppo frontend e backend;
  \item ridurre technical debt e migliorare qualità del codice;
  \item ridisegnare la presenza web tramite landing pages specializzate;
  \item acquisire e consolidare competenze tecniche e trasversali.
\end{itemize}

% ----------------------------------------------------------------------------------------
\chapter{Tecnologie}
\section{Metodologia di lavoro}
Approccio \textbf{Agile} con sprint di due settimane, daily standup, planning, retrospettive e one-to-one periodici.  

\section{Stack tecnologico}
\textbf{Frontend}: React + TypeScript, React Query, Tailwind CSS, ShadCN UI, Next.js.  
\textbf{Backend}: Django (Python), PostgreSQL, AWS.  
\textbf{Strumenti}: VS Code, Figma, Cursor, GitHub/GitLab, Jira, Discord, Google Workspace, Mixpanel, Redash, DBeaver, Notion.

% ----------------------------------------------------------------------------------------
\chapter{Progettazione}
Avvio del \textbf{gestionale interno} aziendale:
\begin{itemize}
  \item setup scheletro progetto;
  \item definizione routing e architettura base;
  \item condivisione convenzioni di sviluppo.
\end{itemize}

% ----------------------------------------------------------------------------------------
\chapter{Sviluppo}
\section{Onboarding iniziale}
Studio delle codebase, architettura dei prodotti, workflow di sviluppo e deployment.

\section{Riduzione technical debt}
\begin{itemize}
  \item standardizzazione API calls con React Query;
  \item migrazione UI verso ShadCN;
  \item refactoring componenti e nomenclature;
  \item rimozione librerie inutilizzate (-15\% bundle size).
\end{itemize}

\section{Landing pages}
Creazione di 6 landing pages strategiche:
\begin{enumerate}
  \item Home page;
  \item Tech Recruiting;
  \item Community;
  \item AI Adoption;
  \item AI Engineering;
  \item Jobs.
\end{enumerate}
Con mappatura eventi Mixpanel in ottica GDPR.

\section{Customer support}
Attività di manutenzione e bug fixing durante tutto il periodo.

\section{Nuove feature}
Implementazione di funzionalità evolutive nei prodotti Jobs e Company, con rilascio in produzione e monitoraggio impatto utenti.

% ----------------------------------------------------------------------------------------
\chapter{Dispiegamento in opera}
Deploy progressivi tramite pipeline CI/CD aziendali. Monitoraggio post-deploy con analytics e metriche.

% ----------------------------------------------------------------------------------------
\chapter{Verifica sperimentale}
\begin{itemize}
  \item test di regressione e usabilità;
  \item monitoraggio eventi utente con Mixpanel;
  \item analisi funnel e conversioni con Redash.
\end{itemize}

% ----------------------------------------------------------------------------------------
\chapter{Conclusioni e sviluppi futuri}
L’esperienza in Datapizza ha consentito:
\begin{itemize}
  \item crescita tecnica (React, Next.js, Django, DevOps di base);
  \item sviluppo di soft skills (comunicazione, product mindset, lavoro agile);
  \item partecipazione a decisioni tecniche con impatto reale sul business.
\end{itemize}

Sviluppi futuri:
\begin{itemize}
  \item estensione funzionalità del gestionale interno;
  \item ottimizzazione continua delle landing pages;
  \item maggiore integrazione di soluzioni di intelligenza artificiale.
\end{itemize}


%----------------------------------------------------------------------------------------
% BIBLIOGRAPHY
%----------------------------------------------------------------------------------------

\backmatter

\nocite{*} % Remove this as soon as you have the first citation

\bibliographystyle{alpha}
\bibliography{bibliography}

% --- Ringraziamenti ---
\chapter{Ringraziamenti}
\begin{itemize}
\item Ringrazio il team di Datapizza, il mio referente Giuseppe Renna,
la mia famiglia e i miei amici per il costante supporto.
\end{itemize}

\end{document}
