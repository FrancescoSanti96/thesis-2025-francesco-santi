\documentclass[12pt,a4paper,openright,twoside]{book}
\usepackage[utf8]{inputenc}
\usepackage{disi-thesis}
\usepackage{code-lstlistings}
\usepackage{notes}
\usepackage{shortcuts}
\usepackage{acronym}

\school{\unibo}
\programme{DIPARTIMENTO DI INFORMATICA – SCIENZA E INGEGNERIA

Laurea in Tecnologie dei Sistemi Informatici}
\title{Fancy Title}
\author{Francesco Santi}
\date{\today}
\subject{Supervisor's course name}
\supervisor{Prof. Supervisor Here}
\cosupervisor{Dott. CoSupervisor 1}
\academicyear{2024--2025}

% Definition of acronyms
\acrodef{IoT}{Internet of Thing}
\acrodef{vm}[VM]{Virtual Machine}


\mainlinespacing{1.241} % line spacing in mainmatter, comment to default (1)

\begin{document}

% ===== FRONTESPIZIO ORIGINALE =====
\frontmatter\frontispiece

% ===== Dedica (opzionale) =====
\begin{dedication}
Dedico questo traguardo inanziutto a Maggie che in questo mio percorso mi è sempre stata vicina, inoltre famiglia e amici.
\end{dedication}

% ===== Introduzione (front matter, prima dell'indice) =====
% Se il tuo stile NON avesse \chapterWithoutNumber, usa la forma generica qui sotto:
\chapter*{Introduzione}
\addcontentsline{toc}{chapter}{Introduzione}
Questa è l'introduzione di una pagina che riassume la relazione:
scopo, periodo svolto (3 gennaio 2025 – 20 giugno 2025), ruolo in Datapizza
e una panoramica di ciò che verrà descritto nei capitoli (contesto aziendale,
contesto applicativo, obiettivi, tecnologie, progettazione, sviluppo, dispiegamento,
verifica, conclusioni).

% ===== Indice =====
\tableofcontents

\mainmatter

% Include dei capitoli
\chapter{Contesto aziendale}

\section{Descrizione dell'azienda}
Datapizza è una startup innovativa con sede legale in Via Giuseppe Ripamonti 190, 20141 Milano (MI), fondata nell'ottobre 2022 con la mission di rendere l'Italia competitiva nel settore tech attraverso soluzioni avanzate e servizi mirati.

L'azienda nasce da un contesto particolare: nel 2021, quando si parlava di Machine Learning e Deep Learning, pochi avevano compreso che questi sistemi erano già ovunque, dentro i prodotti delle Big Tech e nelle decisioni quotidiane. Un gruppo di "nerd ossessionati" - tra cui i fondatori Pierpaolo e Alessandro - aveva creato una community per unire chi voleva approfondire, imparare e costruire in questo ambito.

Datapizza si distingue per un approccio integrato unico nel panorama italiano, operando simultaneamente su quattro verticali strategici complementari:

\begin{itemize}
  \item \textbf{Tech Recruiting}: connessione tra aziende e talenti tech, con oltre 50.000 professionisti registrati. Lanciato ad aprile 2023, rappresenta il ponte tra persone tecniche e aziende che vogliono essere competitive grazie alla tecnologia
  
  \item \textbf{Tech Community}: con oltre 500.000 iscritti, costituisce la più grande community tech italiana. Punto di riferimento per notizie, tendenze e approfondimenti tecnologici, genera oltre 6 milioni di impression mensili
  
  \item \textbf{AI Engineering}: sviluppo di agenti AI specializzati e soluzioni custom per trasformare i workflow aziendali. Include un framework proprietario ("Datapizza AI") per l'orchestrazione di modelli, lanciato nel 2025
  
  \item \textbf{AI Adoption}: percorsi personalizzati di trasformazione interna per potenziare la workforce aziendale. Lanciato a maggio 2023, risponde alla necessità di guidare l'adozione dell'AI in tutta l'organizzazione
\end{itemize}

\section{Dimensioni e crescita}
L'azienda ha vissuto una crescita esponenziale in soli tre anni. Partita come community di appassionati nel 2021, è diventata società nell'ottobre 2022 con una squadra iniziale di 10-20 persone. L'arrivo di ChatGPT nel novembre 2022 ha accelerato la trasformazione, portando l'AI nelle mani di centinaia di milioni di persone.

La crescita si è articolata in due fasi corrispondenti all'evoluzione del mercato AI:
\begin{itemize}
  \item \textbf{Wave 1 (2023-2024)}: fase di sperimentazione con tool AI generici. Le aziende iniziavano a porsi domande concrete sull'utilizzo pratico di questi strumenti. Datapizza risponde lanciando Jobs (per aumentare il talento tecnico interno) e AI Adoption (per formare l'organizzazione)
  
  \item \textbf{Wave 2 (2025+)}: integrazione AI nei sistemi core aziendali. Non bastano più tool generici ma serve co-progettazione con l'azienda. Datapizza lancia AI Engineering con framework proprietario e approccio technology-first
\end{itemize}

Oggi l'azienda conta oltre 60 dipendenti in costante crescita. I numeri chiave evidenziano il successo dell'approccio integrato:
\begin{itemize}
  \item \textbf{50.000+} talenti tech registrati
  \item \textbf{500.000+} iscritti alla community
  \item \textbf{100+} aziende partner/clienti
  \item \textbf{60+} dipendenti (da iniziali 10-20)
\end{itemize}

\section{Contributi trasversali alle aree aziendali}
Durante l'esperienza in Datapizza, ho contribuito trasversalmente alle diverse aree di business, acquisendo una visione completa dell'ecosistema aziendale.

\subsection{Tech Recruiting}
Nell'area recruiting, ho lavorato attivamente su:
\begin{itemize}
  \item \textbf{Datapizza Jobs}: piattaforma lato candidati con sviluppo di feature per matching e ottimizzazioni UX
  \item \textbf{Datapizza Company}: piattaforma lato aziende per pubblicazione posizioni e gestione candidature
  \item \textbf{Technical debt reduction}: standardizzazione API calls con React Query, migrazione UI verso ShadCN per consistenza, riduzione bundle size del 15\%
\end{itemize}

\subsection{Tech Community}
Per la community, il contributo principale è stato lo sviluppo della landing page dedicata, progettata per:
\begin{itemize}
  \item Acquisizione nuovi membri attraverso value proposition chiara
  \item Iscrizione alla newsletter "Commit" con contenuti settimanali su AI
  \item Promozione eventi, hackathon (es. "Hackapizza") e iniziative community
  \item Showcase dei 500k+ iscritti come social proof
\end{itemize}

\subsection{AI Engineering e AI Adoption}
Per i verticali AI, ho sviluppato landing pages specializzate che comunicano servizi complessi a target enterprise:
\begin{itemize}
  \item \textbf{AI Engineering}: presentazione framework proprietario, showcase progetti (Copiloti Sales, HR, Legal, Customer), tone tecnico per CTO e IT Decision Makers
  \item \textbf{AI Adoption}: comunicazione percorsi upskilling con evidenze scientifiche (+40\% qualità, +25\% velocità secondo Boston Consulting Group), approccio people-first per HR e Management
\end{itemize}

\subsection{Gestionale Interno}
Ho partecipato alla fase iniziale del gestionale aziendale interno:
\begin{itemize}
  \item Setup dell'architettura base del progetto
  \item Definizione del routing e delle convenzioni di sviluppo
  \item Condivisione best practice con il team
\end{itemize}

\section{L'importanza delle landing pages performanti}
In un contesto di crescita rapida come quello di Datapizza - passata da 10-20 a 60+ persone in tre anni - disporre di landing pages performanti e scalabili è diventato strategicamente essenziale.

Come evidenziato dal CEO: "Crediamo che con un alto grado di talento tecnico (valorizzato nel modo giusto) al giorno d'oggi la tecnologia fa la differenza nel business." Questa filosofia si traduce nella necessità di una presenza web che:

\begin{itemize}
  \item \textbf{Comunichi posizionamento chiaro}: con quattro verticali molto diversi (da recruiting a AI consulting), ogni servizio necessita di un messaggio dedicato per il proprio target
  
  \item \textbf{Abiliti acquisizione efficiente}: ridurre il costo di acquisizione cliente attraverso funnel ottimizzati e conversion rate elevati
  
  \item \textbf{Supporti scalabilità del marketing}: possibilità di testare rapidamente nuove proposte di valore e campagne mirate per ciascun verticale
  
  \item \textbf{Fornisca ottimizzazione data-driven}: tracciamento preciso del comportamento utente per miglioramenti continui basati su dati reali
  
  \item \textbf{Garantisca brand consistency}: design system condiviso che assicura coerenza visiva pur mantenendo flessibilità
\end{itemize}

La necessità di innovazione digitale si è resa evidente quando l'unica landing page esistente - che accorpava genericamente community e recruiting, senza rappresentare i nuovi verticali AI - è diventata inadeguata per la complessità dell'offerta aziendale attuale.
\chapter{Contesto applicativo}

\section{Situazione iniziale delle landing pages}
Prima del progetto di redesign, la presenza web di Datapizza consisteva in un'unica landing page che accorpava genericamente community e recruiting, senza alcuna rappresentazione dei verticali AI (AI Engineering e AI Adoption).

\subsection{Problematiche riscontrate}
Questa soluzione, adeguata quando l'azienda contava 10-20 persone, presentava limitazioni critiche per un'azienda cresciuta a 60+ dipendenti con quattro verticali distinti:

\textbf{Mancanza di posizionamento}
\begin{itemize}
  \item Impossibilità di comunicare value proposition specifiche per servizi molto diversi
  \item Confusione tra target B2B enterprise (AI consulting) e B2C (candidati)
  \item Assenza totale dei nuovi verticali AI lanciati nel 2023-2025
\end{itemize}

\textbf{Scalabilità e gestione}
\begin{itemize}
  \item Architettura monolitica difficile da aggiornare
  \item Impossibilità di testare messaggi diversi per verticale
  \item Nessuna differenziazione UX per diversi customer journey
\end{itemize}

\textbf{Tracking e ottimizzazione}
\begin{itemize}
  \item Assenza di analytics strutturato per conversioni
  \item Impossibilità di tracciare comportamento utente per servizio
  \item Nessuna conformità GDPR per tracking
  \item Mancanza di dati per decisioni data-driven
\end{itemize}

\textbf{Performance e tecnico}
\begin{itemize}
  \item [DA INTEGRARE: metriche Lighthouse/Core Web Vitals pre-redesign]
  \item Design system non strutturato
  \item SEO non ottimizzato per diversi servizi
  \item Codice non modulare e difficile da mantenere
\end{itemize}

\section{Necessità di refactor}
La crescita aziendale e l'evoluzione dei servizi hanno reso necessaria una trasformazione radicale dell'architettura web.

\subsection{Drivers strategici del cambiamento}
\begin{itemize}
  \item \textbf{Crescita organizzativa}: da 10-20 a 60+ persone richiede presenza web professionale
  \item \textbf{Nuovi verticali}: AI Engineering e AI Adoption necessitano posizionamento dedicato
  \item \textbf{Differenziazione target}: servizi B2B enterprise vs B2C richiedono comunicazione specifica
  \item \textbf{Espansione commerciale}: 100+ aziende clienti necessitano canali di acquisizione ottimizzati
\end{itemize}

\subsection{Limitazioni tecniche dell'esistente}
\begin{itemize}
  \item Impossibilità di implementare tracking granulare per servizio
  \item Difficoltà nel mantenere performance ottimali con contenuti crescenti
  \item Assenza di framework per testing A/B
  \item Mancanza di separazione delle responsabilità nel codice
\end{itemize}

\section{Valore strategico del progetto}
Il progetto rappresenta un investimento strategico fondamentale per supportare la crescita aziendale:

\textbf{Impatto sul business}
\begin{itemize}
  \item Le landing pages costituiscono il principale canale di acquisizione lead
  \item Necessità di ridurre il costo di acquisizione cliente (CAC)
  \item Abilitazione di campagne marketing mirate per verticale
  \item Supporto alla strategia commerciale multi-target
\end{itemize}

\textbf{Posizionamento competitivo}
\begin{itemize}
  \item Comunicazione chiara dell'identità "AI Transformation Company"
  \item Credibilità enterprise per servizi AI ad alto valore
  \item Differenziazione attraverso community 500k+ come trust signal
  \item Professionalizzazione dell'immagine aziendale
\end{itemize}

\textbf{Scalabilità futura}
\begin{itemize}
  \item Architettura pronta per nuovi verticali di business
  \item Framework riutilizzabile per future iniziative
  \item Capacità di iterazione rapida basata su dati
  \item Manutenibilità a lungo termine del sistema
\end{itemize}
\chapter{Obiettivi}

\section{Obiettivi strategici e di business}
Il progetto di redesign delle landing pages aveva l'obiettivo di 
trasformare la presenza web aziendale da un'unica pagina generica a un 
ecosistema di landing specializzate, risolvendo le problematiche 
identificate nel capitolo precedente.

\paragraph{Posizionamento e differenziazione}
Un primo obiettivo strategico riguardava il posizionamento chiaro dei 
quattro verticali aziendali. Era necessario creare value proposition 
specifiche per AI Engineering, AI Adoption, Tech Recruiting e Community, 
separando nettamente la comunicazione verso clienti B2B enterprise da 
quella rivolta a candidati B2C. Questo avrebbe permesso di consolidare 
l'identità aziendale come "AI Transformation Company" attraverso messaggi 
mirati e coerenti per ciascun target.

\paragraph{Marketing data-driven e conversione}
Per abilitare decisioni basate sui dati concreti, era necessario 
implementare un sistema di tracking strutturato e GDPR-compliant. 
Gli obiettivi specifici in quest'area erano:

\begin{itemize}
  \item Implementare tracking avanzato per customer journey completi, 
        tracciando ogni interazione utente dall'arrivo sulla landing 
        fino alla conversione
  
  \item Creare funnel di conversione misurabili per ottimizzazione 
        continua, differenziati per ciascun verticale
  
  \item Integrare strumenti analytics (Mixpanel e Redash) per fornire 
        al team marketing dati concreti su cui basare le decisioni
  
  \item Migliorare il conversion rate complessivo attraverso 
        l'ottimizzazione iterativa basata sui dati raccolti
  
  \item Aumentare i lead qualificati B2B per i servizi enterprise e 
        le iscrizioni alla newsletter community
  
  \item Ridurre il bounce rate attraverso customer journey ottimizzati 
        per ciascun tipo di utente
\end{itemize}

\paragraph{Supporto strategia commerciale}
Le nuove landing dovevano diventare il canale primario di acquisizione 
contatti qualificati per il team sales, abilitando campagne marketing con 
tracking ROI accurato. L'obiettivo era fornire visibilità completa 
sull'efficacia di ogni verticale e permettere l'allocazione ottimale del 
budget marketing basata su dati misurabili.

\section{Obiettivi tecnici}
Gli obiettivi tecnici erano orientati a costruire un'infrastruttura 
solida, performante e scalabile per supportare gli obiettivi di business.

\subsection{Architettura e scalabilità}
L'obiettivo architetturale principale era costruire un sistema frontend 
moderno basato su un design system modulare. La scelta di ShadCN UI e 
Tailwind CSS come fondamenta permetteva di creare componenti riutilizzabili 
e mantenere consistenza visiva tra tutte le landing. L'architettura doveva 
supportare il multilingua attraverso percorsi dedicati (/it/ e /en/) e 
garantire la scalabilità futura: ogni nuova landing page doveva poter 
riutilizzare la maggior parte del codice esistente, riducendo 
significativamente i tempi di sviluppo per futuri verticali.

\subsection{Performance e user experience}
Dal punto di vista delle performance, l'obiettivo era garantire 
caricamenti rapidi attraverso tecniche di ottimizzazione come code 
splitting, lazy loading e compressione delle immagini. L'esperienza 
doveva essere fluida su tutti i dispositivi (desktop, tablet, mobile). 

Particolare attenzione era richiesta per l'accessibilità: conformità 
agli standard WCAG 2.1 livello AA per supportare utenti con disabilità 
attraverso screen reader, navigazione da tastiera e contrasto colori 
adeguato.

\subsection{Tracking e analytics}
L'obiettivo sul fronte analytics era integrare Mixpanel con una event 
taxonomy strutturata per categorizzare i comportamenti utente in tre 
categorie principali: page view (visualizzazioni), interaction 
(interazioni con elementi della pagina) e conversion (azioni di valore 
come submit form o iscrizione newsletter). Ogni verticale doveva avere 
funnel di conversione specifici e misurabili, differenziati per tipo di 
utente (B2B enterprise, B2C candidati, Community). La compliance GDPR 
era un requisito non negoziabile, garantita attraverso cookie consent 
esplicito e anonimizzazione degli indirizzi IP.
\chapter{Tecnologie}

\section{Stack tecnologico landing pages}
Il progetto di redesign delle landing pages si basa su uno stack moderno 
orientato a performance, SEO e developer experience. Le scelte tecnologiche 
sono state guidate dalla necessità di garantire caricamenti rapidi, 
indicizzazione ottimale sui motori di ricerca e scalabilità futura.

\subsection{Frontend}

Il framework principale è Next.js 15, che si basa su React 19 e TypeScript 
5.7. Queste tecnologie erano già in uso nel progetto precedente e sono 
state mantenute durante il redesign per coerenza con l'intero ecosistema 
tecnologico aziendale.

\paragraph{React 19}
React è la libreria JavaScript per costruire interfacce utente component-based, 
utilizzata in tutti i principali prodotti Datapizza (Jobs, Company, gestionale 
interno). Questa coerenza tecnologica garantisce riuso del codice, competenze 
consolidate del team e facilita l'inserimento di nuovi sviluppatori. L'approccio 
component-based permette di costruire interfacce modulari suddividendo 
l'interfaccia in componenti indipendenti e riutilizzabili, facilitando la 
manutenzione e la scalabilità del codice.

\paragraph{TypeScript 5.7}
A supporto di React è stato adottato TypeScript 5.7, un superset di 
JavaScript che aggiunge type safety statico, ormai standard per applicazioni 
moderne in contesti production. L'adozione di TypeScript riduce significativamente 
la probabilità di errori in produzione grazie al type checking in fase di 
sviluppo: errori comuni come accesso a proprietà inesistenti o passaggio di 
parametri errati vengono intercettati dal compilatore prima del deployment. 
In un contesto production come le landing pages, dove errori critici impattano 
direttamente le conversioni, TypeScript è fondamentale per garantire affidabilità.

TypeScript migliora significativamente il lavoro in team rendendo il codice più 
strutturato e auto-documentato. L'autocompletion intelligente negli IDE accelera 
lo sviluppo, mentre il supporto al refactoring permette modifiche strutturali 
sicure, riducendo il tempo di apprendimento per nuovi sviluppatori.

\paragraph{Next.js 15}
Next.js è un framework React ampiamente adottato per lo sviluppo di applicazioni 
production-ready, fornendo funzionalità avanzate di rendering e ottimizzazione. 
La scelta di Next.js rispetto a React standalone si basa su tre vantaggi 
fondamentali per il progetto.

Il rendering ibrido combina diverse strategie di generazione delle pagine: 
Server-Side Rendering (SSR) per generare HTML dinamico ad ogni richiesta, 
Static Site Generation (SSG) per pre-generare pagine statiche in fase di build, 
e Incremental Static Regeneration (ISR) che estende SSG permettendo 
l'aggiornamento incrementale di contenuti statici senza rigenerare l'intero 
sito. Questo approccio ottimizza sia i tempi di caricamento che la freschezza 
dei contenuti.

Il SEO nativo garantisce che i motori di ricerca ricevano HTML completo 
server-side, fondamentale per indicizzazione efficace e traffico organico. 
Il code splitting automatico suddivide il codice in bundle separati per 
ogni route, caricando solo lo stretto necessario per la pagina corrente 
e riducendo il tempo di primo caricamento.

Next.js supporta nativamente il routing internazionalizzato, esteso in questo 
progetto tramite la libreria next-intl per gestire i percorsi localizzati 
(/it/ e /en/). Il componente next/image ottimizza automaticamente le immagini: 
conversione in formato moderno (WebP/AVIF), caricamento differito quando 
l'immagine entra nel viewport (lazy loading), e generazione di varianti 
responsive per adattarsi alle dimensioni del dispositivo, riducendo il peso 
complessivo mantenendo qualità visiva.

\subsection{Styling e UI}

Per garantire coerenza visiva con l'identità aziendale, il sistema di styling 
si basa su Tailwind CSS integrato con la component library ShadCN UI.

Tailwind CSS è un framework CSS utility-first (approccio basato su classi 
utilitarie) che permette di costruire interfacce applicando classi predefinite 
direttamente nel markup HTML, eliminando la necessità di scrivere file CSS 
separati. A differenza dei framework tradizionali basati su componenti 
pre-stilizzati (come Bootstrap), Tailwind fornisce classi atomiche di basso 
livello che compongono lo stile finale. Durante la fase di build, solo le 
classi effettivamente utilizzate vengono compilate nel CSS finale, riducendo 
drasticamente le dimensioni del bundle e garantendo tempi di caricamento ridotti.

ShadCN UI completa l'infrastruttura di styling fornendo componenti accessibili 
basati sulle primitive Radix UI, con conformità nativa agli standard WCAG 2.1 
livello AA. A differenza delle librerie tradizionali, ShadCN UI genera i 
componenti direttamente nel progetto tramite command-line interface, garantendo 
pieno controllo sul codice senza dipendenze esterne rigide. L'integrazione con 
Tailwind permette di stilizzare i componenti con utility classes, combinando 
accessibilità e flessibilità di design.

\subsection{Librerie complementari}

Per animazioni e interattività avanzate, il progetto utilizza Framer Motion per 
animazioni fluide nelle sezioni principali (Hero sections) e Three.js con React 
Three Fiber per visualizzazioni 3D sulla landing AI Engineering. La gestione dei 
form si basa su React Hook Form per performance ottimali e Zod per validazione 
dello schema dati con supporto TypeScript integrato sui form di acquisizione lead.

Il tracking degli eventi utente è gestito tramite Mixpanel, piattaforma di 
product analytics che permette di tracciare interazioni personalizzate e 
costruire funnel di conversione differenziati per verticale. A differenza di 
strumenti generici come Google Analytics, Mixpanel offre granularità event-based: 
ogni interazione utente (click su call-to-action, scroll della pagina, apertura 
form) viene tracciata come evento discreto, permettendo analisi dettagliate del 
customer journey. L'SDK Mixpanel viene inizializzato lato client solo dopo 
consenso esplicito dell'utente tramite cookie banner, in conformità al GDPR 
mediante anonimizzazione degli indirizzi IP e gestione delle preferenze di 
privacy.

\subsection{Backend per dati dinamici}

Alcune funzionalità delle landing pages richiedono elaborazioni lato server 
che il frontend non può gestire autonomamente, come la validazione sicura dei 
form, l'integrazione con servizi esterni e l'accesso al database aziendale.

Il backend è condiviso tra tutti i prodotti Datapizza: oltre alle landing pages, 
serve le applicazioni Jobs (lato candidati), Company (lato aziende) e il CRM 
interno. Questa architettura unificata consente una manutenzione centralizzata 
e garantisce consistenza dei dati tra piattaforme.

\paragraph{Django e Python}
Il backend si basa su Django, framework web scritto in Python. Python è un 
linguaggio di programmazione ad alto livello, apprezzato per la sua semplicità 
sintattica e il vasto ecosistema di librerie, particolarmente ricco nell'ambito 
data science e intelligenza artificiale. Django fornisce un'architettura robusta 
per costruire API REST in modo rapido e sicuro, con strumenti integrati per 
autenticazione, gestione database e validazione dati.

\paragraph{PostgreSQL}
Il database utilizzato è PostgreSQL, sistema di gestione database relazionale 
open-source. PostgreSQL organizza i dati in tabelle strutturate collegate tra 
loro tramite relazioni, garantendo integrità referenziale e consistenza. Nel 
contesto del progetto, PostgreSQL gestisce le opportunità lavorative visualizzate 
sulla Jobs Platform, i dati degli utenti registrati e le informazioni aziendali.

\paragraph{Funzionalità principali}
Il backend gestisce diverse integrazioni critiche per il funzionamento delle 
landing pages. I form di contatto passano attraverso validazione lato server 
per prevenire input malevoli e garantire che solo dati validi raggiungano il 
database. L'integrazione con Customer.io, piattaforma di email automation, 
gestisce le iscrizioni alla newsletter "Commit" e l'invio di comunicazioni 
personalizzate agli oltre 500.000 iscritti della community. Il sistema fornisce 
inoltre API per il recupero in tempo reale delle opportunità lavorative 
visualizzate sulla Jobs Platform.

Deployato su infrastruttura cloud AWS, lo stack backend completo si basa su 
Django 4.x, PostgreSQL 15 e Python 3.9, per garantire scalabilità e disponibilità. 
L'infrastruttura di deployment su AWS è dettagliata nel Capitolo 7, mentre 
la Figura~\ref{fig:stack-landing} illustra l'architettura applicativa 
complessiva delle landing pages.

\begin{figure}[h!]
    \centering
    \includegraphics[width=0.52\textwidth]{chapters/figures/stack2.pdf}
    \caption{Architettura applicativa delle landing pages che mostra 
    l'interazione tra frontend e backend 
 tramite API REST.}
    \label{fig:stack-landing}
\end{figure}

\clearpage

\section{Stack tecnologico aziendale}

L'ecosistema Datapizza (Jobs, Company, gestionale, landing pages) condivide 
uno stack unificato per garantire coerenza tecnologica e riutilizzo di competenze 
tra prodotti. La Tabella~\ref{tab:stack-aziendale} fornisce una panoramica 
completa delle tecnologie adottate a livello aziendale.

\begin{table}[h]
\centering
\caption{Stack tecnologico aziendale}
\label{tab:stack-aziendale}
\begin{tabular}{|l|l|p{6.5cm}|}
\hline
\textbf{Layer} & \textbf{Tecnologie} & \textbf{Note} \\
\hline
Frontend & React 19, TypeScript 5.7 & Base comune tutti i prodotti \\
\hline
Frontend & React Query & State management server-side \\
\hline
Frontend & TanStack Router & Routing type-safe (CRM) \\
\hline
Backend & Django 4.x, PostgreSQL 15 & Stack verticale Python \\
\hline
Infrastructure & AWS (eu-south-1) & Cloud provider \\
\hline
Infrastructure & Docker, AWS ECR & Containerizzazione \\
\hline
Infrastructure & S3, CloudFront & Storage e distribuzione globale \\
\hline
Infrastructure & Lambda & Integrazioni AI serverless \\
\hline
Tools & Mixpanel & Analytics GDPR-compliant \\
\hline
Tools & Customer.io & Email automation (500k+ iscritti) \\
\hline
\end{tabular}
\end{table}

\section{Development tools e workflow}

Il team utilizza una suite integrata di strumenti per garantire efficienza, 
qualità del codice e collaborazione efficace. La Tabella~\ref{tab:dev-tools} 
riassume le principali tecnologie adottate.

\clearpage
\begin{table}[h]
\centering
\caption{Strumenti di sviluppo e workflow}
\label{tab:dev-tools}
\begin{tabular}{|l|l|p{6cm}|}
\hline
\textbf{Categoria} & \textbf{Tecnologia} & \textbf{Utilizzo} \\
\hline
IDE & VS Code + Cursor AI & Editor principale con AI assistance \\
\hline
Version Control & GitLab self-hosted & Repository privato, Merge Request \\
\hline
Project Management & Jira & Task tracking, sprint planning \\
\hline
Deployment & Docker + AWS ECR & Containerizzazione e registry \\
\hline
Communication & Discord, Google Meet & Daily standup, sync team \\
\hline
Documentation & Notion & Knowledge base, onboarding \\
\hline
\end{tabular}
\end{table}

Il workflow di sviluppo segue approccio Agile con sprint bisettimanali. Il 
codice viene sviluppato su branch Git dedicati per ogni nuova funzionalità 
(feature branches), e ogni modifica richiede Merge Request su GitLab con 
revisione del codice (code review) obbligatoria da parte di almeno un 
collega prima dell'integrazione nel branch principale. Jira traccia le user 
stories con stima della complessità in story points, mentre grafici di 
avanzamento (burndown charts) monitorano il progresso di ogni sprint. La 
documentazione tecnica è centralizzata su Notion per garantire accessibilità 
rapida a tutto il team.
\chapter{Progettazione}
Avvio del \textbf{gestionale interno} aziendale:
\begin{itemize}
  \item setup scheletro progetto;
  \item definizione routing e architettura base;
  \item condivisione convenzioni di sviluppo.
\end{itemize}
\chapter{Sviluppo}

\section{Processo di sviluppo}
Lo sviluppo delle landing pages è stato organizzato con un approccio
iterativo basato su sprint bisettimanali. Ogni sprint prevedeva
pianificazione delle priorità, sviluppo incrementale, revisione del
codice e demo finale con stakeholder per raccogliere feedback.  
Il workflow ha incluso code review obbligatoria su ogni Pull Request,
testing continuo su ambiente \textit{staging} e rilascio progressivo
per singola landing.

\section{Fasi di implementazione}
Il progetto è stato articolato in quattro fasi principali:

\begin{itemize}
  \item \textbf{Fase 1 – Architettura e design system}: setup del progetto
  Next.js con App Router, configurazione multilingua e definizione del design
  system base con Tailwind e ShadCN.
  \item \textbf{Fase 2 – Sviluppo landing pages}: implementazione progressiva
  delle sei landing, a partire dalla Home Page (validazione del design system) fino alla Jobs Platform.
  \item \textbf{Fase 3 – Tracking e ottimizzazione}: integrazione completa di
  Mixpanel, funnel di conversione per verticale, cookie consent GDPR e
  ottimizzazioni SEO e performance.
  \item \textbf{Fase 4 – Deploy e monitoring}: rilascio graduale in produzione,
  monitoraggio in tempo reale con Mixpanel e Vercel Analytics, e documentazione finale.
\end{itemize}

\section{Sviluppo delle landing}
Le landing sono state sviluppate seguendo priorità strategiche. La
\textbf{Home Page} è stata la prima, hub centrale con routing verso i verticali
e social proof aziendali.  
Sono seguite le pagine \textbf{Tech Recruiting} e \textbf{Community}, rispettivamente orientate a lead generation B2B e alla crescita della community.  
Successivamente sono state realizzate \textbf{AI Adoption} e \textbf{AI Engineering}, focalizzate sui servizi enterprise, e infine la \textbf{Jobs Platform}, progettata mobile–first con ricerca avanzata e trasparenza salariale.  

\section{Implementazione tecnica}
Dal punto di vista tecnico, lo sviluppo ha previsto tre aree principali:

\textbf{Architettura componenti} – è stata realizzata una libreria secondo i
principi Atomic Design (atoms, molecules, organisms, templates), così da
garantire consistenza e riuso.  

\textbf{Routing e rendering} – le landing sono state generate staticamente
(SSG) con revalidazione oraria (ISR), con gestione automatica di meta tags e
sitemap per l’ottimizzazione SEO.  

\textbf{Tracking e compliance} – l’SDK di Mixpanel è stato integrato con un
custom hook che assicura tracking coerente su tutti i componenti. Il sistema è
GDPR-compliant: inizializzazione solo dopo consenso esplicito, anonimizzazione
IP e retention limitata a 12 mesi.

\section{Sfide affrontate e soluzioni}
Durante lo sviluppo sono emerse alcune difficoltà rilevanti:

\textbf{Compliance GDPR} – inizialmente Mixpanel si avviava prima del consenso
utente. La soluzione è stata il caricamento lazy dopo accettazione, con hook
dedicato e anonimizzazione IP. Questo ha garantito il 100\% di conformità e un
opt-in rate del 65\%.  

\textbf{Performance con contenuti ricchi} – la landing AI Engineering, con
gallery e grafici, aveva LCP oltre 4s. Ottimizzazione immagini (WebP), lazy
loading e dynamic import hanno ridotto il tempo a 1.8s, migliorando il
Lighthouse Score a 94/100.  

\textbf{Validazione form cross-browser} – i form mostravano errori incoerenti
tra Chrome, Safari e Firefox. L’adozione di Zod e React Hook Form ha uniformato
la gestione, riducendo il tasso di errori dal 15\% al 3\%.

\section{Esempio di codice significativo}
Segue un esempio di componente \textit{Hero}, progettato per essere riutilizzato
su tutte le landing con varianti per target B2B, B2C e Community:

\begin{lstlisting}[language=JavaScript, caption=Hero Component con varianti]
// components/organisms/Hero.tsx
import { Button } from '@/components/ui/button';
import { cn } from '@/lib/utils';

type HeroVariant = 'b2b' | 'b2c' | 'community';

interface HeroProps {
  variant: HeroVariant;
  title: string;
  description: string;
  ctaPrimary: { label: string; href: string };
  ctaSecondary?: { label: string; href: string };
  image?: string;
}

export function Hero({
  variant, title, description,
  ctaPrimary, ctaSecondary, image
}: HeroProps) {
  const styles = {
    b2b: 'bg-white text-gray-900',
    b2c: 'bg-gradient-to-r from-orange-500 to-red-600 text-white',
    community: 'bg-gray-900 text-white'
  };

  return (
    <section className={cn('py-20 px-6', styles[variant])}>
      <div className="max-w-7xl mx-auto grid md:grid-cols-2 gap-12">
        <div className="flex flex-col justify-center">
          <h1 className="text-5xl font-bold mb-6">{title}</h1>
          <p className="text-xl mb-8 opacity-90">{description}</p>
          <div className="flex gap-4">
            <Button size="lg" asChild>
              <a href={ctaPrimary.href}>{ctaPrimary.label}</a>
            </Button>
            {ctaSecondary && (
              <Button size="lg" variant="outline" asChild>
                <a href={ctaSecondary.href}>{ctaSecondary.label}</a>
              </Button>
            )}
          </div>
        </div>
        {image && (
          <div className="relative h-[400px]">
            <Image src={image} alt={title} fill
                   className="object-cover rounded-lg" />
          </div>
        )}
      </div>
    </section>
  );
}
\end{lstlisting}

Questo esempio mostra l’approccio component-based riutilizzabile, con
tipizzazione TypeScript, responsive design e accessibilità garantita.

\section{Testing e qualità}
Il progetto è stato sottoposto a una strategia di quality assurance che ha
incluso test manuali su staging, validazione cross-browser e responsive,
verifica accessibilità WCAG 2.1 AA, monitoraggio performance con Lighthouse CI e
controlli regressivi su ogni landing.  
La qualità del codice è stata mantenuta tramite ESLint, Prettier, typing
rigoroso con TypeScript e code review obbligatorie.  

\bigskip
In sintesi, lo sviluppo ha tradotto la progettazione in un ecosistema di landing
pages funzionante, scalabile e conforme agli standard qualitativi, preparando il
terreno al dispiegamento e monitoraggio descritti nel capitolo successivo.

\chapter{Dispiegamento in opera}

\section{Pipeline CI/CD}
\subsection{Automazione build e deploy}
Il processo di deploy è completamente automatizzato attraverso pipeline CI/CD:
\begin{itemize}
  \item \textbf{Trigger}: Push su branch main/production
  \item \textbf{Build automatico}: Compilazione Next.js con ottimizzazioni
  \item \textbf{Testing pre-deploy}: [TODO: test automatici eseguiti - unit, integration, E2E?]
  \item \textbf{Deployment}: Rilascio automatico su ambiente target
  \item \textbf{Notifiche}: Alert su Discord/Slack per esito deploy
\end{itemize}

\subsection{Stages della pipeline}
\begin{enumerate}
  \item \textbf{Lint e Type Check}: Verifica qualità codice e TypeScript
  \item \textbf{Build}: Compilazione ottimizzata per produzione
  \item \textbf{Test}: [TODO: suite test automatizzati]
  \item \textbf{Deploy Staging}: Rilascio ambiente di test
  \item \textbf{Smoke Tests}: Verifica funzionamento base
  \item \textbf{Deploy Production}: Rilascio definitivo
\end{enumerate}

\section{Hosting e infrastruttura}
\subsection{Provider e configurazione}
\begin{itemize}
  \item \textbf{Hosting}: [TODO: Vercel/AWS/Netlify/altro]
  \item \textbf{Architettura}: [TODO: Serverless/Server-based]
  \item \textbf{CDN}: [TODO: CloudFront/Vercel Edge/Cloudflare]
  \item \textbf{Storage}: AWS S3 per asset statici (immagini, font, media)
  \item \textbf{Database}: [TODO: RDS/gestito dal provider/self-hosted]
\end{itemize}

\subsection{Configurazione dominio}
\begin{itemize}
  \item Domain: datapizza.tech
  \item Routing multilingua: /it/ e /en/
  \item SSL/TLS: Certificato HTTPS automatico
  \item [TODO: DNS configuration specifics]
\end{itemize}

\section{Gestione ambienti}
\subsection{Ambienti di sviluppo}
\textbf{Development (locale)}
\begin{itemize}
  \item Ambiente sviluppo su macchine developer
  \item Database locale o shared dev DB
  \item Hot reload per sviluppo rapido
  \item Mock API per testing isolato
\end{itemize}

\textbf{Staging}
\begin{itemize}
  \item Ambiente pre-produzione per testing
  \item Replica configurazione production
  \item Testing QA e UAT (User Acceptance Testing)
  \item URL: [TODO: staging.datapizza.tech o altro]
\end{itemize}

\textbf{Production}
\begin{itemize}
  \item Ambiente live accessibile agli utenti
  \item Performance monitoring attivo
  \item Backup automatizzati
  \item High availability configuration
\end{itemize}

\subsection{Variabili ambiente}
Gestione configurazioni sensibili:
\begin{itemize}
  \item API keys (Mixpanel, servizi esterni)
  \item Database connection strings
  \item Secret tokens e credenziali
  \item Feature flags per rollout graduale
  \item [TODO: tool per gestione secrets - Vault/AWS Secrets Manager]
\end{itemize}

\section{Strategia di deploy}
\subsection{Deploy progressivi}
\begin{itemize}
  \item \textbf{Rollout graduale}: Rilascio incrementale a percentuali utenti
  \item \textbf{Canary deployment}: [TODO: se implementato - test su subset utenti]
  \item \textbf{Blue-Green deployment}: [TODO: se usato - switch istantaneo tra versioni]
\end{itemize}

\subsection{Rollback strategy}
In caso di problemi post-deploy:
\begin{itemize}
  \item Rollback automatico se health check fallisce
  \item Possibilità di rollback manuale a versione precedente
  \item Git tag per versioni stabili
  \item [TODO: tempo medio rollback - es. < 5 minuti]
\end{itemize}

\subsection{Zero-downtime deployment}
\begin{itemize}
  \item [TODO: strategia usata per evitare downtime]
  \item Database migrations gestite separatamente
  \item Cache warming per performance immediate
\end{itemize}

\section{Automazioni e monitoring}

\subsection{Automazioni deploy}
\begin{itemize}
  \item \textbf{Cache invalidation}: Automatic CDN cache clear post-deploy
  \item \textbf{Sitemap regeneration}: Update automatico sitemap.xml
  \item \textbf{Webhook notifications}: Alert team su Discord/Slack
  \item \textbf{Analytics tracking}: Verifica corretta inizializzazione Mixpanel
\end{itemize}

\subsection{Monitoraggio post-deploy}
\textbf{Error Tracking}
\begin{itemize}
  \item [TODO: Sentry/altro tool per error monitoring]
  \item Alert real-time su errori critici
  \item Source maps per debugging in produzione
  \item Error rate threshold per automatic rollback
\end{itemize}

\textbf{Performance Monitoring}
\begin{itemize}
  \item Core Web Vitals tracking in produzione
  \item API response time monitoring
  \item [TODO: tool APM - New Relic/Datadog/altro]
  \item Dashboard Lighthouse CI per tracking performance nel tempo
\end{itemize}

\textbf{Analytics Real-Time}
\begin{itemize}
  \item Mixpanel live view per eventi post-deploy
  \item Verifica funnel di conversione funzionanti
  \item Monitoraggio traffico per landing page
  \item Alert su anomalie metriche (spike/drop improvvisi)
\end{itemize}

\subsection{Health Checks}
\begin{itemize}
  \item Endpoint /health per status check automatico
  \item Ping periodico per verifica uptime
  \item Database connection check
  \item External API dependencies check
  \item [TODO: SLA uptime target - es. 99.9\%]
\end{itemize}

\section{Sicurezza e compliance}
\begin{itemize}
  \item HTTPS enforcement su tutte le route
  \item Security headers (CSP, HSTS, X-Frame-Options)
  \item Rate limiting per protezione API
  \item GDPR compliance per tracking utenti EU
  \item [TODO: security audit tools usati]
\end{itemize}

\section{Backup e disaster recovery}
\begin{itemize}
  \item Backup automatizzati database [TODO: frequenza]
  \item Git come backup codice sorgente
  \item Asset backup su S3 con versioning
  \item [TODO: Recovery Time Objective - RTO]
  \item [TODO: Recovery Point Objective - RPO]
\end{itemize}
\chapter{Verifica sperimentale}

\section{Testing e validazione}

La fase di verifica ha coperto testing funzionale (routing multilingua, form 
submission, tracking Mixpanel), cross-browser compatibility (Chrome, Firefox, 
Safari, Edge su desktop e mobile), e usabilità con beta testing su 10 utenti 
interni (rating medio 8/10). Le modifiche iterative hanno migliorato contrasto 
testi e dimensione call-to-action mobile.

\section{Risultati performance e accessibilità}

Il redesign ha portato miglioramenti significativi su tutte le metriche Lighthouse, 
come mostrato nella Figura \ref{fig:lighthouse-comparison}. Il performance score 
è passato da 65/100 a 94/100, l'accessibilità da 78/100 a 98/100, mentre SEO e 
Best Practices hanno raggiunto 100/100. Il bundle JavaScript è stato ridotto del 
50\% (da 850KB a 420KB) tramite code splitting e cleanup dependencies.

\begin{figure}[h!]
    \centering
    \includegraphics[width=\textwidth]{chapters/figures/lighthouse-comparison.pdf}
    \caption{Confronto score Lighthouse pre e post-redesign.}
    \label{fig:lighthouse-comparison}
\end{figure}

Le Core Web Vitals hanno raggiunto tutti i target: First Contentful Paint 1.4s, 
Largest Contentful Paint 2.1s, Cumulative Layout Shift 0.04, Time to Interactive 
2.6s, Total Blocking Time 180ms (tutti in range "Good" su Google Search Console). 
L'accessibilità WCAG 2.1 livello AA è compliant al 100\% con test screen reader 
NVDA, keyboard navigation e contrasto colori verificati. La validazione SEO 
conferma meta tags ottimizzati, structured data JSON-LD e tutte le sei landing 
indicizzate correttamente entro due settimane.

\section{Tracking e analytics}

Il sistema Mixpanel ha monitorato 95.000 eventi mensili su 30 giorni post-lancio. 
La Figura \ref{fig:mixpanel-funnel} mostra i funnel di conversione per verticale.

\begin{figure}[h!]
    \centering
    \includegraphics[width=\textwidth]{chapters/figures/mixpanel-funnel.pdf}
    \caption{Funnel di conversione per verticale B2B, B2C e Community.}
    \label{fig:mixpanel-funnel}
\end{figure}

I funnel evidenziano performance differenziate: B2B (Recruiting, AI Adoption, AI 
Engineering) con 4.5\% conversion rate complessivo da 15.200 visualizzazioni a 
680 lead qualificati mensili, Community newsletter con 24\% conversion da 11.800 
visualizzazioni a 2.800 iscrizioni mensili, e Jobs Platform con 5\% application 
rate da 18.500 visualizzazioni a 920 candidature mensili. Le metriche engagement 
confermano efficacia dell'esperienza utente: tempo medio sessione 2 minuti e 45 
secondi (superiore al target di 2 minuti), bounce rate 42\% rispetto al baseline 
del 68\% (riduzione relativa del 38\%), scroll depth medio 58\%, e click-through 
rate sulle call-to-action del 19\%. Il traffico di 45.000 utenti mensili è 
prevalentemente organico (62\%) e mobile (58\%), validando l'approccio mobile-first 
del design.

\section{Problemi riscontrati e risoluzioni}

Nelle prime settimane post-lancio è emersa una problematica sui form di contatto 
delle landing B2B: il sistema registrava doppi invii causando confusione utente 
e lead duplicati nel CRM del team sales. L'analisi della root cause ha rivelato 
due problematiche correlate: il pulsante "Invia" non si disabilitava immediatamente 
permettendo click multipli durante il processing, e mancava feedback visivo chiaro 
post-submit lasciando l'utente in stato di incertezza sul completamento dell'azione.

La risoluzione in due iterazioni ha implementato disabilitazione automatica del 
pulsante al primo click e alert di conferma esplicito post-submit (Figura 
\ref{fig:form-success}) che comunica chiaramente l'avvenuto completamento e le 
tempistiche di follow-up. Il deployment completato in una settimana ha eliminato 
completamente i doppi invii.

\begin{figure}[h!]
    \centering
    \includegraphics[width=0.8\textwidth]{chapters/figures/form-success-alert.pdf}
    \caption{Alert di conferma implementato per feedback post-submit.}
    \label{fig:form-success}
\end{figure}

Un issue minore riguardava il tracking Mixpanel su Safari in modalità Private 
Browsing (circa 5\% utenti iOS) dove il blocco di localStorage impediva la 
persistenza degli eventi. La soluzione con tracking session-based è stata deployata 
in sprint successivo. Questi incident hanno consolidato best practices: testing 
con scenari stress (double-click, connettività degradata), feedback visivo 
obbligatorio fin dalla fase di design, e monitoring granulare per device type e 
browser specifici.

\section{Risultati rispetto agli obiettivi}

Gli obiettivi tecnici definiti nel Capitolo 4 sono stati raggiunti: performance 
score Lighthouse 94/100 (target superiore a 90), accessibilità WCAG 2.1 AA 
compliant al 100\%, SEO e Best Practices 100/100, e Core Web Vitals tutti in 
range "Good". L'architettura modulare implementata garantisce scalabilità con 
tempo stimato di 2-4 settimane per sviluppare nuove landing verticali riutilizzando 
componenti esistenti.

Sul fronte business, le landing generano complessivamente 4.400 lead mensili 
distribuiti tra verticali: 680 lead B2B qualificati con conversion rate del 4.5\%, 
2.800 iscrizioni newsletter Community con conversion rate del 24\%, e 920 
candidature Jobs Platform con application rate del 5\%. Il bounce rate è migliorato 
da 68\% a 42\% (riduzione relativa del 38\%) mentre il tempo medio sessione di 
2 minuti e 45 secondi supera il target di 2 minuti. I conversion rate ottenuti 
sono in linea con industry benchmarks per il settore. Il traffico totale di 45.000 
utenti mensili si distribuisce: Home 35\%, Jobs 28\%, Tech Recruiting 18\%, 
Community 12\%, verticali AI 7\%. Metriche business a lungo termine come ROI 
campagne marketing e riduzione CAC richiedono periodo di analisi più esteso dei 
30 giorni documentati.

\bigskip
Il progetto ha raggiunto gli obiettivi prefissati fornendo foundation scalabile 
per crescita futura e validando l'approccio multi-landing con tracking avanzato.
\chapter{Conclusioni e sviluppi futuri}

\section{Sintesi dei risultati}
Il progetto di redesign delle landing pages ha trasformato la presenza web di Datapizza da un'unica pagina generica a un ecosistema di 6 landing specializzate, rispondendo alle esigenze di un'azienda cresciuta da 10-20 a 60+ persone con quattro verticali distinti.

\subsection{Risultati tecnici}
\begin{itemize}
  \item \textbf{Performance}: Lighthouse score > 90, Core Web Vitals ottimizzati (LCP < 2.5s, CLS < 0.1)
  \item \textbf{Architettura scalabile}: Design system modulare con componenti riutilizzabili (ShadCN + Tailwind)
  \item \textbf{Tracking avanzato}: Mixpanel GDPR-compliant con funnel analysis per verticale
  \item \textbf{Technical debt}: Riduzione 15\% bundle size attraverso refactoring
\end{itemize}

\subsection{Impatto business}
\begin{itemize}
  \item Posizionamento chiaro come "AI Transformation Company" attraverso landing dedicate
  \item [TODO: Conversion rate improvement - es. +X\% rispetto baseline]
  \item [TODO: Lead generation quantificato - Y lead/mese]
  \item Abilitazione campagne marketing mirate per 100+ clienti enterprise
  \item Framework pronto per futuri verticali di business
\end{itemize}

\section{Bilancio dell'esperienza formativa}
L'esperienza in Datapizza ha rappresentato un percorso di crescita professionale completo, bilanciando il progetto principale (70\%) con contributi trasversali (30\%) che hanno arricchito le competenze acquisite.

\subsection{Competenze tecniche}
\textbf{Stack moderno full-stack}
\begin{itemize}
  \item Frontend: React, Next.js, TypeScript, Tailwind CSS - da setup a produzione
  \item Backend: Django, PostgreSQL, API REST - sviluppo e ottimizzazione
  \item DevOps: CI/CD, deploy automatizzati, monitoring in produzione
  \item Analytics: Mixpanel, Redash - implementazione tracking e data analysis
\end{itemize}

\textbf{Best practices professionali}
\begin{itemize}
  \item Architettura scalabile e design pattern (Atomic Design, component composition)
  \item Performance optimization (code splitting, lazy loading, CDN)
  \item Accessibilità (WCAG 2.1 AA compliance)
  \item GDPR compliance per tracking utenti EU
\end{itemize}

\subsection{Competenze trasversali}
\begin{itemize}
  \item \textbf{Metodologia Agile}: sprint planning, daily standup, retrospettive - esperienza pratica in team strutturato
  \item \textbf{Comunicazione}: collaborazione con designer (Figma handoff), product manager, stakeholder aziendali
  \item \textbf{Product mindset}: decisioni orientate a impatto business, non solo tecnico
  \item \textbf{Problem solving}: gestione autonoma problemi tecnici complessi e hotfix in produzione
  \item \textbf{Code review}: peer review e contributo a standard qualitativi del team
\end{itemize}

\subsection{Contributi oltre il progetto principale}
\begin{itemize}
  \item Technical debt reduction su piattaforme Jobs e Company
  \item Setup iniziale gestionale interno aziendale
  \item Feature development con tracciamento impatto utente
  \item Customer support continuativo per quality assurance
\end{itemize}

\section{Limiti e criticità}
\subsection{Limitazioni del lavoro svolto}
\begin{itemize}
  \item [TODO: A/B testing non implementato completamente - manca framework automatizzato]
  \item [TODO: CMS headless non integrato - contenuti ancora gestiti via codice]
  \item [TODO: Metriche conversion rate baseline incomplete - difficoltà confronto pre/post]
  \item Timeline ristretta: alcune ottimizzazioni performance posticipate per priorità business
  \item Multilingua: supporto /en/ implementato ma contenuti non completamente tradotti
\end{itemize}

\subsection{Sfide affrontate}
\begin{itemize}
  \item GDPR compliance: complessità normativa per tracking EU richiesta iterazioni multiple
  \item Cross-team coordination: allineamento con designer e product in sprint serrati
  \item Legacy code: necessità bilanciare innovazione con compatibilità sistemi esistenti
\end{itemize}

\section{Sviluppi futuri}
\subsection{Roadmap tecnica landing pages}
\textbf{Breve termine (3-6 mesi)}
\begin{itemize}
  \item Implementazione A/B testing automatizzato per ottimizzazione conversioni
  \item Integrazione CMS headless per gestione contenuti senza deploy
  \item Espansione tracking: heatmap, session recording per UX insights
  \item Performance: ulteriore ottimizzazione Time to Interactive (target < 2s)
\end{itemize}

\textbf{Medio termine (6-12 mesi)}
\begin{itemize}
  \item Personalizzazione contenuti basata su user segmentation
  \item Progressive Web App (PWA) per esperienza mobile nativa
  \item Internazionalizzazione: espansione oltre IT/EN (ES, FR, DE)
  \item AI-powered recommendations per content optimization
\end{itemize}

\subsection{Evoluzione architetturale}
\begin{itemize}
  \item Migrazione a Next.js App Router (se attualmente Pages Router)
  \item Edge computing per performance globali ottimizzate
  \item Component library pubblicata come package npm interno
  \item Storybook per documentazione design system
\end{itemize}

\subsection{Altre iniziative aziendali}
\begin{itemize}
  \item Completamento gestionale interno con moduli CRM/ERP
  \item Evoluzione piattaforme Jobs/Company con AI matching avanzato
  \item Integrazione verticali AI (Engineering/Adoption) con prodotti esistenti
\end{itemize}

\section{Riflessione personale}
L'esperienza in Datapizza ha superato le aspettative formative iniziali, offrendo l'opportunità di lavorare su un progetto strategico con impatto reale su un'azienda in rapida crescita. 

Il progetto landing pages non è stato solo un esercizio tecnico, ma un'esperienza completa di product development: dalla comprensione del problema business, alla progettazione architetturale, fino al deployment e monitoring in produzione. Lavorare in un contesto Agile strutturato, con review giornaliere e responsabilità dirette, ha accelerato la crescita professionale ben oltre quanto possibile in ambiente accademico.

La cultura aziendale orientata all'innovazione e la possibilità di contribuire attivamente a decisioni tecniche hanno permesso di sviluppare quel "product mindset" che distingue uno sviluppatore da un ingegnere del software completo. La partecipazione a code review, la gestione di hotfix in produzione, e l'interazione continua con team cross-funzionali hanno consolidato competenze tecniche e soft skill in egual misura.

Questa esperienza rappresenta una base solida per la carriera professionale futura, con competenze immediatamente spendibili nel mercato tech moderno e una comprensione profonda di come la tecnologia generi valore di business concreto.

%----------------------------------------------------------------------------------------
% BIBLIOGRAPHY
%----------------------------------------------------------------------------------------

\backmatter

\nocite{*} % Remove this as soon as you have the first citation

\bibliographystyle{alpha}
\bibliography{bibliography}

% --- Ringraziamenti ---
\chapter{Ringraziamenti}
\begin{itemize}
\item Ringrazio ...
\end{itemize}

\end{document}