\documentclass[12pt,a4paper,openright,twoside]{book}
\usepackage[utf8]{inputenc}
\usepackage{disi-thesis}
\usepackage{code-lstlistings}
\usepackage{notes}
\usepackage{shortcuts}
\usepackage{acronym}
\usepackage[none]{hyphenat}
\usepackage{multirow}

% paragraph new line
\usepackage{titlesec}
\titleformat{\paragraph}[hang]{\normalfont\normalsize\bfseries}{\theparagraph}{1em}{}
\titlespacing*{\paragraph}{0pt}{3.25ex plus 1ex minus .2ex}{1em}


\school{\unibo}
\programme{DIPARTIMENTO DI INFORMATICA – SCIENZA E INGEGNERIA

Laurea in Tecnologie dei Sistemi Informatici}
\title{Progettazione e sviluppo di un ecosistema di landing pages scalabile per una AI Transformation Company
}
\author{Francesco Santi}
\date{\today}
\subject{Progettazioe e sviluppo del software}
\supervisor{Gianluca Aguzzi}
\academicyear{2024--2025}

% Definition of acronyms
\acrodef{IoT}{Internet of Thing}
\acrodef{vm}[VM]{Virtual Machine}


\mainlinespacing{1.241} % line spacing in mainmatter, comment to default (1)

\begin{document}

% ===== FRONTESPIZIO ORIGINALE =====
\frontmatter\frontispiece

% ===== Dedica (opzionale) =====
\begin{dedication}
Dedico questo traguardo innanzitutto a Maggie che in questi anni é sempre stata al mio fianco, 
inoltre lo dedico ai miei genitori.
\end{dedication}

% ===== Introduzione (front matter, prima dell'indice) =====
% Se il tuo stile NON avesse \chapterWithoutNumber, usa la forma generica qui sotto:
\chapter*{Introduzione}
\addcontentsline{toc}{chapter}{Introduzione}

\section*{Scopo della tesi}
Lo scopo di questa tesi è documentare l'esperienza svolta presso Datapizza nel periodo compreso tra il 3 gennaio 2025 e il 20 giugno 2025, con focus sul progetto di \textbf{redesign completo delle landing pages aziendali}. Attualmente sono dipendente full-time dell'azienda nel ruolo di Software Engineer.

\section*{Il Contesto}
Datapizza è una scale-up milanese in rapida crescita che si è affermata come 
punto di riferimento nel panorama tech italiano. L'azienda opera su quattro 
verticali complementari: \textbf{Tech Recruiting} per connettere talenti e 
aziende, \textbf{Tech Community} con oltre 500.000 iscritti, \textbf{AI 
Engineering} per lo sviluppo di soluzioni custom, e \textbf{AI Adoption} per 
percorsi di trasformazione digitale. Questa struttura integrata posiziona 
Datapizza come \textbf{AI Transformation Company} nel mercato italiano. 

\section*{Ruolo e responsabilità}
Sono stato inserito nel team di prodotto come Software Engineer, con 
responsabilità prevalentemente frontend (70\%) e backend (30\%). La composizione dettagliata del team e l'organizzazione del lavoro sono approfondite nel Capitolo 1.

\section*{Il Progetto Principale}
Ho partecipato alla progettazione e sviluppo di un ecosistema di 
\textbf{6 landing pages specializzate}, ognuna progettata per un target 
specifico con posizionamento chiaro, sistema di tracking avanzato e 
architettura scalabile. L'ecosistema si articola come segue:

\begin{enumerate}
  \item Home Page - Hub centrale aziendale
  \item Tech Recruiting - Matching talenti-aziende
  \item Tech Community - Community tech italiana
  \item AI Adoption - Upskilling e trasformazione
  \item AI Engineering - Sviluppo soluzioni AI
  \item Jobs Platform - Piattaforma candidati
\end{enumerate}

Questa architettura multi-landing ha permesso di differenziare i messaggi per 
ciascun verticale aziendale, implementare funnel di conversione misurabili per 
ottimizzazione data-driven, e abilitare strategie di marketing mirato con ROI 
tracciabile. Il passaggio da una singola landing generica a sei landing 
specializzate rappresenta il cuore della trasformazione digitale documentata 
in questa tesi.

\section*{Attività complementari}
Oltre al progetto principale, la mia esperienza in Datapizza ha incluso 
contributi trasversali su diverse aree del prodotto, arricchendo la formazione 
con competenze complementari:

\begin{itemize}
  \item \textbf{Technical debt reduction}: standardizzazione API con 
        React Query, migrazione UI verso ShadCN, refactoring componenti, 
        miglioramento qualità del codice 
  \item \textbf{Gestionale interno}: setup architettura iniziale del 
        nuovo CRM aziendale, definizione routing e convenzioni sviluppo
  \item \textbf{Tech Recruiting}: sviluppo feature su Datapizza Jobs 
        (lato candidati) e Datapizza Company (lato aziende)
  \item \textbf{Customer support}: bug fixing, manutenzione e quality 
        assurance continuativa
\end{itemize}

Queste attività hanno permesso di acquisire competenze trasversali e una 
visione completa dell'ecosistema di prodotto, integrando il lavoro sul 
progetto principale con esperienze pratiche su manutenzione, evoluzione e 
supporto operativo di applicazioni in produzione.

% ===== Indice =====
\tableofcontents

\mainmatter

% Include dei capitoli
\chapter{Contesto aziendale}

\section{Descrizione dell'azienda}
Datapizza è una startup innovativa con sede legale in Via Giuseppe Ripamonti 190, 20141 Milano (MI), fondata nell'ottobre 2022 con la mission di rendere l'Italia competitiva nel settore tech attraverso soluzioni avanzate e servizi mirati.

L'azienda nasce da un contesto particolare: nel 2021, quando si parlava di Machine Learning e Deep Learning, pochi avevano compreso che questi sistemi erano già ovunque, dentro i prodotti delle Big Tech e nelle decisioni quotidiane. Un gruppo di "nerd ossessionati" - tra cui i fondatori Pierpaolo e Alessandro - aveva creato una community per unire chi voleva approfondire, imparare e costruire in questo ambito.

Datapizza si distingue per un approccio integrato unico nel panorama italiano, operando simultaneamente su quattro verticali strategici complementari:

\begin{itemize}
  \item \textbf{Tech Recruiting}: connessione tra aziende e talenti tech, con oltre 50.000 professionisti registrati. Lanciato ad aprile 2023, rappresenta il ponte tra persone tecniche e aziende che vogliono essere competitive grazie alla tecnologia
  
  \item \textbf{Tech Community}: con oltre 500.000 iscritti, costituisce la più grande community tech italiana. Punto di riferimento per notizie, tendenze e approfondimenti tecnologici, genera oltre 6 milioni di impression mensili
  
  \item \textbf{AI Engineering}: sviluppo di agenti AI specializzati e soluzioni custom per trasformare i workflow aziendali. Include un framework proprietario ("Datapizza AI") per l'orchestrazione di modelli, lanciato nel 2025
  
  \item \textbf{AI Adoption}: percorsi personalizzati di trasformazione interna per potenziare la workforce aziendale. Lanciato a maggio 2023, risponde alla necessità di guidare l'adozione dell'AI in tutta l'organizzazione
\end{itemize}

\section{Dimensioni e crescita}
L'azienda ha vissuto una crescita esponenziale in soli tre anni. Partita come community di appassionati nel 2021, è diventata società nell'ottobre 2022 con una squadra iniziale di 10-20 persone. L'arrivo di ChatGPT nel novembre 2022 ha accelerato la trasformazione, portando l'AI nelle mani di centinaia di milioni di persone.

La crescita si è articolata in due fasi corrispondenti all'evoluzione del mercato AI:
\begin{itemize}
  \item \textbf{Wave 1 (2023-2024)}: fase di sperimentazione con tool AI generici. Le aziende iniziavano a porsi domande concrete sull'utilizzo pratico di questi strumenti. Datapizza risponde lanciando Jobs (per aumentare il talento tecnico interno) e AI Adoption (per formare l'organizzazione)
  
  \item \textbf{Wave 2 (2025+)}: integrazione AI nei sistemi core aziendali. Non bastano più tool generici ma serve co-progettazione con l'azienda. Datapizza lancia AI Engineering con framework proprietario e approccio technology-first
\end{itemize}

Oggi l'azienda conta oltre 60 dipendenti in costante crescita. I numeri chiave evidenziano il successo dell'approccio integrato:
\begin{itemize}
  \item \textbf{50.000+} talenti tech registrati
  \item \textbf{500.000+} iscritti alla community
  \item \textbf{100+} aziende partner/clienti
  \item \textbf{60+} dipendenti (da iniziali 10-20)
\end{itemize}

\section{Contributi trasversali alle aree aziendali}
Durante l'esperienza in Datapizza, ho contribuito trasversalmente alle diverse aree di business, acquisendo una visione completa dell'ecosistema aziendale.

\subsection{Tech Recruiting}
Nell'area recruiting, ho lavorato attivamente su:
\begin{itemize}
  \item \textbf{Datapizza Jobs}: piattaforma lato candidati con sviluppo di feature per matching e ottimizzazioni UX
  \item \textbf{Datapizza Company}: piattaforma lato aziende per pubblicazione posizioni e gestione candidature
  \item \textbf{Technical debt reduction}: standardizzazione API calls con React Query, migrazione UI verso ShadCN per consistenza, riduzione bundle size del 15\%
\end{itemize}

\subsection{Tech Community}
Per la community, il contributo principale è stato lo sviluppo della landing page dedicata, progettata per:
\begin{itemize}
  \item Acquisizione nuovi membri attraverso value proposition chiara
  \item Iscrizione alla newsletter "Commit" con contenuti settimanali su AI
  \item Promozione eventi, hackathon (es. "Hackapizza") e iniziative community
  \item Showcase dei 500k+ iscritti come social proof
\end{itemize}

\subsection{AI Engineering e AI Adoption}
Per i verticali AI, ho sviluppato landing pages specializzate che comunicano servizi complessi a target enterprise:
\begin{itemize}
  \item \textbf{AI Engineering}: presentazione framework proprietario, showcase progetti (Copiloti Sales, HR, Legal, Customer), tone tecnico per CTO e IT Decision Makers
  \item \textbf{AI Adoption}: comunicazione percorsi upskilling con evidenze scientifiche (+40\% qualità, +25\% velocità secondo Boston Consulting Group), approccio people-first per HR e Management
\end{itemize}

\subsection{Gestionale Interno}
Ho partecipato alla fase iniziale del gestionale aziendale interno:
\begin{itemize}
  \item Setup dell'architettura base del progetto
  \item Definizione del routing e delle convenzioni di sviluppo
  \item Condivisione best practice con il team
\end{itemize}

\section{L'importanza delle landing pages performanti}
In un contesto di crescita rapida come quello di Datapizza - passata da 10-20 a 60+ persone in tre anni - disporre di landing pages performanti e scalabili è diventato strategicamente essenziale.

Come evidenziato dal CEO: "Crediamo che con un alto grado di talento tecnico (valorizzato nel modo giusto) al giorno d'oggi la tecnologia fa la differenza nel business." Questa filosofia si traduce nella necessità di una presenza web che:

\begin{itemize}
  \item \textbf{Comunichi posizionamento chiaro}: con quattro verticali molto diversi (da recruiting a AI consulting), ogni servizio necessita di un messaggio dedicato per il proprio target
  
  \item \textbf{Abiliti acquisizione efficiente}: ridurre il costo di acquisizione cliente attraverso funnel ottimizzati e conversion rate elevati
  
  \item \textbf{Supporti scalabilità del marketing}: possibilità di testare rapidamente nuove proposte di valore e campagne mirate per ciascun verticale
  
  \item \textbf{Fornisca ottimizzazione data-driven}: tracciamento preciso del comportamento utente per miglioramenti continui basati su dati reali
  
  \item \textbf{Garantisca brand consistency}: design system condiviso che assicura coerenza visiva pur mantenendo flessibilità
\end{itemize}

La necessità di innovazione digitale si è resa evidente quando l'unica landing page esistente - che accorpava genericamente community e recruiting, senza rappresentare i nuovi verticali AI - è diventata inadeguata per la complessità dell'offerta aziendale attuale.
\chapter{Contesto applicativo}

\section{Situazione iniziale delle landing pages}
Prima del progetto di redesign, la presenza web di Datapizza consisteva in un'unica landing page che accorpava genericamente community e recruiting, senza alcuna rappresentazione dei verticali AI (AI Engineering e AI Adoption).

\subsection{Problematiche riscontrate}
Questa soluzione, adeguata quando l'azienda contava 10-20 persone, presentava limitazioni critiche per un'azienda cresciuta a 60+ dipendenti con quattro verticali distinti:

\textbf{Mancanza di posizionamento}
\begin{itemize}
  \item Impossibilità di comunicare value proposition specifiche per servizi molto diversi
  \item Confusione tra target B2B enterprise (AI consulting) e B2C (candidati)
  \item Assenza totale dei nuovi verticali AI lanciati nel 2023-2025
\end{itemize}

\textbf{Scalabilità e gestione}
\begin{itemize}
  \item Architettura monolitica difficile da aggiornare
  \item Impossibilità di testare messaggi diversi per verticale
  \item Nessuna differenziazione UX per diversi customer journey
\end{itemize}

\textbf{Tracking e ottimizzazione}
\begin{itemize}
  \item Assenza di analytics strutturato per conversioni
  \item Impossibilità di tracciare comportamento utente per servizio
  \item Nessuna conformità GDPR per tracking
  \item Mancanza di dati per decisioni data-driven
\end{itemize}

\textbf{Performance e tecnico}
\begin{itemize}
  \item [DA INTEGRARE: metriche Lighthouse/Core Web Vitals pre-redesign]
  \item Design system non strutturato
  \item SEO non ottimizzato per diversi servizi
  \item Codice non modulare e difficile da mantenere
\end{itemize}

\section{Necessità di refactor}
La crescita aziendale e l'evoluzione dei servizi hanno reso necessaria una trasformazione radicale dell'architettura web.

\subsection{Drivers strategici del cambiamento}
\begin{itemize}
  \item \textbf{Crescita organizzativa}: da 10-20 a 60+ persone richiede presenza web professionale
  \item \textbf{Nuovi verticali}: AI Engineering e AI Adoption necessitano posizionamento dedicato
  \item \textbf{Differenziazione target}: servizi B2B enterprise vs B2C richiedono comunicazione specifica
  \item \textbf{Espansione commerciale}: 100+ aziende clienti necessitano canali di acquisizione ottimizzati
\end{itemize}

\subsection{Limitazioni tecniche dell'esistente}
\begin{itemize}
  \item Impossibilità di implementare tracking granulare per servizio
  \item Difficoltà nel mantenere performance ottimali con contenuti crescenti
  \item Assenza di framework per testing A/B
  \item Mancanza di separazione delle responsabilità nel codice
\end{itemize}

\section{Valore strategico del progetto}
Il progetto rappresenta un investimento strategico fondamentale per supportare la crescita aziendale:

\textbf{Impatto sul business}
\begin{itemize}
  \item Le landing pages costituiscono il principale canale di acquisizione lead
  \item Necessità di ridurre il costo di acquisizione cliente (CAC)
  \item Abilitazione di campagne marketing mirate per verticale
  \item Supporto alla strategia commerciale multi-target
\end{itemize}

\textbf{Posizionamento competitivo}
\begin{itemize}
  \item Comunicazione chiara dell'identità "AI Transformation Company"
  \item Credibilità enterprise per servizi AI ad alto valore
  \item Differenziazione attraverso community 500k+ come trust signal
  \item Professionalizzazione dell'immagine aziendale
\end{itemize}

\textbf{Scalabilità futura}
\begin{itemize}
  \item Architettura pronta per nuovi verticali di business
  \item Framework riutilizzabile per future iniziative
  \item Capacità di iterazione rapida basata su dati
  \item Manutenibilità a lungo termine del sistema
\end{itemize}
\chapter{Obiettivi}

\section{Obiettivi strategici e di business}
Il progetto di redesign delle landing pages aveva l'obiettivo di 
trasformare la presenza web aziendale da un'unica pagina generica a un 
ecosistema di landing specializzate, risolvendo le problematiche 
identificate nel capitolo precedente.

\paragraph{Posizionamento e differenziazione}
Un primo obiettivo strategico riguardava il posizionamento chiaro dei 
quattro verticali aziendali. Era necessario creare value proposition 
specifiche per AI Engineering, AI Adoption, Tech Recruiting e Community, 
separando nettamente la comunicazione verso clienti B2B enterprise da 
quella rivolta a candidati B2C. Questo avrebbe permesso di consolidare 
l'identità aziendale come "AI Transformation Company" attraverso messaggi 
mirati e coerenti per ciascun target.

\paragraph{Marketing data-driven e conversione}
Per abilitare decisioni basate sui dati concreti, era necessario 
implementare un sistema di tracking strutturato e GDPR-compliant. 
Gli obiettivi specifici in quest'area erano:

\begin{itemize}
  \item Implementare tracking avanzato per customer journey completi, 
        tracciando ogni interazione utente dall'arrivo sulla landing 
        fino alla conversione
  
  \item Creare funnel di conversione misurabili per ottimizzazione 
        continua, differenziati per ciascun verticale
  
  \item Integrare strumenti analytics (Mixpanel e Redash) per fornire 
        al team marketing dati concreti su cui basare le decisioni
  
  \item Migliorare il conversion rate complessivo attraverso 
        l'ottimizzazione iterativa basata sui dati raccolti
  
  \item Aumentare i lead qualificati B2B per i servizi enterprise e 
        le iscrizioni alla newsletter community
  
  \item Ridurre il bounce rate attraverso customer journey ottimizzati 
        per ciascun tipo di utente
\end{itemize}

\paragraph{Supporto strategia commerciale}
Le nuove landing dovevano diventare il canale primario di acquisizione 
contatti qualificati per il team sales, abilitando campagne marketing con 
tracking ROI accurato. L'obiettivo era fornire visibilità completa 
sull'efficacia di ogni verticale e permettere l'allocazione ottimale del 
budget marketing basata su dati misurabili.

\section{Obiettivi tecnici}
Gli obiettivi tecnici erano orientati a costruire un'infrastruttura 
solida, performante e scalabile per supportare gli obiettivi di business.

\subsection{Architettura e scalabilità}
L'obiettivo architetturale principale era costruire un sistema frontend 
moderno basato su un design system modulare. La scelta di ShadCN UI e 
Tailwind CSS come fondamenta permetteva di creare componenti riutilizzabili 
e mantenere consistenza visiva tra tutte le landing. L'architettura doveva 
supportare il multilingua attraverso percorsi dedicati (/it/ e /en/) e 
garantire la scalabilità futura: ogni nuova landing page doveva poter 
riutilizzare la maggior parte del codice esistente, riducendo 
significativamente i tempi di sviluppo per futuri verticali.

\subsection{Performance e user experience}
Dal punto di vista delle performance, l'obiettivo era garantire 
caricamenti rapidi attraverso tecniche di ottimizzazione come code 
splitting, lazy loading e compressione delle immagini. L'esperienza 
doveva essere fluida su tutti i dispositivi (desktop, tablet, mobile). 

Particolare attenzione era richiesta per l'accessibilità: conformità 
agli standard WCAG 2.1 livello AA per supportare utenti con disabilità 
attraverso screen reader, navigazione da tastiera e contrasto colori 
adeguato.

\subsection{Tracking e analytics}
L'obiettivo sul fronte analytics era integrare Mixpanel con una event 
taxonomy strutturata per categorizzare i comportamenti utente in tre 
categorie principali: page view (visualizzazioni), interaction 
(interazioni con elementi della pagina) e conversion (azioni di valore 
come submit form o iscrizione newsletter). Ogni verticale doveva avere 
funnel di conversione specifici e misurabili, differenziati per tipo di 
utente (B2B enterprise, B2C candidati, Community). La compliance GDPR 
era un requisito non negoziabile, garantita attraverso cookie consent 
esplicito e anonimizzazione degli indirizzi IP.
\chapter{Tecnologie}

\section{Stack tecnologico landing pages}
Il progetto di redesign delle landing pages si basa su uno stack moderno 
orientato a performance, SEO e developer experience. Le scelte tecnologiche 
sono state guidate dalla necessità di garantire caricamenti rapidi, 
indicizzazione ottimale sui motori di ricerca e scalabilità futura.

\subsection{Frontend}

Il framework principale è Next.js 15, che si basa su React 19 e TypeScript 
5.7. Queste tecnologie erano già in uso nel progetto precedente e sono 
state mantenute durante il redesign per coerenza con l'intero ecosistema 
tecnologico aziendale.

\paragraph{React 19}
React è la libreria JavaScript per costruire interfacce utente component-based, 
utilizzata in tutti i principali prodotti Datapizza (Jobs, Company, gestionale 
interno). Questa coerenza tecnologica garantisce riuso del codice, competenze 
consolidate del team e facilita l'inserimento di nuovi sviluppatori. L'approccio 
component-based permette di costruire interfacce modulari suddividendo 
l'interfaccia in componenti indipendenti e riutilizzabili, facilitando la 
manutenzione e la scalabilità del codice.

\paragraph{TypeScript 5.7}
A supporto di React è stato adottato TypeScript 5.7, un superset di 
JavaScript che aggiunge type safety statico, ormai standard per applicazioni 
moderne in contesti production. L'adozione di TypeScript riduce significativamente 
la probabilità di errori in produzione grazie al type checking in fase di 
sviluppo: errori comuni come accesso a proprietà inesistenti o passaggio di 
parametri errati vengono intercettati dal compilatore prima del deployment. 
In un contesto production come le landing pages, dove errori critici impattano 
direttamente le conversioni, TypeScript è fondamentale per garantire affidabilità.

TypeScript migliora significativamente il lavoro in team rendendo il codice più 
strutturato e auto-documentato. L'autocompletion intelligente negli IDE accelera 
lo sviluppo, mentre il supporto al refactoring permette modifiche strutturali 
sicure, riducendo il tempo di apprendimento per nuovi sviluppatori.

\paragraph{Next.js 15}
Next.js è un framework React ampiamente adottato per lo sviluppo di applicazioni 
production-ready, fornendo funzionalità avanzate di rendering e ottimizzazione. 
La scelta di Next.js rispetto a React standalone si basa su tre vantaggi 
fondamentali per il progetto.

Il rendering ibrido combina diverse strategie di generazione delle pagine: 
Server-Side Rendering (SSR) per generare HTML dinamico ad ogni richiesta, 
Static Site Generation (SSG) per pre-generare pagine statiche in fase di build, 
e Incremental Static Regeneration (ISR) che estende SSG permettendo 
l'aggiornamento incrementale di contenuti statici senza rigenerare l'intero 
sito. Questo approccio ottimizza sia i tempi di caricamento che la freschezza 
dei contenuti.

Il SEO nativo garantisce che i motori di ricerca ricevano HTML completo 
server-side, fondamentale per indicizzazione efficace e traffico organico. 
Il code splitting automatico suddivide il codice in bundle separati per 
ogni route, caricando solo lo stretto necessario per la pagina corrente 
e riducendo il tempo di primo caricamento.

Next.js supporta nativamente il routing internazionalizzato, esteso in questo 
progetto tramite la libreria next-intl per gestire i percorsi localizzati 
(/it/ e /en/). Il componente next/image ottimizza automaticamente le immagini: 
conversione in formato moderno (WebP/AVIF), caricamento differito quando 
l'immagine entra nel viewport (lazy loading), e generazione di varianti 
responsive per adattarsi alle dimensioni del dispositivo, riducendo il peso 
complessivo mantenendo qualità visiva.

\subsection{Styling e UI}

Per garantire coerenza visiva con l'identità aziendale, il sistema di styling 
si basa su Tailwind CSS integrato con la component library ShadCN UI.

Tailwind CSS è un framework CSS utility-first (approccio basato su classi 
utilitarie) che permette di costruire interfacce applicando classi predefinite 
direttamente nel markup HTML, eliminando la necessità di scrivere file CSS 
separati. A differenza dei framework tradizionali basati su componenti 
pre-stilizzati (come Bootstrap), Tailwind fornisce classi atomiche di basso 
livello che compongono lo stile finale. Durante la fase di build, solo le 
classi effettivamente utilizzate vengono compilate nel CSS finale, riducendo 
drasticamente le dimensioni del bundle e garantendo tempi di caricamento ridotti.

ShadCN UI completa l'infrastruttura di styling fornendo componenti accessibili 
basati sulle primitive Radix UI, con conformità nativa agli standard WCAG 2.1 
livello AA. A differenza delle librerie tradizionali, ShadCN UI genera i 
componenti direttamente nel progetto tramite command-line interface, garantendo 
pieno controllo sul codice senza dipendenze esterne rigide. L'integrazione con 
Tailwind permette di stilizzare i componenti con utility classes, combinando 
accessibilità e flessibilità di design.

\subsection{Librerie complementari}

Per animazioni e interattività avanzate, il progetto utilizza Framer Motion per 
animazioni fluide nelle sezioni principali (Hero sections) e Three.js con React 
Three Fiber per visualizzazioni 3D sulla landing AI Engineering. La gestione dei 
form si basa su React Hook Form per performance ottimali e Zod per validazione 
dello schema dati con supporto TypeScript integrato sui form di acquisizione lead.

Il tracking degli eventi utente è gestito tramite Mixpanel, piattaforma di 
product analytics che permette di tracciare interazioni personalizzate e 
costruire funnel di conversione differenziati per verticale. A differenza di 
strumenti generici come Google Analytics, Mixpanel offre granularità event-based: 
ogni interazione utente (click su call-to-action, scroll della pagina, apertura 
form) viene tracciata come evento discreto, permettendo analisi dettagliate del 
customer journey. L'SDK Mixpanel viene inizializzato lato client solo dopo 
consenso esplicito dell'utente tramite cookie banner, in conformità al GDPR 
mediante anonimizzazione degli indirizzi IP e gestione delle preferenze di 
privacy.

\subsection{Backend per dati dinamici}

Alcune funzionalità delle landing pages richiedono elaborazioni lato server 
che il frontend non può gestire autonomamente, come la validazione sicura dei 
form, l'integrazione con servizi esterni e l'accesso al database aziendale.

Il backend è condiviso tra tutti i prodotti Datapizza: oltre alle landing pages, 
serve le applicazioni Jobs (lato candidati), Company (lato aziende) e il CRM 
interno. Questa architettura unificata consente una manutenzione centralizzata 
e garantisce consistenza dei dati tra piattaforme.

\paragraph{Django e Python}
Il backend si basa su Django, framework web scritto in Python. Python è un 
linguaggio di programmazione ad alto livello, apprezzato per la sua semplicità 
sintattica e il vasto ecosistema di librerie, particolarmente ricco nell'ambito 
data science e intelligenza artificiale. Django fornisce un'architettura robusta 
per costruire API REST in modo rapido e sicuro, con strumenti integrati per 
autenticazione, gestione database e validazione dati.

\paragraph{PostgreSQL}
Il database utilizzato è PostgreSQL, sistema di gestione database relazionale 
open-source. PostgreSQL organizza i dati in tabelle strutturate collegate tra 
loro tramite relazioni, garantendo integrità referenziale e consistenza. Nel 
contesto del progetto, PostgreSQL gestisce le opportunità lavorative visualizzate 
sulla Jobs Platform, i dati degli utenti registrati e le informazioni aziendali.

\paragraph{Funzionalità principali}
Il backend gestisce diverse integrazioni critiche per il funzionamento delle 
landing pages. I form di contatto passano attraverso validazione lato server 
per prevenire input malevoli e garantire che solo dati validi raggiungano il 
database. L'integrazione con Customer.io, piattaforma di email automation, 
gestisce le iscrizioni alla newsletter "Commit" e l'invio di comunicazioni 
personalizzate agli oltre 500.000 iscritti della community. Il sistema fornisce 
inoltre API per il recupero in tempo reale delle opportunità lavorative 
visualizzate sulla Jobs Platform.

Deployato su infrastruttura cloud AWS, lo stack backend completo si basa su 
Django 4.x, PostgreSQL 15 e Python 3.9, per garantire scalabilità e disponibilità. 
L'infrastruttura di deployment su AWS è dettagliata nel Capitolo 7, mentre 
la Figura~\ref{fig:stack-landing} illustra l'architettura applicativa 
complessiva delle landing pages.

\begin{figure}[h!]
    \centering
    \includegraphics[width=0.52\textwidth]{chapters/figures/stack2.pdf}
    \caption{Architettura applicativa delle landing pages che mostra 
    l'interazione tra frontend e backend 
 tramite API REST.}
    \label{fig:stack-landing}
\end{figure}

\clearpage

\section{Stack tecnologico aziendale}

L'ecosistema Datapizza (Jobs, Company, gestionale, landing pages) condivide 
uno stack unificato per garantire coerenza tecnologica e riutilizzo di competenze 
tra prodotti. La Tabella~\ref{tab:stack-aziendale} fornisce una panoramica 
completa delle tecnologie adottate a livello aziendale.

\begin{table}[h]
\centering
\caption{Stack tecnologico aziendale}
\label{tab:stack-aziendale}
\begin{tabular}{|l|l|p{6.5cm}|}
\hline
\textbf{Layer} & \textbf{Tecnologie} & \textbf{Note} \\
\hline
Frontend & React 19, TypeScript 5.7 & Base comune tutti i prodotti \\
\hline
Frontend & React Query & State management server-side \\
\hline
Frontend & TanStack Router & Routing type-safe (CRM) \\
\hline
Backend & Django 4.x, PostgreSQL 15 & Stack verticale Python \\
\hline
Infrastructure & AWS (eu-south-1) & Cloud provider \\
\hline
Infrastructure & Docker, AWS ECR & Containerizzazione \\
\hline
Infrastructure & S3, CloudFront & Storage e distribuzione globale \\
\hline
Infrastructure & Lambda & Integrazioni AI serverless \\
\hline
Tools & Mixpanel & Analytics GDPR-compliant \\
\hline
Tools & Customer.io & Email automation (500k+ iscritti) \\
\hline
\end{tabular}
\end{table}

\section{Development tools e workflow}

Il team utilizza una suite integrata di strumenti per garantire efficienza, 
qualità del codice e collaborazione efficace. La Tabella~\ref{tab:dev-tools} 
riassume le principali tecnologie adottate.

\clearpage
\begin{table}[h]
\centering
\caption{Strumenti di sviluppo e workflow}
\label{tab:dev-tools}
\begin{tabular}{|l|l|p{6cm}|}
\hline
\textbf{Categoria} & \textbf{Tecnologia} & \textbf{Utilizzo} \\
\hline
IDE & VS Code + Cursor AI & Editor principale con AI assistance \\
\hline
Version Control & GitLab self-hosted & Repository privato, Merge Request \\
\hline
Project Management & Jira & Task tracking, sprint planning \\
\hline
Deployment & Docker + AWS ECR & Containerizzazione e registry \\
\hline
Communication & Discord, Google Meet & Daily standup, sync team \\
\hline
Documentation & Notion & Knowledge base, onboarding \\
\hline
\end{tabular}
\end{table}

Il workflow di sviluppo segue approccio Agile con sprint bisettimanali. Il 
codice viene sviluppato su branch Git dedicati per ogni nuova funzionalità 
(feature branches), e ogni modifica richiede Merge Request su GitLab con 
revisione del codice (code review) obbligatoria da parte di almeno un 
collega prima dell'integrazione nel branch principale. Jira traccia le user 
stories con stima della complessità in story points, mentre grafici di 
avanzamento (burndown charts) monitorano il progresso di ogni sprint. La 
documentazione tecnica è centralizzata su Notion per garantire accessibilità 
rapida a tutto il team.
\chapter{Conclusioni e sviluppi futuri}

\section{Sintesi dei risultati}
Il progetto di redesign delle landing pages ha trasformato la presenza web di Datapizza da un'unica pagina generica a un ecosistema di 6 landing specializzate, rispondendo alle esigenze di un'azienda cresciuta da 10-20 a 60+ persone con quattro verticali distinti.

\subsection{Risultati tecnici}
\begin{itemize}
  \item \textbf{Performance}: Lighthouse score > 90, Core Web Vitals ottimizzati (LCP < 2.5s, CLS < 0.1)
  \item \textbf{Architettura scalabile}: Design system modulare con componenti riutilizzabili (ShadCN + Tailwind)
  \item \textbf{Tracking avanzato}: Mixpanel GDPR-compliant con funnel analysis per verticale
  \item \textbf{Technical debt}: Riduzione 15\% bundle size attraverso refactoring
\end{itemize}

\subsection{Impatto business}
\begin{itemize}
  \item Posizionamento chiaro come "AI Transformation Company" attraverso landing dedicate
  \item [TODO: Conversion rate improvement - es. +X\% rispetto baseline]
  \item [TODO: Lead generation quantificato - Y lead/mese]
  \item Abilitazione campagne marketing mirate per 100+ clienti enterprise
  \item Framework pronto per futuri verticali di business
\end{itemize}

\section{Bilancio dell'esperienza formativa}
L'esperienza in Datapizza ha rappresentato un percorso di crescita professionale completo, bilanciando il progetto principale (70\%) con contributi trasversali (30\%) che hanno arricchito le competenze acquisite.

\subsection{Competenze tecniche}
\textbf{Stack moderno full-stack}
\begin{itemize}
  \item Frontend: React, Next.js, TypeScript, Tailwind CSS - da setup a produzione
  \item Backend: Django, PostgreSQL, API REST - sviluppo e ottimizzazione
  \item DevOps: CI/CD, deploy automatizzati, monitoring in produzione
  \item Analytics: Mixpanel, Redash - implementazione tracking e data analysis
\end{itemize}

\textbf{Best practices professionali}
\begin{itemize}
  \item Architettura scalabile e design pattern (Atomic Design, component composition)
  \item Performance optimization (code splitting, lazy loading, CDN)
  \item Accessibilità (WCAG 2.1 AA compliance)
  \item GDPR compliance per tracking utenti EU
\end{itemize}

\subsection{Competenze trasversali}
\begin{itemize}
  \item \textbf{Metodologia Agile}: sprint planning, daily standup, retrospettive - esperienza pratica in team strutturato
  \item \textbf{Comunicazione}: collaborazione con designer (Figma handoff), product manager, stakeholder aziendali
  \item \textbf{Product mindset}: decisioni orientate a impatto business, non solo tecnico
  \item \textbf{Problem solving}: gestione autonoma problemi tecnici complessi e hotfix in produzione
  \item \textbf{Code review}: peer review e contributo a standard qualitativi del team
\end{itemize}

\subsection{Contributi oltre il progetto principale}
\begin{itemize}
  \item Technical debt reduction su piattaforme Jobs e Company
  \item Setup iniziale gestionale interno aziendale
  \item Feature development con tracciamento impatto utente
  \item Customer support continuativo per quality assurance
\end{itemize}

\section{Limiti e criticità}
\subsection{Limitazioni del lavoro svolto}
\begin{itemize}
  \item [TODO: A/B testing non implementato completamente - manca framework automatizzato]
  \item [TODO: CMS headless non integrato - contenuti ancora gestiti via codice]
  \item [TODO: Metriche conversion rate baseline incomplete - difficoltà confronto pre/post]
  \item Timeline ristretta: alcune ottimizzazioni performance posticipate per priorità business
  \item Multilingua: supporto /en/ implementato ma contenuti non completamente tradotti
\end{itemize}

\subsection{Sfide affrontate}
\begin{itemize}
  \item GDPR compliance: complessità normativa per tracking EU richiesta iterazioni multiple
  \item Cross-team coordination: allineamento con designer e product in sprint serrati
  \item Legacy code: necessità bilanciare innovazione con compatibilità sistemi esistenti
\end{itemize}

\section{Sviluppi futuri}
\subsection{Roadmap tecnica landing pages}
\textbf{Breve termine (3-6 mesi)}
\begin{itemize}
  \item Implementazione A/B testing automatizzato per ottimizzazione conversioni
  \item Integrazione CMS headless per gestione contenuti senza deploy
  \item Espansione tracking: heatmap, session recording per UX insights
  \item Performance: ulteriore ottimizzazione Time to Interactive (target < 2s)
\end{itemize}

\textbf{Medio termine (6-12 mesi)}
\begin{itemize}
  \item Personalizzazione contenuti basata su user segmentation
  \item Progressive Web App (PWA) per esperienza mobile nativa
  \item Internazionalizzazione: espansione oltre IT/EN (ES, FR, DE)
  \item AI-powered recommendations per content optimization
\end{itemize}

\subsection{Evoluzione architetturale}
\begin{itemize}
  \item Migrazione a Next.js App Router (se attualmente Pages Router)
  \item Edge computing per performance globali ottimizzate
  \item Component library pubblicata come package npm interno
  \item Storybook per documentazione design system
\end{itemize}

\subsection{Altre iniziative aziendali}
\begin{itemize}
  \item Completamento gestionale interno con moduli CRM/ERP
  \item Evoluzione piattaforme Jobs/Company con AI matching avanzato
  \item Integrazione verticali AI (Engineering/Adoption) con prodotti esistenti
\end{itemize}

\section{Riflessione personale}
L'esperienza in Datapizza ha superato le aspettative formative iniziali, offrendo l'opportunità di lavorare su un progetto strategico con impatto reale su un'azienda in rapida crescita. 

Il progetto landing pages non è stato solo un esercizio tecnico, ma un'esperienza completa di product development: dalla comprensione del problema business, alla progettazione architetturale, fino al deployment e monitoring in produzione. Lavorare in un contesto Agile strutturato, con review giornaliere e responsabilità dirette, ha accelerato la crescita professionale ben oltre quanto possibile in ambiente accademico.

La cultura aziendale orientata all'innovazione e la possibilità di contribuire attivamente a decisioni tecniche hanno permesso di sviluppare quel "product mindset" che distingue uno sviluppatore da un ingegnere del software completo. La partecipazione a code review, la gestione di hotfix in produzione, e l'interazione continua con team cross-funzionali hanno consolidato competenze tecniche e soft skill in egual misura.

Questa esperienza rappresenta una base solida per la carriera professionale futura, con competenze immediatamente spendibili nel mercato tech moderno e una comprensione profonda di come la tecnologia generi valore di business concreto.

%----------------------------------------------------------------------------------------
% BIBLIOGRAPHY
%----------------------------------------------------------------------------------------

\backmatter


% --- Ringraziamenti ---
\chapter{Ringraziamenti}
\begin{itemize}
\item Ringrazio ...
\end{itemize}

\end{document}