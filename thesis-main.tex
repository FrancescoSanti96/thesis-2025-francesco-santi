\documentclass[12pt,a4paper,openright,twoside]{book}
\usepackage[utf8]{inputenc}
\usepackage{disi-thesis}
\usepackage{code-lstlistings}
\usepackage{notes}
\usepackage{shortcuts}
\usepackage{acronym}
\usepackage[none]{hyphenat}

% paragraph new line
\usepackage{titlesec}
\titleformat{\paragraph}[hang]{\normalfont\normalsize\bfseries}{\theparagraph}{1em}{}
\titlespacing*{\paragraph}{0pt}{3.25ex plus 1ex minus .2ex}{1em}


\school{\unibo}
\programme{DIPARTIMENTO DI INFORMATICA – SCIENZA E INGEGNERIA

Laurea in Tecnologie dei Sistemi Informatici}
\title{Progettazione e sviluppo di un ecosistema di landing pages scalabile per una AI Transformation Company
}
\author{Francesco Santi}
\date{\today}
\subject{Progettazioe e sviluppo del software}
\supervisor{Gianluca Aguzzi}
\academicyear{2024--2025}

% Definition of acronyms
\acrodef{IoT}{Internet of Thing}
\acrodef{vm}[VM]{Virtual Machine}


\mainlinespacing{1.241} % line spacing in mainmatter, comment to default (1)

\begin{document}

% ===== FRONTESPIZIO ORIGINALE =====
\frontmatter\frontispiece

% ===== Dedica (opzionale) =====
\begin{dedication}
Dedico questo traguardo innanzitutto a Maggie che in questi anni é sempre stata al mio fianco, 
inoltre lo dedico ai miei genitori.
\end{dedication}

% ===== Introduzione (front matter, prima dell'indice) =====
% Se il tuo stile NON avesse \chapterWithoutNumber, usa la forma generica qui sotto:
\chapter*{Introduzione}
\addcontentsline{toc}{chapter}{Introduzione}

\section*{Scopo della tesi}
Lo scopo di questa tesi è documentare l'esperienza svolta presso Datapizza nel periodo compreso tra il 3 gennaio 2025 e il 20 giugno 2025, con focus sul progetto di \textbf{redesign completo delle landing pages aziendali}. Attualmente sono dipendente full-time dell'azienda nel ruolo di Software Engineer.

\section*{Il Contesto}
Datapizza è una scale-up con sede a Milano in rapida crescita che opera principalmente su quattro verticali: Tech Recruiting, Tech Community (500k+ iscritti), AI Engineering e AI Adoption. 

\section*{Ruolo e responsabilità}
Sono stato inserito nel team di prodotto come Software Engineer, con 
responsabilità prevalentemente frontend (70\%) e backend (30\%). La composizione dettagliata del team e l'organizzazione del lavoro sono approfondite nel Capitolo 1.

\section*{Il Progetto Principale}
Ho partecipato alla progettazione e sviluppo di un ecosistema di \textbf{6 landing pages specializzate}, ognuna con posizionamento chiaro, sistema di tracking avanzato e architettura scalabile:

\begin{enumerate}
  \item Home Page - Hub centrale aziendale
  \item Tech Recruiting - Matching talenti-aziende
  \item Tech Community - Community tech italiana
  \item AI Adoption - Upskilling e trasformazione
  \item AI Engineering - Sviluppo soluzioni AI
  \item Jobs Platform - Piattaforma candidati
\end{enumerate}

Questo ha permesso di: differenziare i messaggi per target specifici, implementare tracking per ottimizzare le conversioni, e abilitare marketing mirato e misurabile.

\section*{Attività complementari}
Oltre alle landing pages, ho contribuito a:
\begin{itemize}
  \item \textbf{Technical debt reduction}: standardizzazione API con 
        React Query, migrazione UI verso ShadCN, refactoring componenti, 
        miglioramento qualità del codice 
  \item \textbf{Gestionale interno}: setup architettura iniziale del 
        nuovo CRM aziendale, definizione routing e convenzioni sviluppo
  \item \textbf{Tech Recruiting}: sviluppo feature su Datapizza Jobs 
        (lato candidati) e Datapizza Company (lato aziende)
  \item \textbf{Customer support}: bug fixing, manutenzione e quality 
        assurance continuativa
\end{itemize}

Queste attività hanno arricchito l'esperienza permettendo di acquisire 
competenze trasversali e visione completa dell'ecosistema di prodotto.

% ===== Indice =====
\tableofcontents

\mainmatter

% Include dei capitoli
\chapter{Contesto aziendale}

\section{Descrizione dell'azienda}
Datapizza è una startup innovativa con sede legale in Via Giuseppe Ripamonti 190, 20141 Milano (MI), fondata nell'ottobre 2022 con la mission di rendere l'Italia competitiva nel settore tech attraverso soluzioni avanzate e servizi mirati.

L'azienda nasce da un contesto particolare: nel 2021, quando si parlava di Machine Learning e Deep Learning, pochi avevano compreso che questi sistemi erano già ovunque, dentro i prodotti delle Big Tech e nelle decisioni quotidiane. Un gruppo di "nerd ossessionati" - tra cui i fondatori Pierpaolo e Alessandro - aveva creato una community per unire chi voleva approfondire, imparare e costruire in questo ambito.

Datapizza si distingue per un approccio integrato unico nel panorama italiano, operando simultaneamente su quattro verticali strategici complementari:

\begin{itemize}
  \item \textbf{Tech Recruiting}: connessione tra aziende e talenti tech, con oltre 50.000 professionisti registrati. Lanciato ad aprile 2023, rappresenta il ponte tra persone tecniche e aziende che vogliono essere competitive grazie alla tecnologia
  
  \item \textbf{Tech Community}: con oltre 500.000 iscritti, costituisce la più grande community tech italiana. Punto di riferimento per notizie, tendenze e approfondimenti tecnologici, genera oltre 6 milioni di impression mensili
  
  \item \textbf{AI Engineering}: sviluppo di agenti AI specializzati e soluzioni custom per trasformare i workflow aziendali. Include un framework proprietario ("Datapizza AI") per l'orchestrazione di modelli, lanciato nel 2025
  
  \item \textbf{AI Adoption}: percorsi personalizzati di trasformazione interna per potenziare la workforce aziendale. Lanciato a maggio 2023, risponde alla necessità di guidare l'adozione dell'AI in tutta l'organizzazione
\end{itemize}

\section{Dimensioni e crescita}
L'azienda ha vissuto una crescita esponenziale in soli tre anni. Partita come community di appassionati nel 2021, è diventata società nell'ottobre 2022 con una squadra iniziale di 10-20 persone. L'arrivo di ChatGPT nel novembre 2022 ha accelerato la trasformazione, portando l'AI nelle mani di centinaia di milioni di persone.

La crescita si è articolata in due fasi corrispondenti all'evoluzione del mercato AI:
\begin{itemize}
  \item \textbf{Wave 1 (2023-2024)}: fase di sperimentazione con tool AI generici. Le aziende iniziavano a porsi domande concrete sull'utilizzo pratico di questi strumenti. Datapizza risponde lanciando Jobs (per aumentare il talento tecnico interno) e AI Adoption (per formare l'organizzazione)
  
  \item \textbf{Wave 2 (2025+)}: integrazione AI nei sistemi core aziendali. Non bastano più tool generici ma serve co-progettazione con l'azienda. Datapizza lancia AI Engineering con framework proprietario e approccio technology-first
\end{itemize}

Oggi l'azienda conta oltre 60 dipendenti in costante crescita. I numeri chiave evidenziano il successo dell'approccio integrato:
\begin{itemize}
  \item \textbf{50.000+} talenti tech registrati
  \item \textbf{500.000+} iscritti alla community
  \item \textbf{100+} aziende partner/clienti
  \item \textbf{60+} dipendenti (da iniziali 10-20)
\end{itemize}

\section{Contributi trasversali alle aree aziendali}
Durante l'esperienza in Datapizza, ho contribuito trasversalmente alle diverse aree di business, acquisendo una visione completa dell'ecosistema aziendale.

\subsection{Tech Recruiting}
Nell'area recruiting, ho lavorato attivamente su:
\begin{itemize}
  \item \textbf{Datapizza Jobs}: piattaforma lato candidati con sviluppo di feature per matching e ottimizzazioni UX
  \item \textbf{Datapizza Company}: piattaforma lato aziende per pubblicazione posizioni e gestione candidature
  \item \textbf{Technical debt reduction}: standardizzazione API calls con React Query, migrazione UI verso ShadCN per consistenza, riduzione bundle size del 15\%
\end{itemize}

\subsection{Tech Community}
Per la community, il contributo principale è stato lo sviluppo della landing page dedicata, progettata per:
\begin{itemize}
  \item Acquisizione nuovi membri attraverso value proposition chiara
  \item Iscrizione alla newsletter "Commit" con contenuti settimanali su AI
  \item Promozione eventi, hackathon (es. "Hackapizza") e iniziative community
  \item Showcase dei 500k+ iscritti come social proof
\end{itemize}

\subsection{AI Engineering e AI Adoption}
Per i verticali AI, ho sviluppato landing pages specializzate che comunicano servizi complessi a target enterprise:
\begin{itemize}
  \item \textbf{AI Engineering}: presentazione framework proprietario, showcase progetti (Copiloti Sales, HR, Legal, Customer), tone tecnico per CTO e IT Decision Makers
  \item \textbf{AI Adoption}: comunicazione percorsi upskilling con evidenze scientifiche (+40\% qualità, +25\% velocità secondo Boston Consulting Group), approccio people-first per HR e Management
\end{itemize}

\subsection{Gestionale Interno}
Ho partecipato alla fase iniziale del gestionale aziendale interno:
\begin{itemize}
  \item Setup dell'architettura base del progetto
  \item Definizione del routing e delle convenzioni di sviluppo
  \item Condivisione best practice con il team
\end{itemize}

\section{L'importanza delle landing pages performanti}
In un contesto di crescita rapida come quello di Datapizza - passata da 10-20 a 60+ persone in tre anni - disporre di landing pages performanti e scalabili è diventato strategicamente essenziale.

Come evidenziato dal CEO: "Crediamo che con un alto grado di talento tecnico (valorizzato nel modo giusto) al giorno d'oggi la tecnologia fa la differenza nel business." Questa filosofia si traduce nella necessità di una presenza web che:

\begin{itemize}
  \item \textbf{Comunichi posizionamento chiaro}: con quattro verticali molto diversi (da recruiting a AI consulting), ogni servizio necessita di un messaggio dedicato per il proprio target
  
  \item \textbf{Abiliti acquisizione efficiente}: ridurre il costo di acquisizione cliente attraverso funnel ottimizzati e conversion rate elevati
  
  \item \textbf{Supporti scalabilità del marketing}: possibilità di testare rapidamente nuove proposte di valore e campagne mirate per ciascun verticale
  
  \item \textbf{Fornisca ottimizzazione data-driven}: tracciamento preciso del comportamento utente per miglioramenti continui basati su dati reali
  
  \item \textbf{Garantisca brand consistency}: design system condiviso che assicura coerenza visiva pur mantenendo flessibilità
\end{itemize}

La necessità di innovazione digitale si è resa evidente quando l'unica landing page esistente - che accorpava genericamente community e recruiting, senza rappresentare i nuovi verticali AI - è diventata inadeguata per la complessità dell'offerta aziendale attuale.
\chapter{Contesto applicativo}

\section{Situazione iniziale delle landing pages}
Prima del progetto di redesign, la presenza web di Datapizza consisteva in un'unica landing page che accorpava genericamente community e recruiting, senza alcuna rappresentazione dei verticali AI (AI Engineering e AI Adoption).

\subsection{Problematiche riscontrate}
Questa soluzione, adeguata quando l'azienda contava 10-20 persone, presentava limitazioni critiche per un'azienda cresciuta a 60+ dipendenti con quattro verticali distinti:

\textbf{Mancanza di posizionamento}
\begin{itemize}
  \item Impossibilità di comunicare value proposition specifiche per servizi molto diversi
  \item Confusione tra target B2B enterprise (AI consulting) e B2C (candidati)
  \item Assenza totale dei nuovi verticali AI lanciati nel 2023-2025
\end{itemize}

\textbf{Scalabilità e gestione}
\begin{itemize}
  \item Architettura monolitica difficile da aggiornare
  \item Impossibilità di testare messaggi diversi per verticale
  \item Nessuna differenziazione UX per diversi customer journey
\end{itemize}

\textbf{Tracking e ottimizzazione}
\begin{itemize}
  \item Assenza di analytics strutturato per conversioni
  \item Impossibilità di tracciare comportamento utente per servizio
  \item Nessuna conformità GDPR per tracking
  \item Mancanza di dati per decisioni data-driven
\end{itemize}

\textbf{Performance e tecnico}
\begin{itemize}
  \item [DA INTEGRARE: metriche Lighthouse/Core Web Vitals pre-redesign]
  \item Design system non strutturato
  \item SEO non ottimizzato per diversi servizi
  \item Codice non modulare e difficile da mantenere
\end{itemize}

\section{Necessità di refactor}
La crescita aziendale e l'evoluzione dei servizi hanno reso necessaria una trasformazione radicale dell'architettura web.

\subsection{Drivers strategici del cambiamento}
\begin{itemize}
  \item \textbf{Crescita organizzativa}: da 10-20 a 60+ persone richiede presenza web professionale
  \item \textbf{Nuovi verticali}: AI Engineering e AI Adoption necessitano posizionamento dedicato
  \item \textbf{Differenziazione target}: servizi B2B enterprise vs B2C richiedono comunicazione specifica
  \item \textbf{Espansione commerciale}: 100+ aziende clienti necessitano canali di acquisizione ottimizzati
\end{itemize}

\subsection{Limitazioni tecniche dell'esistente}
\begin{itemize}
  \item Impossibilità di implementare tracking granulare per servizio
  \item Difficoltà nel mantenere performance ottimali con contenuti crescenti
  \item Assenza di framework per testing A/B
  \item Mancanza di separazione delle responsabilità nel codice
\end{itemize}

\section{Valore strategico del progetto}
Il progetto rappresenta un investimento strategico fondamentale per supportare la crescita aziendale:

\textbf{Impatto sul business}
\begin{itemize}
  \item Le landing pages costituiscono il principale canale di acquisizione lead
  \item Necessità di ridurre il costo di acquisizione cliente (CAC)
  \item Abilitazione di campagne marketing mirate per verticale
  \item Supporto alla strategia commerciale multi-target
\end{itemize}

\textbf{Posizionamento competitivo}
\begin{itemize}
  \item Comunicazione chiara dell'identità "AI Transformation Company"
  \item Credibilità enterprise per servizi AI ad alto valore
  \item Differenziazione attraverso community 500k+ come trust signal
  \item Professionalizzazione dell'immagine aziendale
\end{itemize}

\textbf{Scalabilità futura}
\begin{itemize}
  \item Architettura pronta per nuovi verticali di business
  \item Framework riutilizzabile per future iniziative
  \item Capacità di iterazione rapida basata su dati
  \item Manutenibilità a lungo termine del sistema
\end{itemize}
\chapter{Obiettivi}

\section{Obiettivi strategici e di business}
Il progetto di redesign delle landing pages aveva l'obiettivo di 
trasformare la presenza web aziendale da un'unica pagina generica a un 
ecosistema di landing specializzate, risolvendo le problematiche 
identificate nel capitolo precedente.

\paragraph{Posizionamento e differenziazione}
Un primo obiettivo strategico riguardava il posizionamento chiaro dei 
quattro verticali aziendali. Era necessario creare value proposition 
specifiche per AI Engineering, AI Adoption, Tech Recruiting e Community, 
separando nettamente la comunicazione verso clienti B2B enterprise da 
quella rivolta a candidati B2C. Questo avrebbe permesso di consolidare 
l'identità aziendale come "AI Transformation Company" attraverso messaggi 
mirati e coerenti per ciascun target.

\paragraph{Marketing data-driven e conversione}
Per abilitare decisioni basate sui dati concreti, era necessario 
implementare un sistema di tracking strutturato e GDPR-compliant. 
Gli obiettivi specifici in quest'area erano:

\begin{itemize}
  \item Implementare tracking avanzato per customer journey completi, 
        tracciando ogni interazione utente dall'arrivo sulla landing 
        fino alla conversione
  
  \item Creare funnel di conversione misurabili per ottimizzazione 
        continua, differenziati per ciascun verticale
  
  \item Integrare strumenti analytics (Mixpanel e Redash) per fornire 
        al team marketing dati concreti su cui basare le decisioni
  
  \item Migliorare il conversion rate complessivo attraverso 
        l'ottimizzazione iterativa basata sui dati raccolti
  
  \item Aumentare i lead qualificati B2B per i servizi enterprise e 
        le iscrizioni alla newsletter community
  
  \item Ridurre il bounce rate attraverso customer journey ottimizzati 
        per ciascun tipo di utente
\end{itemize}

\paragraph{Supporto strategia commerciale}
Le nuove landing dovevano diventare il canale primario di acquisizione 
contatti qualificati per il team sales, abilitando campagne marketing con 
tracking ROI accurato. L'obiettivo era fornire visibilità completa 
sull'efficacia di ogni verticale e permettere l'allocazione ottimale del 
budget marketing basata su dati misurabili.

\section{Obiettivi tecnici}
Gli obiettivi tecnici erano orientati a costruire un'infrastruttura 
solida, performante e scalabile per supportare gli obiettivi di business.

\subsection{Architettura e scalabilità}
L'obiettivo architetturale principale era costruire un sistema frontend 
moderno basato su un design system modulare. La scelta di ShadCN UI e 
Tailwind CSS come fondamenta permetteva di creare componenti riutilizzabili 
e mantenere consistenza visiva tra tutte le landing. L'architettura doveva 
supportare il multilingua attraverso percorsi dedicati (/it/ e /en/) e 
garantire la scalabilità futura: ogni nuova landing page doveva poter 
riutilizzare la maggior parte del codice esistente, riducendo 
significativamente i tempi di sviluppo per futuri verticali.

\subsection{Performance e user experience}
Dal punto di vista delle performance, l'obiettivo era garantire 
caricamenti rapidi attraverso tecniche di ottimizzazione come code 
splitting, lazy loading e compressione delle immagini. L'esperienza 
doveva essere fluida su tutti i dispositivi (desktop, tablet, mobile). 

Particolare attenzione era richiesta per l'accessibilità: conformità 
agli standard WCAG 2.1 livello AA per supportare utenti con disabilità 
attraverso screen reader, navigazione da tastiera e contrasto colori 
adeguato.

\subsection{Tracking e analytics}
L'obiettivo sul fronte analytics era integrare Mixpanel con una event 
taxonomy strutturata per categorizzare i comportamenti utente in tre 
categorie principali: page view (visualizzazioni), interaction 
(interazioni con elementi della pagina) e conversion (azioni di valore 
come submit form o iscrizione newsletter). Ogni verticale doveva avere 
funnel di conversione specifici e misurabili, differenziati per tipo di 
utente (B2B enterprise, B2C candidati, Community). La compliance GDPR 
era un requisito non negoziabile, garantita attraverso cookie consent 
esplicito e anonimizzazione degli indirizzi IP.

%----------------------------------------------------------------------------------------
% BIBLIOGRAPHY
%----------------------------------------------------------------------------------------

\backmatter


% --- Ringraziamenti ---
\chapter{Ringraziamenti}
\begin{itemize}
\item Ringrazio ...
\end{itemize}

\end{document}