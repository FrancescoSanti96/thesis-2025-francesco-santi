\documentclass[12pt,a4paper,openright,twoside]{book}
\usepackage[utf8]{inputenc}
\usepackage{disi-thesis}
\usepackage{code-lstlistings}
\usepackage{notes}
\usepackage{shortcuts}
\usepackage{acronym}
\usepackage[none]{hyphenat}

% paragraph new line
\usepackage{titlesec}
\titleformat{\paragraph}[hang]{\normalfont\normalsize\bfseries}{\theparagraph}{1em}{}
\titlespacing*{\paragraph}{0pt}{3.25ex plus 1ex minus .2ex}{1em}


\school{\unibo}
\programme{DIPARTIMENTO DI INFORMATICA – SCIENZA E INGEGNERIA

Laurea in Tecnologie dei Sistemi Informatici}
\title{Progettazione e sviluppo di un ecosistema di landing pages scalabile per una AI Transformation Company
}
\author{Francesco Santi}
\date{\today}
\subject{Progettazioe e sviluppo del software}
\supervisor{Gianluca Aguzzi}
\academicyear{2024--2025}

% Definition of acronyms
\acrodef{IoT}{Internet of Thing}
\acrodef{vm}[VM]{Virtual Machine}


\mainlinespacing{1.241} % line spacing in mainmatter, comment to default (1)

\begin{document}

% ===== FRONTESPIZIO ORIGINALE =====
\frontmatter\frontispiece

% ===== Dedica (opzionale) =====
\begin{dedication}
Dedico questo traguardo innanzitutto a Maggie che in questi anni é sempre stata al mio fianco, 
inoltre lo dedico ai miei genitori.
\end{dedication}

% ===== Introduzione (front matter, prima dell'indice) =====
% Se il tuo stile NON avesse \chapterWithoutNumber, usa la forma generica qui sotto:
\chapter*{Introduzione}
\addcontentsline{toc}{chapter}{Introduzione}

\section*{Scopo della tesi}
Lo scopo di questa tesi è documentare l'esperienza svolta presso Datapizza nel periodo compreso tra il 3 gennaio 2025 e il 20 giugno 2025, con focus sul progetto di \textbf{redesign completo delle landing pages aziendali}. Attualmente sono dipendente full-time dell'azienda nel ruolo di Software Engineer.

\section*{Il Contesto}
Datapizza è una scale-up con sede a Milano in rapida crescita che opera principalmente su quattro verticali: Tech Recruiting, Tech Community (500k+ iscritti), AI Engineering e AI Adoption. 

\section*{Ruolo e responsabilità}
Sono stato inserito nel team di prodotto come Software Engineer, con 
responsabilità prevalentemente frontend (70\%) e backend (30\%). La composizione dettagliata del team e l'organizzazione del lavoro sono approfondite nel Capitolo 1.

\section*{Il Progetto Principale}
Ho partecipato alla progettazione e sviluppo di un ecosistema di \textbf{6 landing pages specializzate}, ognuna con posizionamento chiaro, sistema di tracking avanzato e architettura scalabile:

\begin{enumerate}
  \item Home Page - Hub centrale aziendale
  \item Tech Recruiting - Matching talenti-aziende
  \item Tech Community - Community tech italiana
  \item AI Adoption - Upskilling e trasformazione
  \item AI Engineering - Sviluppo soluzioni AI
  \item Jobs Platform - Piattaforma candidati
\end{enumerate}

Questo ha permesso di: differenziare i messaggi per target specifici, implementare tracking per ottimizzare le conversioni, e abilitare marketing mirato e misurabile.

\section*{Attività complementari}
Oltre alle landing pages, ho contribuito a:
\begin{itemize}
  \item \textbf{Technical debt reduction}: standardizzazione API con 
        React Query, migrazione UI verso ShadCN, refactoring componenti, 
        miglioramento qualità del codice 
  \item \textbf{Gestionale interno}: setup architettura iniziale del 
        nuovo CRM aziendale, definizione routing e convenzioni sviluppo
  \item \textbf{Tech Recruiting}: sviluppo feature su Datapizza Jobs 
        (lato candidati) e Datapizza Company (lato aziende)
  \item \textbf{Customer support}: bug fixing, manutenzione e quality 
        assurance continuativa
\end{itemize}

Queste attività hanno arricchito l'esperienza permettendo di acquisire 
competenze trasversali e visione completa dell'ecosistema di prodotto.

% ===== Indice =====
\tableofcontents

\mainmatter

% Include dei capitoli
\chapter{Contesto aziendale}
\sloppypar
\section{Descrizione dell'azienda}
Datapizza è una scale-up innovativa con sede legale in Via Giuseppe Ripamonti 190, 20141 Milano (MI), fondata nell'ottobre 2022 con la mission di rendere l'Italia competitiva nel settore tech attraverso soluzioni avanzate e servizi mirati.

L'azienda nasce nel 2021 e ad oggi, dopo 4 anni, conta più di 60 persone e si conferma in forte crescita.

Datapizza si distingue per un approccio integrato unico nel panorama 
italiano. Questa strategia si basa su quattro verticali strategici 
complementari che operano in sinergia:


\begin{itemize}
  \item \textbf{Tech Recruiting}: connessione tra aziende e talenti tech, con oltre 50.000 professionisti registrati. Lanciato ad aprile 2023, rappresenta il ponte tra persone tecniche e aziende che vogliono essere competitive grazie alla tecnologia.
  
  \item \textbf{Tech Community}: con oltre 500.000 iscritti, costituisce la più grande community tech italiana. Punto di riferimento per notizie, tendenze e approfondimenti tecnologici, genera oltre 6 milioni di impression mensili.
  
  \item \textbf{AI Engineering}: sviluppo di agenti AI specializzati e soluzioni custom per trasformare i workflow aziendali. Include un framework proprietario (``Datapizza AI``) per l'orchestrazione di modelli, lanciato nel 2025.
  
  \item \textbf{AI Adoption}: percorsi personalizzati di trasformazione interna per potenziare la workforce aziendale. Lanciato a maggio 2023, risponde alla necessità di guidare l'adozione dell'AI in tutta l'organizzazione.
\end{itemize}

Questa struttura integrata permette all'azienda di offrire un supporto 
completo alle organizzazioni che vogliono crescere nel tech: dal 
recruiting dei talenti giusti, alla formazione delle persone, fino allo 
sviluppo di soluzioni AI personalizzate.

\section{Dimensioni e crescita}
L'azienda ha vissuto una crescita esponenziale in soli tre anni. Partita 
come community di appassionati nel 2021, è diventata società nell'ottobre 
2022 con una squadra iniziale di 10-20 persone. L'arrivo di ChatGPT nel 
novembre 2022 ha accelerato la trasformazione, portando l'AI nelle mani 
di centinaia di milioni di persone.

\medskip 
La crescita si è articolata in due fasi corrispondenti all'evoluzione del 
mercato AI. La prima fase, che copre il periodo 2023-2024, è stata 
caratterizzata dalla sperimentazione con tool AI generici. Le aziende 
iniziavano a porsi domande concrete sull'utilizzo pratico di questi 
strumenti: chi se ne doveva occupare? Come integrarli nei processi 
esistenti? Datapizza ha risposto a queste esigenze lanciando Jobs, per 
aumentare il talento tecnico interno alle organizzazioni, e AI Adoption, 
per formare le persone e favorire l'adozione consapevole della tecnologia.

\medskip 
La seconda fase, iniziata nel 2025 e tuttora in corso, vede 
l'integrazione dell'AI nei sistemi core aziendali. Non bastano più tool 
generici: serve co-progettazione con l'azienda per creare soluzioni su 
misura che rispondano a esigenze specifiche. Per questa nuova sfida, 
Datapizza ha lanciato AI Engineering con framework proprietario e 
approccio technology-first, mantenendo sempre l'essere umano al centro 
del processo decisionale.

\section{Struttura del team}
Durante il periodo documentato (3 gennaio - 20 giugno 2025), sono stato 
inserito nel team di prodotto, responsabile dello sviluppo e dell'evoluzione 
delle piattaforme software aziendali. La struttura era multidisciplinare, 
con competenze necessarie per gestire progetti complessi dall'ideazione 
al rilascio.

\medskip 
La componente design era coperta da un UX/UI Designer, che si occupava 
della progettazione dell'esperienza utente e dei wireframe, e da un 
Product Designer che definiva i requisiti di prodotto traducendo le 
esigenze di business in specifiche tecniche. Il coordinamento tecnico era 
affidato a un Tech Lead che guidava le decisioni architetturali e 
supervisionava la qualità del codice attraverso attività di code review. 
Completavano il team un AI Engineer dedicato allo sviluppo di soluzioni 
di intelligenza artificiale e integrazione dei modelli, e quattro Software 
Engineer, me compreso, focalizzati sullo sviluppo frontend e backend delle 
applicazioni.

\medskip 
Il mio ruolo è stato quello di Software Engineer con responsabilità 
prevalentemente frontend (70\%) e secondariamente backend (30\%). Questa 
suddivisione mi ha permesso di acquisire competenze approfondite sullo 
sviluppo dell'interfaccia utente e sull'esperienza d'uso, mantenendo al 
contempo una visione completa dello stack tecnologico aziendale.
\chapter{Contesto applicativo}
L'attività ha riguardato principalmente tre prodotti software:
\begin{itemize}
  \item \textbf{Datapizza Jobs} – piattaforma recruiting lato candidati;
  \item \textbf{Datapizza Company} – piattaforma recruiting lato aziende;
  \item \textbf{Datapizza Tech} – landing pages aziendali.
\end{itemize}
\chapter{Obiettivi}

\section{Obiettivi strategici e di business}
Il progetto di redesign delle landing pages aveva l'obiettivo di 
trasformare la presenza web aziendale da un'unica pagina generica a un 
ecosistema di landing specializzate, risolvendo le problematiche 
identificate nel capitolo precedente.

\paragraph{Posizionamento e differenziazione}
Un primo obiettivo strategico riguardava il posizionamento chiaro dei 
quattro verticali aziendali. Era necessario creare value proposition 
specifiche per AI Engineering, AI Adoption, Tech Recruiting e Community, 
separando nettamente la comunicazione verso aziende clienti 
business-to-business (B2B) da quella rivolta a singoli 
candidati business-to-consumer (B2C)
Questo avrebbe permesso di consolidare l'identità aziendale come 
"AI Transformation Company" attraverso messaggi mirati e coerenti per 
ciascun target.

\paragraph{Marketing data-driven e conversione}
Per abilitare decisioni basate sui dati concreti, era necessario 
implementare un sistema di tracking strutturato e conforme al regolamento 
europeo GDPR (General Data Protection Regulation) per la protezione dei 
dati personali. Gli obiettivi specifici in quest'area erano:

\begin{itemize}
  \item Implementare tracking avanzato per customer journey completi, 
        tracciando ogni interazione utente dall'arrivo sulla landing 
        fino alla conversione.
  
  \item Creare funnel di conversione misurabili per ottimizzazione 
        continua, differenziati per ciascun verticale.
  
  \item Integrare strumenti analytics (Mixpanel e Redash) per fornire 
        al team marketing dati concreti su cui basare le decisioni.
  
  \item Migliorare il conversion rate complessivo attraverso 
        l'ottimizzazione iterativa basata sui dati raccolti.
  
  \item Aumentare i lead qualificati B2B per i servizi enterprise e 
        le iscrizioni alla newsletter community.
  
  \item Ridurre il bounce rate attraverso customer journey ottimizzati 
        per ciascun tipo di utente.
\end{itemize}

\paragraph{Supporto strategia commerciale}
Le nuove landing dovevano diventare il canale primario di acquisizione 
contatti qualificati per il team sales, abilitando campagne marketing con 
misurazione accurata del ritorno sull'investimento (ROI). L'obiettivo era 
fornire visibilità completa sull'efficacia di ogni verticale e permettere 
l'allocazione ottimale del budget marketing basata su dati misurabili.

\section{Obiettivi tecnici}
Gli obiettivi tecnici erano orientati a costruire un'infrastruttura 
solida, performante e scalabile per supportare gli obiettivi di business.

\subsection{Architettura e scalabilità}
L'obiettivo architetturale principale era costruire un sistema frontend 
moderno basato su un design system modulare. La scelta di ShadCN UI e 
Tailwind CSS come fondamenta permetteva di creare componenti riutilizzabili 
e mantenere consistenza visiva tra tutte le landing. L'architettura doveva 
supportare il multilingua attraverso percorsi dedicati (/it/ e /en/) e 
garantire la scalabilità futura: ogni nuova landing page doveva poter 
riutilizzare la maggior parte del codice esistente, riducendo 
significativamente i tempi di sviluppo per futuri verticali.

\subsection{Performance e user experience}
Dal punto di vista delle performance, l'obiettivo era garantire 
caricamenti rapidi attraverso tecniche di ottimizzazione come code 
splitting, lazy loading e compressione delle immagini. L'esperienza 
doveva essere fluida su tutti i dispositivi (desktop, tablet, mobile). 

Particolare attenzione era richiesta per l'accessibilità: conformità 
agli standard WCAG 2.1 (Web Content Accessibility Guidelines)
per supportare utenti con disabilità attraverso screen reader, navigazione 
da tastiera e contrasto colori adeguato. Questi standard internazionali 
definiscono criteri tecnici per rendere i contenuti web accessibili a 
persone con diverse tipologie di disabilità visive, uditive, motorie e 
cognitive.

\subsection{Tracking e analytics}
L'obiettivo sul fronte analytics era integrare Mixpanel con una event 
taxonomy strutturata per categorizzare i comportamenti utente in tre 
categorie principali: page view (visualizzazioni), interaction 
(interazioni con elementi della pagina) e conversion (azioni di valore 
come submit form o iscrizione newsletter). Ogni verticale doveva avere 
funnel di conversione specifici e misurabili, differenziati per tipo di 
utente (B2B enterprise, B2C candidati, Community). La conformità GDPR 
era un requisito non negoziabile, da garantire attraverso consenso 
esplicito ai cookie, anonimizzazione degli indirizzi IP e gestione 
delle preferenze privacy degli utenti.

%----------------------------------------------------------------------------------------
% BIBLIOGRAPHY
%----------------------------------------------------------------------------------------

\backmatter


% --- Ringraziamenti ---
\chapter*{Ringraziamenti}
\addcontentsline{toc}{chapter}{Ringraziamenti}

Ringrazio innanzitutto Maggie, che in questi anni è sempre stata al mio fianco...

\end{document}